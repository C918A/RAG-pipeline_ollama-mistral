\section{1. Кольцо многочленов над полем. Наибольший общий делитель. Алгоритм Евклида. Линейное выражение НОД.}

\begin{reminder}
    \textit{Кольцом} называется множество $R$ с определенными на нем бинарными операциями \textit{сложения} $+ : R \times R \to R$ и \textit{умножения} $\cdot: R \times R \rightarrow R$, удовлетворяющими следующим условиям:
    \begin{itemize}
        \item $(R, +)$ "--- абелева группа, нейтральный элемент в которой обозначается через $0$
        \item $\forall a, b, c \in R: (ab)c = a(bc)$ (ассоциативность умножения)
        \item $\forall a, b, c \in R: a(b + c) = ab + ac$ и $(a + b)c = ac + bc$ (дистрибутивность умножения относительно сложения)
    \end{itemize}
\end{reminder}

\begin{reminder}
    \textit{Полем} называется такое коммутативное кольцо $(F, +, \cdot)$, для которого выполнено равенство $F^* = F\backslash\{0\}$.
\end{reminder}

\begin{definition}
    Последовательность $(a_0, a_1, a_2,\ldots), a_i \in R$ называют \textit{финитной} если 
    $\exists N : \forall n>N \hookrightarrow a_n = 0$, т.е. если начиная с некоторого номера $N$ все значения $a_n$ равны нулю.
\end{definition}

\begin{definition}
    Пусть R -- коммутативное кольцо с единицей. \textit{Многочлен} над R -- финитная последовательность элементов A = $(a_0, a_1, a_2,\ldots), a_i \in R$. Дополнительно будем использовать обозначение $(A)_i = a_i$.
\end{definition}

\begin{definition}
    $R[x]$ -- множество многочленов над кольцом R.
\end{definition}

\begin{definition}
    Пусть $A, B \in R[x]$, тогда верны следующие свойства:
    \begin{enumerate}
        \item $(A + B)_n = (A)_n + (B)_n$,
        \item $(A \cdot B)_n = \displaystyle\sum_{i = 0}^{n}(A)_i \cdot (B)_{n-i}$,
        \item $\lambda \in R \; (\lambda A)_n = \lambda \cdot (A)_n$.
    \end{enumerate}
\end{definition}

\begin{proposition}
    Множество всех многочленов R[x] является коммутативным кольцом с 1. $1 = (1, 0, 0,\ldots)$ -- нейтральный по умножению многочлен.
\end{proposition}

\begin{definition}
    Введем обозначения $x = (0, 1, 0, 0, \ldots)$, $x^2 = (0, 0, 1, 0, \ldots)$ и т.д. Тогда многочлен $A = (a_0, a_1, a_2, \ldots)$ можно записать как $A = a_0 \cdot 1 + a_1 \cdot x + a_2 \cdot x^2 + \ldots$.
\end{definition}

\begin{definition}
    Пусть $P = (a_0, a_1, a_2, \ldots)$ -- многочлен. Последний отличный от нуля коэффициент называется \textit{старшим коэффициентом} многочлена. Номер старшего коэффициента называется степенью многочлена и обозначается как $\deg P$.
\end{definition}

\begin{note}
    Будем считать, что степень нулевого многочлена и только нулевого многочлена не определена.
\end{note}

\begin{reminder}
    Делителями нуля называются такие числа $a$ и $b$, что $a \neq 0$ и $b \neq 0$ но $a \cdot b = 0$.
\end{reminder}

\begin{definition}
    Коммутативное кольцо с единицей называется областью целостности или целостным кольцом если оно 
    не имеет делителей нуля.
\end{definition}

\begin{proposition} 
    В области целостности выполняется правило сокращения:
    $ab = ac, a \neq 0 \Rightarrow b = c$.
\end{proposition}

\begin{proof}
    $a(b-c) = 0$, $a \neq 0 \Rightarrow b-c = 0$.
\end{proof}

\begin{proposition}
    Пусть R -- коммутативное кольцо с единицей, $A, B \in R[x]$, тогда:
    \begin{enumerate}
        \item $\deg(A+B) \leq max(\deg(A), \deg(B))$,
        \item $\deg(A \cdot B) \leq \deg(A) + \deg(B)$,
        \item Если вдобавок R -- область целостности, то $\deg(AB) = \deg(A) + \deg(B)$.
    \end{enumerate}
\end{proposition}

\begin{proof}
    \begin{enumerate}
        \item Обозначим $\deg A = a$, $\deg B = b$. Пусть $n > max(a, b)$, тогда
        $(A+B)_n = (A)_n + (B)_n = 0 + 0 = 0$, а значит $\forall n > max(a, b) \Rightarrow (A+B)_n = 0$. 
        Тогда номер последнего ненулевого элемента не превосходит $max(a, b)$, а значит 
        $deg(A+B) \leqslant max(a, b)$

        \item Пусть $n > a + b$, покажем что $(AB)_n = 0$:
        
        $$(AB)_n = \sum_{i = 0}^{a}(A_i)(B_{n-i}) + \sum_{i = a+1}^{n}(A_i)(B_{n-i}) = 0 + 0 = 0$$

        В первой сумме $B_{n-i} = 0$ во всех слагаемых так как $n > a + b$, а значит $n - i > b$ для
        всех $i$ от $0$ до $a$. Во второй сумма во всех слагаемых $A_i = 0$ так как $i > a$ на всем
        диапазоне суммирования. Таким образом обе суммы равны нулю, а значит $(AB)_n = 0$.

        \item Положим $n = a + b$, тогда:
        
        $$(AB)_{n} = \sum_{i=0}^{a-1} (A)_i(B)_{n-i} + (A)_a(B)_b + \sum_{i=a+1}^{a+b} (A)_i(B)_{n-i}$$

        Аналогично предыдущему пункту первое и третье слагаемое будут нулевыми. 
        При этом $(A)_i \neq 0$ и $(B)_{n-i} = (B)_b \neq 0$,  и в силу целостности 
        $((A)_i(B)_{n-i} \neq 0)$, то есть $(AB)_{n} \neq 0$. Для больших
        чем $n$ номеров сумма будет нулевой из предыдущего пункта, а значит $\deg AB = a$.
    \end{enumerate}
\end{proof}

\begin{theorem}[о существовании деления с остатком]
    Пусть $A, B \in F[x]$, F -- поле, $B \neq 0$. Тогда:
    \begin{enumerate}
        \item Cуществуют $Q, R \in F[x]$ т.ч. $A = QB + R$, где $R = 0$ или $\deg R < \deg B$.
        \item Многочлены R и Q определены однозначно.
    \end{enumerate}
\end{theorem}

\begin{proof}~
    \begin{enumerate}
        \item Индукция по $\deg A$:
    
        Пусть $A = 0$ или $\deg A < \deg B$, тогда очевидно $A = 0 \cdot B +A$.

        Пусть теперь $\deg A \geq \deg B$, и они равны a и b соответственно. Тогда старшие 
        члены равны $HT(A) = \alpha x^a$ и $HT(B) = \beta x^b$. Подберем моном M такой что 
        $HT(A) = M \cdot HT(B)$, например $M = \frac{\alpha}{\beta} \cdot x^{a - b}$.

        Введем обозначение $A' = A - MB$, $\deg A' < \deg A$ по построению $M$. По предположению 
        $A' = Q'B + R'$, где $R' = 0$ или $\deg R' < \deg B$. Тогда 
        $A = A' + MB = Q'B + MB + R' = (Q' + M)B + R'$ -- искомое разложение.

        \item Предположим существуют два разложения $A = Q_1 B + R_1 = Q_2 B + R_2$, многочлены 
        удовлетворяют условиям. 

        $(Q_1 - Q_2)B = R_2 - R_1$. Предположим $Q_1 \neq Q_2$, тогда $\deg((Q_1 - Q_2)B) \geq \deg(B)$.
        При этом $\deg (R_2 - R_1) < \deg B$, а значит мы пришли к противоречию и $Q_1 = Q_2$ 
        и $R_1 = R_2$. 
    \end{enumerate}
\end{proof}

\begin{definition}
    $A$ делится на $B$, если существует такой многочлен $Q$ что $A = QB$. Пишут $A \vdots B$ или 
    $B \vert A$.
\end{definition}

\begin{definition}
    Пусть $f(x)$ и $g(x) \in F[x]$ -- не нулевые одновременно многочлены. Многочлен $d(x) \in F[x]$ 
    называется наибольшим общим делителем (НОД, gcd) если:
    \begin{enumerate}
        \item $d \vert f$, $d \vert g$.
        \item если $d'$ -- общий делитель $f$ и $g$, то $d' \vert d$.
    \end{enumerate}

    Иначе говоря, $\gd(f, g)$ -- такой общий делитель, который делится на любой общий делитель.
\end{definition}

\begin{theorem}[алгоритм Евклида, линейное выражение НОД]
    Пусть $f, g \in F[x]$ и $f, g$ ненулевые одновременно. Тогда существует 
    $d(x) = \gd(f, g) \in F[x]$ и, более того, существуют 
    $u(x), v(x) \in F(x)$, такие что $u(x)f(x) + v(x)g(x) = d(x)$.
\end{theorem}

\begin{proof}
    Пусть без ограничения общности $f(x) = 0$, $g(x) \neq 0$. Тогда $d(x) = g(x)$, $d = 0\cdot f + 1\cdot g$.

    Пусть теперь оба многочлена ненулевые. Тогда можно выполнить цепочку делений многочленов, где 
    на каждом новом шаге делимым и делителем будут становиться делитель и частное предыдущего деления
    соответственно. Таким образом для каждой пары НОД будет сохраняться, так как если делитель кратен 
    некоторому многочлену, то делимое и частное будут кратны ему одновременно. Первые несколько шагов:
    \begin{align*}
        f(x)   & = q_1(x)g(x) + r_1(x), \\
        g(x)   & = q_2(x)r_1(x) + r_2(x), \\
        r_1(x) & = q_3(x)r_2(x) + r_3(x). \\
    \end{align*}
    Продолжая действовать так дойдем до последних двух шагов, после которых остаток будет равен нулю.
    При делении степень остатка меньше степени делителя, а значит, в силу конечности номеров старших 
    членов начальных многочленов, в некоторый момент процесс действительно остановится:
    \begin{align*}
        r_{n-2}(x) & = q_n(x)r_{n-1}(x) + r_n(x), \\
        r_{n-1}(x) & = q_{n+1}(x)r_{n}(x).
    \end{align*}
    Получается, что $\gd(f, g)$ = $r_n$ -- последний ненулевой остаток. Проверим:
    \begin{enumerate}
        \item $r_n \vert r_{n-1}$, $r_n \vert r_{n-2}, \ldots$. 
        Продолжая подниматься наверх, получаем $r_n \vert f$, $r_n \vert g$
        \item Теперь будем спускаться вниз, пусть $d' \vert f$, $d' \vert g$. 
        Таким образом мы дойдем до $d' \vert d$.
    \end{enumerate}
    Покажем, что все остатки $r_1, r_2, \ldots, r_n$ являются линейными комбинациями $f$ и $g$:
    $$r_1 = f - q_1g$$
    $$r_2 = g - q_2r_1 = -q_2f + (1 + q_1q_2)g$$
    Спускаясь вниз и подставляя выражения предыдущих остатков в последующие, получим все разложения.
    Положим $r_{n-2} = u''f + v''g$ и $r_{n-1} = u'f + v'g$. Тогда:
    $$d = r_n = r_{n-2} - q_n r_{n-1} = f(u'' - u'q_n) + g(v'' - v'g_n).$$
    Таким образом, все остатки можно выразить через $f$ и $g$.
\end{proof}

\section{2. Неприводимость многочленов. Основная теорема арифметики для многочленов.}

\begin{definition}
    Многочлены $A$ и $B$ называются \textit{ассоциироваными} если $B \vert A$ и $A \vert B$, то есть когда верны 
    представления $A = Q_1 B$, $B = Q_2 A$. При этом:
    \begin{eqnarray*}
        \deg A = \deg Q_1 + \deg B \geq \deg B, \\
        \deg B = \deg Q_2 + \deg A \geq \deg A,
    \end{eqnarray*}
    откуда $\deg A = \deg B$, $\deg Q_1 = \deg Q_2 = 0$.
\end{definition}

\begin{definition}
    Многочлен $P \in F[x]$ степени больше нуля называется \textit{неприводимым} над полем $F$, если из $P = AB$ 
    следует $\deg A = 0$ или $\deg B = 0$.

    Иначе говоря многочлен называется неприводимым над полем F, если его нельзя разложить в 
    произведение двух многочленов более низких степеней из этого же кольца $F[x]$.
\end{definition}

\begin{note}
    Важно над каким полем многочлен является неприводимым, например многочлен $x^2 + 1$ является 
    приводимым над полем комплексных чисел $\mathbb{C}$, но неприводимым над полем действительных 
    чисел $\mathbb{R}$.
\end{note}

\begin{proposition}
    Пусть $F$ "--- поле, $P, Q, R \in F[x]$, многочлен $P$ неприводим и выполнено $P\mid QR$. Тогда $P\mid Q$ или $P\mid R$.
\end{proposition}

\begin{proof}
    Предположим, что $P\nmid Q$. Тогда, в силу неприводимости многочлена $P$, выполнено равенство $\gd(P, Q) = 1$, поэтому существуют многочлены $K, L \in F[x]$ такие, что $KP + LQ = 1$. Умножая обе части равенства на $R$, получим, что $KPR + LQR = R$, откуда $P \mid KPR + LQR = R$.
\end{proof}

\begin{theorem} [основная теорема арифметики для многочлена]
    Пусть $F$ -- поле, $A \in F[x], A \neq 0$. Тогда верны следующие утверждения:
    \begin{enumerate}
        \item Существует разложение $A$ на неприводимые: $$A = \alpha P_1P_2 \dots P_n,$$ 
        где $\alpha \in F^*$, $P_i$ неприводимый над $F$ многочлен.
        \item Пусть A представляется в виде неприводимых многочленов двумя различными способами: 
        $$A = \alpha \cdot P_1P_2 \dots P_n = \beta \cdot Q_1Q_2 \dots Q_m,$$ 
        где $\beta\in F^*$, $Q_j$ -- неприводимый над $F$ многочлен. 

        Тогда $n = m$ и существует перестановка $\sigma\in S_n$ такая, что многочлены ассоциированы: 
        \begin{gather*}
            P_i \thicksim Q_j \, (1 \leq i \leq n), \text{ где } j = \sigma(i).
        \end{gather*}
    \end{enumerate}
\end{theorem}

\begin{proof}~
    \begin{enumerate}
        \item Докажем существование разложения на неприводимые множители индукцией по $\deg A$.
        Если $deg A = 0$, то $A = \alpha$, $\alpha \in F^*$.
        Пусть $A$ неприводим над $F$, $\deg A \geq 1$. Будем считать, что в таком случае разложение получено ($A = P$).
        Пусть теперь $A$ приводим над $F$, тогда его можно представить в виде $A = B \cdot C$, $B, C \in F[x]$, $\deg B < \deg A$, $\deg C < \deg A$. Тогда  к $B$ и $C$ применимо предположение индукции и, перемножая их, получим разложение для $A$.
        \item (Индукция по n)
        Пусть $A = \alpha P_1P_2 \dots P_n = \beta Q_1Q_2...Q_m$. Тогда произведение $Q_1Q_2 \dots Q_m \vdots P_n$, следовательно, $\exists j: Q_j \vdots P_n$ и $\exists \gamma \in F^*: Q_j = \gamma P_n$ ($P_n$ и $Q_j$ неприводимы). Теперь можно подставить выражение для $Q_j$ в представление для $A$ справа и сокрaтить $P_n$ в обеих частях (корректно, так как кольцо многочленов является областью целостности). По предположению индукции число множителей слева и справа после сокращения совпадает и существует биекция $\sigma$ между множествами $\{ 1, \dots , n - 1\} \longrightarrow \{ 1, \dots , j - 1, j + 1, \dots , n \}$, так что $P_i \thicksim Q_l$, где $l = \sigma(i)$. Доопределим биекцию $\sigma: \sigma(n) = j, \sigma \in S_n$. Теперь $\sigma$ удовлетворяет всем условиям, поэтому доказано для $n$.
    \end{enumerate}
\end{proof}

\section{3. Корни многочленов. Теорема Безу. Формальная производная. Кратные корни.}

\begin{definition}
    Элемент поля $F$ является \textit{корнем} многочлена $f$, если $f(c) = 0$.
\end{definition}

\begin{theorem}[Безу]
    Скаляр $a \in F$ является корнем многочлена $P \in F[x]$ $\Leftrightarrow$ $(x - a)\mid P$.
\end{theorem}

\begin{proof}
    Разделим $P$ с остатком на $(x - a)$, то есть выберем $Q, R \in F[x]$ такие, что $P = Q(x - a) \hm{+} R$ и $\deg{R} \ge 0$. Заметим, что $P(a) = R$, тогда выполнены равносильности $P(a) = 0 \Leftrightarrow R = 0 \hm{\Leftrightarrow} (x - a)\mid P$.
\end{proof}

\begin{definition}
    \textit{Формальной производной} многочлена $x^n$ называется $\frac{d}{dx} x^n = n \cdot x^{n-1}$, так же используется обозначение $(x^n)'$. Распространим $\frac{d}{dx}$ на остальные векторы $F[x]$ по линейности. Тогда дифференцирование является линейным опреатором: $\frac{d}{dx}: F[x] \to F[x]$.
\end{definition}

\begin{proposition}
    Формальная производная $\frac{d}{dx}$ удовлетворяет правилу Лейбница: 
    $$(f \cdot g)' = f' \cdot g + f \cdot g'.$$
\end{proposition}

\begin{proof}
    Обе части являются линейными по многочленам $f$ и $g$, поэтому достаточно доказать правило для базисных векторов. Рассмотрим $f = x^m$ и $g = x^l$ -- базисные вектора в $F[x]$. Тогда $(x^m \cdot x^l)' = (x^{m+l})' = (m+l) \cdot x^{m+l-1}$. Так же можно продифференцировать $f$ и $g$ по отдельности: $(x^m)' = m \cdot x^{m-1}$ и $(x^l)' = l \cdot x^{l-1}$. Отсюда очевидно, что равенство действительно выполняется.
\end{proof}

\begin{definition}
    Пусть задан многочлен $f \in F[x]$. Корень $c \in F$ называется корнем многочлена $f$ кратности $k$ ($k\in \N$) если $f(x)$ кратно $(x-c)^k$, но $f(x)$ не кратно $(x-c)^{k+1}$.
\end{definition}

\begin{theorem}[о кратности корня]
    Пусть $F$ -- поле, $f \in F[x]$, $c \in F$ -- корень многочлена $f$. Тогда верно следующее:
    \begin{enumerate}
        \item c -- кратный корень f $\Leftrightarrow$ $f(c) = 0$ и $f'(c) = 0$.
        \item с -- корень кратности $R$ $\Rightarrow$ $f(c) = 0$, $f'(c) = 0$, $\dots$, $f^{(R-1)}(c) = 0$.
    \end{enumerate}
\end{theorem}

\begin{proof}~
    \begin{enumerate}
        \item \begin{enumerate}
            \item Необходимость.
            
            По условию $f(x) = q(x) (x-c)$. Продифференцируем $f$:
            $$f'(x) = q'(x) (x-c) + q(x).$$ 
            Тогда $f'(c) = q(c)$. При этом многочлен $q(x)$ кратен $(x-c)$ в силу того, что $c$ - кратный корень $f$. 
            Таким образом вся производная $f'$ кратна $(x-c)$. 
            
            \item Достаточность.
            
            Пусть $f(c) = f'(c) = 0$, тогда $q(c) = 0$, а значит $q(x)$ кратен $(x-c)$.
            
        \end{enumerate}

        \item Пусть $c$ -- корень кратности $R$. Тогда многочлен $f$ представим в виде $f = q(x) (x-c)^R$, где $q(c) \neq 0$.
        Возьмем производную от $f$:
        $$f'(x) = q'(x) (x-c)^R + R \cdot q(x) (x-c)^{R-1}.$$ 
        Продолжим брать производные. Тогда для 
        k-производной кратность корня $c$ не меньше $R-k$.
    \end{enumerate}
\end{proof}

\section{4. Лемма Даламбера. Основная теорема алгебры (схема доказательства).}

\begin{reminder}
    $e^{i\phi} = \cos(\phi) + i\sin(\phi)$.
\end{reminder}

\begin{lemma}[Д'Аламбера]
    \label{lemma6}
    Пусть $f(x)$ -- многочлен положительной степени из кольца $\Cm[z]$ и $f(z_0) \neq 0$. Тогда $\forall U_{\varepsilon}(z_0)$ найдется $z \in U_{\varepsilon}(z_0)$, такое что $|f(z)| < |f(z_0)|$.
\end{lemma}

\begin{proof}
    Зафиксируем $\varepsilon > 0$. Разделим $f(z)$ на $z - z_0$ с остатком: $f(z) = q_1(z) (z-z_0) + r_0$. Мы знаем, что остаток имеет смысл значения многочлена в точке $z_0$, то есть $r_0 = f(z_0)$. Разделим теперь $q_1$ на $z - z_0$ и подставим полученное выражение в выражение для $f$ (здесь $r_1 = q_1(z_0)$). Получаем 
    $f = q_2(z)(z-z_0)^2 + r_1(z-z_0) + r_0$. Продолжим разложение и получим
    $f(z) = f(z_0) + \frac{f'(z_0)}{1!}(z - z_0) + \frac{f''(z_0)}{2!}(z - z_0)^2 + \dots$ (здесь $r_0 = f(z_0), r_1 = f'(z_0), \ldots, r_k = \frac {f^{(k)}(z_0)}{k!}$). Такое разложение напоминает разложение в ряд Тейлора.
    Обозначим главную часть за $\alpha (z - z_0)^p$, $\alpha \neq 0$, $p \in \N$. Получим
    $f(z) = f(z_0) + \alpha (z-z_0)^p + o((z-z_0)^p)$.
    $f(z) = f(z_0) + (z - z_0)^p \left( \alpha + \frac{o((z-z_0)^p)}{(z-z_0)^p} \right)$, где $o((z - z_0)^p) \vdots (z - z_0)^p$. Так как последнее частное стремится к 0 при $z$, стремящимся к $z_0$, то верно
    $$\exists \varepsilon_1 < \varepsilon \ \forall z \in U_{\varepsilon_1}(z_0) \hookrightarrow \left| \frac {o((z-z_0)^p)}{(z-z_0)^p} \right| < \frac {|\alpha |}{2}.$$
    $$arg \alpha - \frac {\pi}{6} \leq arg(\alpha + \frac {o((z - z_0))^p}{(z - z_0)^p}) \leq arg \alpha + \frac {\pi}{6}, \ \alpha \in \left[\alpha_0, \alpha_0 + \frac {\pi}{3}\right].$$
    \begin{center}
        \includegraphics[width=0.74\textwidth]{images/lec2_1.png}
    \end{center}
    Тогда если $z = z_0 + re^{i\phi}$, где $r = |z|$, $e^{i\phi} = \cos(\phi) + i\sin(\phi)$, то справедливо $$arg \left( e^{ip\phi} \left( \alpha + \frac {o((z-z_0))^p}{(z-z_0)^p}\right) \right) = p\phi + \alpha_0 + \beta \frac {\pi}{3}, \ 0 \leq \beta \leq 1.$$
    Так как $\phi$ -- любое вещественное число, то теорема доказана.
\end{proof}

\begin{theorem}[Основная теорема алгебры]
    \label{ota}
    Всякий многочлен положительной степени из кольца $\Cm[x]$ имеет хотя бы один корень, в общем случае комплексный.
\end{theorem}

\begin{proof}
    Пусть $f(x)$ -- многочлены положительной степени. Пусть $A = \inf|f(z)|$, $z \in \Cm$. Покажем, что инфимум достигается, то есть существует такое комплексное $z_n \in \Cm$, что $|f(z_n)| = A$. По определению инфимума существует последовательность ${z_n}$ такая, что её модуль стремится к конечному $A$. Из этой последовательности можно извлечь
    подпоследовательность ${z_{n_k}}$ такую, что она сходится к $z_0$ или к бесконечности. Однако второй случай не реализуется, так как иначе $|f(z_{n_k})|$ также сходится к бесконечности. Тогда $\exists \lim_{k\to \infty} {z_{n_k}} = z_0$, откуда $\exists \lim_{k\to \infty} {f(z_{n_k})} = f(z_0)$, а значит, существует предел модуля такой функции, равный $\lim_{k\to \infty} {|f(z_{n_k})|} = |f(z_0)| = A$. Если оказалось так, что $A \neq 0$, то по лемме Д'Аламбера найдется $z \in U_{\epsilon}(z)$ такой что $|f(x)| < |f(z_0)| = A = \inf(|f(z)|)$, что противоречит определению инфимума. Значит, $A = 0$ и $\exists z_{0}: |f(z_{0})| = 0 \Rightarrow f(z_{0}) = 0$.
\end{proof}

\section{5. Инвариантные подпространства. Собственные векторы и собственные значения. Характеристический многочлен и его свойства. Инвариантность следа и определителя матрицы оператора.}

\begin{definition}
    Пусть $V$ -- линейное пространство, $\phi: V \to V$. Подпространство $U \leq V$ называется 
    \textit{инвариантным}, если выполняется $\forall x \in U \; \phi(x) \in U$. Другими словами, действие 
    оператора $\phi$ на вектор из $U$ не выводит его за пределы $U$, а значит 
    $\phi(U) \subset U \Leftrightarrow \phi(U) \leq U$.
\end{definition}

\begin{proposition}
    Пусть $\phi: V \to V$ -- линейный оператор, $U$ -- инвариантное подпространство. Тогда в базисе, 
    согласованном с $U$, оператор $\phi$ имеет матрицу с левым нижним углом нулей:
    \[\phi(A) = \left(\begin{array}{@{}c|c@{}}
        A & B\\
        \hline
        0 & C
    \end{array}\right)\]
    Здесь $A \in M_k(F)$, $k = \dim U$.
\end{proposition}

\begin{proof}
    $U$ инвариантно относительно $\phi$, а значит $\phi(e_1), \phi(e_2), \dots \phi(e_k) \in U$.
    Тогда для базисного вектора из $U$ ненулевыми могут быть только первые $k$ элементов 
    соответствующего ему столбца.
\end{proof}

\begin{note}
    Блок нулей в левом нижнем углу означает, что $\phi(e_1), \phi(e_2), \dots \phi(e_k) \in U$, 
    а значит подпространство $U$ является инвариантным относительно $\phi$.
\end{note}

\begin{definition}
    Пусть $\phi: V \to V$. Ненулевой вектор
    $x \in V: \phi(x) = \lambda x$ называется \textit{собственным вектором} оператора $\phi$, 
    отвечающим собственному значению $\lambda$.
\end{definition}

\begin{definition}
    Число $\lambda \in F$ называется \textit{собственным значением} оператора $\phi$, если 
    $\exists x \in V: x \neq 0,\; \phi(x) = \lambda x$,
    то есть если некоторое $x$ отвечает $\lambda$.
\end{definition}

\begin{definition}
    \textit{Характеристическим многочленом} оператора $\phi: V \to V$ называется определитель $|A - \lambda E| = \chi_A(\lambda)$, где $A$ -- матрица $\phi$ в произвольном базисе.
\end{definition}

\begin{theorem}
    Верны следующие свойства характеристического многочлена:
    \begin{enumerate}
        \item Корни $\chi(\lambda)$ принадлежащие полю $F$ и только они являются собственными 
        значениями $\phi$.
        \item Многочлен $\chi(\lambda)$ не зависит от выбора базиса.
    \end{enumerate} 
\end{theorem}

\begin{proof}
    \begin{enumerate}
        \item Пусть $\lambda_0$ -- корень $\chi_{\phi}(\lambda) \lra \chi_{\phi}(\lambda_0) = 0 \lra \det(A - \lambda_{0}E) = 0 \lra $ система $A - \lambda_{0}E$ имеет ненулевое решение при $x_0 \neq 0$ $\lra \phi x_{0} = \lambda_0 x_0 \lra \lambda_0$ -- собственное значение $\phi$. В силу равносильных переходов, обратное утверждение тоже верно. 
        \item Наряду с $e$ выберем базис $f$, обозначим за $S$ матрицу перехода между ними: 
        $S = S_{e \to f}$. Тогда $\phi \xleftrightarrow[e]{} A$, $\phi \xleftrightarrow[f]{} B$, 
        $B = S^{-1}AS$. Верна следующая цепочка равенств:
        \begin{eqnarray}
            \chi_b(\lambda) = |B - \lambda E| = |S^{-1}AS - \lambda E| 
            = |S^{-1}AS - S^{-1} \lambda E S| = \\ = |S^{-1}(A - \lambda E)S| 
            = |S^{-1}| \cdot |A - \lambda E| \cdot |S| = |A - \lambda E| = \chi_a(\lambda).
        \end{eqnarray}
        Таким образом, характеристический многочлен одинаков для всех базисов.
    \end{enumerate}
\end{proof}

\begin{corollary}
    От выбора базиса не зависят так же коэффициенты характеристического многочлена, в частности $det A$ и $tr A$, поэтому часто пишут $det \phi$ и $tr \phi$ соответственно.
\end{corollary}

\section{6. Линейная независимость собственных векторов, имеющих попарно различные собственные значения. Алгебраическая и геометрическая кратности собственного значения. Условия диагонализируемости линейного оператора.}

\begin{definition}
    Пусть $\lambda$ -- собственное значение оператора $\phi: V \to V$. Собственным подпространством 
    оператора $\phi$, отвечающим $\lambda$ называется подпространство 
    $V_{\lambda} = \ker (\phi - \lambda \epsilon) \leq V$.
\end{definition}

\begin{definition}
    Подпространства $U_1, U_2, \ldots, U_n$ называются линейно независимыми, если из равенства 
    $x_1 + x_2 + \dots + x_n = \bar{0}$, где $x_i \in U_i$, следует, что $x_1 = x_2 = \dots = x_n = 0$.
\end{definition}

\begin{theorem}[О линейной независимости собственных подпространств, отвечающих попарно различным собственным значениям]
    \label{o_lnz}
    Пусть $\phi: V \to V$, $\lambda_1, \lambda_2, \dots, \lambda_n$ -- различные собственные значения. 
    Тогда $V_{\lambda_1}, \ldots, V_{\lambda_n}$ линейно независимы.
\end{theorem}

\begin{proof}
    От противного. Пусть $\exists x_1 \in V_{\lambda_1}, x_2 \in V_{\lambda_2}, \dots, 
    x_n \in V_{\lambda_n}$, не все равные нулю. 
    Назовем эти наборы опровергающими, а его мощностью количество ненулевых векторов. 
    Из всех таких наборов векторов выберем набор наименьшей мощности. Пусть 
    указанный нами набор без ограничения общности -- искомый, перенумеруем множества и $x_i$ так,
    чтобы ненулевыми были первые $j$ векторов. Тогда $x_1 + x_2 + \dots + x_j = 0$, все
    $x_i \neq 0$, иначе есть набор меньшей мощности. Применим к сумме оператор $\phi$ и получим $\lambda_1 x_1 + \lambda_2 x_2 + \ldots + \lambda_j x_j$. Умножим изначальную сумму на $-\lambda_1$
    и сложим с получившейся. Тогда мы получили набор меньшей мощности, противоречие.
\end{proof}

\begin{definition}
    Линейный оператор $\phi: V \to V$ над полем $F$ называется диагонализируемым, если в $V$ 
    существует базис $e$ такой, что $A_{\phi}$ -- диагональная матрица.
\end{definition}

\begin{theorem}[критерий диагонализируемости линейного оператора]
    \label{theorem4.1}
    Пусть $\phi: V \to V$, $V$ над $F$. Пусть $\lambda_1, \lambda_2, \dots, \lambda_k$ -- все 
    попарно различные собственные значения, тогда следующие условия эквивалентны:
    \begin{enumerate}
        \item $\phi$ -- диагонализируем.
        \item В $V$ существует базис, состоящий из собственных векторов оператора $\phi$.
        \item $V = V_{\lambda_1} \oplus V_{\lambda_2} \oplus \dots \oplus V_{\lambda_k}$.
    \end{enumerate}
\end{theorem}

\begin{proof}~
    \begin{enumerate}
        \item $1 \Rightarrow 2$ \\
        Так как $\phi$ диагонализируем, то существует базис, в котором матрица оператора выглядит 
        следующим образом:
        \begin{equation*}
        \left(
            \begin{array}{cccc}
            \lambda_{1} & 0 & \ldots & 0\\
            0 & \lambda_{2} & \ldots & 0\\
            \vdots & \vdots & \ddots & \vdots\\
            0 & 0 & \ldots & \lambda_n
            \end{array}
        \right)
        \end{equation*}
        Значит, $\phi(e_i) = \lambda_i e_i$ для любого $i$, откуда $e_1, \dots, e_n$ -- собственные 
        векторы для $\phi$. Значит, $e$ -- базис из собственных векторов.
        \item $2 \Rightarrow 3$ \\
        Пусть $e$ -- базис из собственных векторов оператора $\phi$. Перегруппируем базисные векторы 
        по собственным значениям: 
        $$\underbrace{(e_{11}, \dots, e_{1s_1})}_{\lambda_1} \underbrace{(e_{21}, \dots, e_{2s_2})}_{\lambda_2} 
        \dots \underbrace{(e_{k1}, \dots, e_{ks_k})}_{\lambda_k}$$
        $\langle e_{11}, \dots, e_{1s_1} \rangle \subseteq V_{\lambda_1}$, 
        $\dots, \langle e_{k1}, \dots, e_{ks_k} \rangle \subseteq V_{\lambda_k}$. 
        Откуда $V = V_{\lambda_1} + \dots + V_{\lambda_k}$, но так как собственные подпространства 
        линейно независимы (по \ref{o_lnz}), то по теореме о характеризации прямой суммы 
        $V = V_{\lambda_1} \oplus V_{\lambda_2} \oplus \dots \oplus V_{\lambda_k}$.
        \item $3 \Rightarrow 1$ \\
        Известно, что $V = V_{\lambda_1} \oplus V_{\lambda_2} \oplus \dots \oplus V_{\lambda_k}$. 
        Выберем в каждом $V_{\lambda_i}$ базис: $e_{i_1}, \dots, e_{is_i}$. Тогда, объединяя базисы 
        собственных подпространств, получим базис всего пространства $V$. 
        При этом по диагонали будут стоять сначала $s_1$ значений $\lambda_1$, 
        затем $s_2$ значений $\lambda_2$ и так далее. Остальные значения -- нули. 
        Значит, $\phi$ -- диагонализируем.
        \begin{equation*}
        \left(
            \begin{array}{ccccc}
            \lambda_{1} & 0 & \ldots & 0 & 0\\
            \vdots & \vdots & \ddots & \vdots & \vdots\\
            0 & \ldots & \lambda_{1} & \ldots & 0\\
            \vdots & \vdots & \ddots & \vdots & \vdots\\
            0 & 0 & \ldots & \ldots & \lambda_n
            \end{array}
        \right)
        \end{equation*}
    \end{enumerate}
\end{proof}

\begin{definition}
    Пусть $\phi: V \to V$, $\lambda \in F$ -- его собственное значение, $\chi_{\phi}(\lambda) = 0$.
    Кратность корня $\lambda$ как корня характеристического многочлена называется алгебраической 
    кратностью собственного значения $\lambda$. Обозначение: $alg(\lambda) \geq 1$.
\end{definition}

\begin{definition}
    Размерность собственного подпространства $V_{\lambda}$ называется геометрической кратностью 
    собственного значения $\lambda$. Обозначение: $geom(\lambda) = \dim V_{\lambda} \geq 1$.
\end{definition}

\begin{proposition}
    \label{pr4.1}
    Пусть $\phi: V \to V$ и $U$ -- инвариантное подпространство относительно $\phi$. 
    Пусть $\psi = \phi \mid_{U}$. Тогда $\chi_{\phi} \vdots \chi_{\psi}$.
\end{proposition}

\begin{proof}
    Пусть $e$ -- базис в $V$, согласованный с инвариантным подпространством $U$: 
    $e = (\underbrace{e_{1}, \dots, e_{k}}_{U}, e_{k + 1}, \dots, e_n)$
    \[A_{\phi} = \left(\begin{array}{@{}c|c@{}}
        B & C\\
        \hline
        0 & D
    \end{array}\right)\]
    \[\chi_{\phi}(\lambda) = \det \left(\begin{array}{@{}c|c@{}}
        B - \lambda E & C\\
        \hline
        0 & D - \lambda E
    \end{array}\right) = |B - \lambda E| |D - \lambda E| = \chi_{\psi} \chi_D.\]
\end{proof}

\begin{corollary}
    Для любого собственного значения $\lambda$: $geom(\lambda) \leq alg(\lambda)$.
\end{corollary}

\begin{proof}
    $U = V_{\lambda}$ -- инвариантно относительно оператора $\phi$.
    \begin{equation*}
    \psi =
        \left(
            \begin{array}{ccc}
            \lambda & \dots & 0 \\
            \vdots & \ddots & \vdots \\
            0 & \dots & \lambda \\
            \end{array}
        \right)
    \end{equation*}
    $\chi_{\psi} = (\lambda - t)^k$, где $k = \dim V_{\lambda} = geom(\lambda)$. 
    По утверждению \ref{pr4.1} $\chi_{\phi} \vdots \chi_{\psi}$, откуда следует, что 
    $\chi_{\phi} \vdots (\lambda - t)^k$. Значит, $alg(\lambda) \geq geom(\lambda)$.
\end{proof}

\begin{theorem}[критерий диагонализируемости в терминах алгебраической и геометрической кратностей линейного оператора]
    Пусть $\phi: V \to V$, $\dim V = n$. $\phi$ -- диагонализируем тогда и только тогда, когда 
    \begin{enumerate}
        \item $\chi_{\phi}(t)$ разлагается на линейные множители над $F$. (Далее: $\phi$ линейно факторизуем над полем $F$)
        \item Для любого собственного значения $\lambda$ оператора $\phi$ выполнено $alg(\lambda) = geom(\lambda).$
    \end{enumerate}
\end{theorem}

\begin{proof}~
    \begin{enumerate}
        \item $(\Rightarrow)$ Пусть $\phi$ -- диагонализируемый, тогда существует базис, в котором матрица оператора $\phi$ 
        имеет диагональный вид и по теореме \ref{theorem4.1} 
        $V = V_{\lambda_1} \oplus V_{\lambda_2} \oplus \dots \oplus V_{\lambda_k}$. 
        Тогда по свойству прямой суммы: 
        $$\sum_{i=1}^k geom(\lambda_i) = \sum_{i=1}^k \dim V_{\lambda_i} = 
        \dim V = n = \deg \chi \geq \sum_{i=1}^k alg(\lambda_i)$$
        С одной стороны, выполнено неравенство выше, но, с другой стороны, по предыдущему следствию 
        $geom(\lambda) \leq alg(\lambda)$, откуда верно, что $\forall i: alg(\lambda_i) = geom(\lambda_i)$.
        \item $(\Leftarrow)$ Пусть $\chi$ -- линейно факторизуем над $F$ и $alg(\lambda_i) = geom(\lambda_i)$. 
        Докажем диагонализируемость оператора $\phi$.
        $$\dim(V_{\lambda_1} \oplus V_{\lambda_2} \oplus \dots \oplus V_{\lambda_k}) = 
        \sum_{i=1}^k \dim V_{\lambda_i} = \sum_{i=1}^k geom(\lambda_i) = \sum_{i=1}^k alg(\lambda_i) = n$$
        Последнее равенство следует из линейной факторизуемости $\chi$. 
        Отсюда получаем, что $V = V_{\lambda_1} \oplus V_{\lambda_2} \oplus \dots \oplus V_{\lambda_k}$ -- 
        эквивалентное определение диагонализируемости.
    \end{enumerate}
\end{proof}

\section{7. Приведение матрицы преобразования к треугольному виду. Теорема Гамильтона-Кэли (случай, когда характеристический многочлен линейного оператора раскладывается на линейные множители).}

\begin{proposition}
    \label{prop4.2}
    Следующие условия на подпространстве $U$ эквивалентны:
    \begin{enumerate}
        \item $U$ -- инвариантно относительно $\phi$.
        \item $\exists \lambda \in F: U$ -- инвариантно относительно $\phi - \lambda E$.
        \item  $\forall \lambda \in F: U$ -- инвариантно относительно $\phi - \lambda E$.
    \end{enumerate}
\end{proposition}

\begin{proof}~
    \begin{enumerate}
        \item $(1 \Rightarrow 3)$ Возьмем вектор $x \in U$. Тогда верно
        $$(\phi - \lambda E)(x) = \phi(x) - (\lambda E)(x) \in U.$$
        \item $(3 \Rightarrow 2)$ Очевидно.
        \item $(2 \Rightarrow 1)$ По условию $\exists \lambda \in F:$
        $$\forall x \in U \; \phi(x) = (\phi - \lambda E)(x) + (\lambda E)(x) \in U.$$
    \end{enumerate}
\end{proof}

\begin{proposition}
    \label{utv4.3}
    Пусть $\phi: V \to V$, $\phi$ линейно факторизуем над $F$ и  $n = \dim V$. Тогда в $V$ найдется 
    $(n - 1)$--мерное подпространство, инвариантное относительно $\phi$.
\end{proposition}

\begin{proof}
    Пусть $\chi_{\phi}(t) = \prod_{i=1}^n (\lambda_i - t)$, $\lambda_n$ -- собственное значение для $\phi$, 
    $V_{\lambda_n} = \ker (\phi - \lambda_n E) \neq \{ \overline{0} \}$. Из того, что ядро не пусто, 
    следует, что образ $\im(\phi - \lambda_n E) \neq V$, и, значит, $\dim (\im(\phi - \lambda_n E)) \leq n - 1$. 
    Тогда существует подпространство $U$ такое, что $\dim U = n - 1$ и образ $\phi - \lambda_n E$ лежит в $U$. 
    
    Докажем, что такое подпространство инвариантно. Пусть $x \in U$, то $(\phi - \lambda_n E)(x) \in \im(\phi - \lambda_n E) \subseteq U$. 
    Значит, $U$ инвариантно относительно $\phi - \lambda_n E$, то есть $U$ инвариантно и относительно $\phi$.
\end{proof}

\begin{definition}
    \textit{Флагом} подпространства над $V$ называется цепочка инвариантных подпространств 
    $$\{ \overline{0} \} = V_0 < V_1 < \ldots < V_n = V, \ \dim V_k = k.$$
\end{definition}

\begin{theorem}[о приведении линейного оператора к верхнетреугольному виду]
    Пусть $\phi: V \to V$, $\phi$ линейно факторизуем над $F$ и  $n = \dim V$. Тогда в $V$ существует базис $e$, в котором матрица $\phi$ -- верхнетреугольная.
    \begin{equation*}
        \left(
            \begin{array}{ccc}
            \lambda_1 & \dots & * \\
            \vdots & \ddots & \vdots \\
            0 & \dots & \lambda_n \\
            \end{array}
        \right)
    \end{equation*}
\end{theorem}

\begin{proof}
    Докажем индукцией по $n$. \\
    \begin{enumerate}
        \item База $n = 1$: $\{ \overline{0} \} < U_1 = V_1$ -- флаг существует.
        \item
        Шаг индукции: пусть для $V$ с $\dim V < n$ утверждение справедливо. Докажем для $V: \dim V = n$. 
        По утверждению \ref{utv4.3} в $V$ найдется $U_{n - 1} < V$; $\dim U_{n - 1} = n - 1$. 
        Рассмотрим функцию $\psi = \phi \mid_{U_{n - 1}}$, тогда по \ref{pr4.1} 
        $\chi_{\phi} \vdots \chi_{\psi}$. Где $\chi_{\phi}$ раскладывается на $n$ линейных множителей. 
        Очевидно, что тогда характеристический многочлен $\chi_{\psi}$ состоит из тех линейных множителей, 
        которые входили в $\chi_{\phi}$. Следовательно, $\chi_{\psi}$ раскладывается на линейные множители. 
        Тогда к определителю $\psi: U_{n - 1} \to U_{n - 1}$ применимо предположение индукции:
        $$\{ \overline{0} \} < U_1 < \dots < U_{n - 1} < U_n = V  (*)$$
        Тут первые $n - 1$ подпространств инвариантны относительно $\psi$, значит, инвариантны и относительно $\phi$. \\
        Выберем базис $e$ в $V$, согласованный с разложением $(*)$, где $(e_1, \dots, e_k)$ -- базис в $U_k$, 
        тогда в матрице базиса $e$ в первой строке будет столбец, согласованный с $U_1$, то есть 
        $\lambda_1$ и нули снизу, далее столбец, согласованный с $U_2$ и так далее.
        \begin{equation*}
            \phi_e =
            \left(
                \begin{array}{cccc}
                    \lambda_1 & * & \dots & * \\
                    0 & \lambda_2 & \dots & * \\
                    \vdots & \vdots & \ddots & \vdots \\
                    0 & 0 & \dots & \lambda_n \\
                \end{array}
                \right)
            \end{equation*}
    \end{enumerate}
\end{proof}

\begin{corollary}
    \label{col2}
    В условиях предыдущей теоремы $\forall k = 1, \dots, n \hookrightarrow (\phi - \lambda_k E)(\phi - \lambda_{k + 1} E) \dots (\phi - \lambda_n E) V \subseteq U_{k - 1}$. (первые несколько скобок -- множители $\chi$)
\end{corollary}

\begin{proof}
    $\chi(V) = (\phi - \lambda_k E) \dots (\phi - \lambda_{n - 1} E) U_{n - 1} \subseteq \dots \subseteq U_{k - 1}$.
\end{proof}

\begin{theorem}[Гамильтона--Кэли]
    \label{th4.4}
    Пусть $\phi: V \to V$, $\phi$ -- линейно факторизуем над $F$, $\chi_{\phi}(t) \in F[t]$ -- характеристический многочлен. Тогда $\chi_{\phi}(\phi) = 0$ (нулевой оператор).\\
    (Иначе: $A \in M_n(F)$, $\chi_A(t)$ -- характеристический многочлен матрицы $A$, то $\chi_A(A) = 0$).
\end{theorem}

\begin{proof}~

    Выберем базис $e$ в $V$, согласованный с флагом инвариантных подпространств. Имеем $\chi_{\phi}(t) = \prod_{i=1}^n (\lambda_i - t)$. Применим $\chi_{\phi}(\phi)$ к пространству $V$, получим $\chi_{\phi}(\phi)V = \left((-1)^n \prod_{i=1}^n (\phi - \lambda_i E)\right)V$. При $k = 0$ из следствия \ref{col2} получаем, что $\chi_{\phi}(\phi)V \leq \{\overline{0}\}$. Значит, $\chi_{\phi}(\phi) = 0$.
\end{proof}

\begin{note}
    Теорема Гамильтона-Кэли справедлива для любого линейного оператора над любым полем.
\end{note}

\section{8. Корневое подпространство линейного оператора. Свойства корневых подпространств. Разложение пространства в прямую сумму корневых подпространств (случай, когда характеристический многочлен линейного оператора раскладывается на линейные множители).}

\begin{definition}
    $\phi : V \to V$, $V$ -- линейное пространство над полем $F$, $\lambda \in f$.
    Вектор $x \in V$ называют \textit{корневым} для $\phi$, отвечающим $\lambda \in F$, если 
    $\exists k \in \N \; : (\phi - \lambda \epsilon)^{k}x = 0$.
\end{definition}

\begin{definition}
    Число $k$ -- \textit{высота корневого вектора} $x$, отвечающего $\lambda$ из $F$, если $k$ -- наименьшее 
    число такое, что $(\phi - \lambda \epsilon)^k x = 0$. 
    Будем считать, что нулевой вектор имеет высоту 0.
\end{definition}

\begin{definition}
    Подпространство $ V^{\lambda}$ -- множество всех корневых векторов для $\phi$, относящихся к $\lambda$. $ V^{\lambda}$ называется \textit{корневым подпространством} для оператора $\phi$ относящегося к $\lambda$.
\end{definition}

\begin{definition}
    Подпространство $W$ называется \textit{дополнительным} к $V^{\lambda}$, 
    если их пересечение состоит только из нуля.
\end{definition}

\begin{theorem}[о свойствах корневых подпространств]~
    \label{th5.1}

    Пусть $V^{\lambda}$ -- корневое подпространство для $\phi$ отвечающее $\lambda$. Тогда
    \begin{enumerate}
        \item $V^{\lambda}$ инвариантно относительно $\phi$.
        \item Подпространство $V^{\lambda}$ имеет единственное собственное значение $\lambda$.
        \item Если $W$ -- тоже инвариантное относительно $\phi$ подпространство, при этом являющееся 
        дополнительным к $V^{\lambda}$, то на $W$ оператор 
        $\phi - \lambda \epsilon$ действует невырожденным образом.
    \end{enumerate} 
\end{theorem}

\begin{proof}~
    \begin{enumerate}
        \item Пусть m -- максимальная высота векторов $x \in V^{\lambda}$, в силу конечномерности 
              $V^{\lambda}$ такая существует и является конечным числом.
              Тогда $V^{\lambda} = \ker (\phi - \lambda \epsilon)^m$. 

              Операторы $\phi$ и $\epsilon$
              коммутируют с $\phi$, а значит и оператор $(\phi - \lambda \epsilon)^m$ коммутирует с 
              $\phi$. Таким образом, можно записать 
              $(\phi - \lambda \epsilon)^m \phi = \phi (\phi - \lambda \epsilon)^m$. 
              Тогда по теореме о коммутирующих линейных операторах получаем, что $\ker (\phi - \lambda \epsilon)^m$ 
              инвариантно относительно $\phi$. 
        \item Докажем от противного, пусть в $V^{\lambda}$ найдется ненулевой собственный вектор $x$ 
              с собственным значением $\mu \neq \lambda$, то есть $\phi(x) = \mu x$. Применим к этому 
              вектору оператор $(\phi - \lambda \epsilon)$:  
              $$(\phi - \lambda \epsilon) x = \phi(x) - (\lambda \epsilon)(x) = (\mu - \lambda) x.$$ 
              Тогда при многократном применении 
              получим $(\phi - \lambda \epsilon)^m x = (\mu - \lambda)^m x = 0$,
              так как $x \in V^{\lambda}$ и должен аннулироваться. Тогда $\mu - \lambda = 0$, 
              что дает противоречие.
        \item По условию $V$ представляется как $V = V^{\lambda} \oplus W$. При этом подпространства 
              $V^{\lambda}$ и $W$ инвариантны относительно $\phi$, а значит, согласно утверждению 
              \ref{prop4.2}, они так же инвариантны относительно $(\phi - \lambda \epsilon)$. 
              Нам нужно доказать, что $\phi - \lambda \epsilon$ невырожден на $W$, то есть что 
              $\ker (\phi - \lambda \epsilon) \vert_{W} = \{0\}$.

              Докажем от противного, пусть $\exists x \neq 0$ такое что 
              $x \in \ker (\phi - \lambda \epsilon) \vert_{W}$. 
              Отсюда следует, что вектор $x$ лежит в пространстве $W$, так как лежит в ядре сужения
              оператора на это подпространство.

              Однако $(\phi - \lambda \epsilon) x = 0$, а значит х -- собственный для $\phi$ 
              с собственным значением $\lambda$. Тогда вектор $x$ так же лежит и в пространстве 
              $V^{\lambda}$, что приводит к 
              противоречию с тем, что по условию $V^{\lambda} \cap W = \{0\}$.
    \end{enumerate}
\end{proof}

\begin{corollary}
    Корневое подпространство $V^{\lambda}$ -- максимальное по включению инвариантное подпространство, 
    на котором $\phi$ имеет единственное собственное значение $\lambda$.
\end{corollary}

\begin{theorem}[о разложении пространства V в прямую сумму корневых]~
    \label{th5.2}

    Пусть $\phi \in \mathcal{L}$, $\phi$ -- линейно факторизуем над $F$.
    Тогда пространство $V$ есть прямая сумма корневых подпространств: 
    $V = V^{\lambda_1} \oplus V^{\lambda_2} \oplus \ldots \oplus V^{\lambda_k}$, где все $\lambda_i$ попарно различны.
\end{theorem}

\begin{proof}
    По условию $\phi$ линейно факторизуем, а значит
    $\chi_{\phi}(t) = \displaystyle\prod_{i= 1}^{k} (\lambda_i - t)^{m_i}$. Многочлены $(\lambda_i - t)^{m_i}$ попарно взаимно просты 
    из попарной различности $\lambda_i$, поэтому по следствию из теоремы о взаимно простых делителях аннулирующего многочлена \footnote{Пусть $\phi \in \mathcal{L}(V)$, $f$ -- аннулирующий многочлен для $\phi$, такой что $f$ 
    раскладывается в произведение $f = f_1 \cdot f_2 \dots f_n$ попарно взаимно-простых многочленов.
    Тогда $V$ раскладывется в прямую сумму $V = V_1 \oplus V_2 \oplus \dots V_n$, 
    где $V_i = \ker f_i(\phi)$ -- инвариантные подпространства.} можно заключить:
    $$V = \ker (\phi - \lambda_1 \epsilon)^{m_1} \oplus \ker (\phi - \lambda_2 \epsilon)^{m_2} 
    \oplus \dots \oplus \ker (\phi - \lambda_k \epsilon)^{m_k}.$$
    При этом $\ker (\phi - \lambda_i \epsilon)^{m_i} \subseteq V^{\lambda_i}$ для всех $i$,
    а значит вектор $x \in V$ представим в виде суммы 
    $x = x_1 + \ldots + x_k$, где $x_i \in V^{\lambda_i}$.
    Отсюда очевидно, что пространство $V$ является суммой подпространств: 
    $$V = V^{\lambda_1} + V^{\lambda_2} + \dots V^{\lambda_k}.$$ 
    
    Осталось доказать что $V^{\lambda_i} \subseteq \ker(\phi - \lambda_i \epsilon)^{m_i}$, 
    в таком случае сумма будет прямой. Докажем от противного, пусть существует индекс $i$ такой, 
    что $\ker (\phi - \lambda_i \epsilon) \leq V^{\lambda_i}$. Тогда найдется вектор 
    $x \in V^{\lambda_i}$ такой, что он не лежит в ядре. Обозначим высоту $x$ за $M > m_i$, тогда:
    $$\chi_{\phi}(\phi) x = \left(\displaystyle\prod_{j \neq i} (\phi - \lambda_j \epsilon)^{m_j}\right) \cdot 
    (\phi - \lambda_i \epsilon)^{m_i} x = \displaystyle\prod_{j \neq i} 
    ((\phi - \lambda_j \epsilon)^{m_j})x' \neq 0.$$ 
    Если найдется такой $j$ что $(\phi - \lambda_j \epsilon)x = 0$, то у $x'$ есть собственное значение
    $\lambda_j$, что приводит к противоречию с пунктом 2 теоремы \ref{th5.1}. 
    В противном случае возникает противоречие с 
    $\chi_{\phi}(\phi) = 0$ по теореме \ref{th4.4} (Гамильтона-Кэли). 
    Таким образом, $V^{\lambda_i} = \ker(\phi - \lambda_i \epsilon)^{m_i}$, а значит $V$ представляется 
    в виде прямой суммы $V^{\lambda_i}$:

    $$V = V^{\lambda_1} \oplus V^{\lambda_2} \oplus \, \dots \, \oplus V^{\lambda_k}.$$
\end{proof}

\section{9. Циклические подпространства. Теорема о нильпотентном операторе. Жорданова нормальная форма и жорданов базис линейного оператора. (Теорема существования жорданова базиса).}

\begin{definition}
    Оператор $\phi: V \to V$ называется \textit{нильпотентным}, если $\exists k \in \N :\: \phi^k = 0$. Наименьшее наутральное число $k$ такое, что $\phi^k = 0$, $\phi^{k-1} \neq 0$, называют \textit{индексом нильпотентности} относительно $\phi$. 
\end{definition}

\begin{definition}
    Пусть $\phi$ -- нильпотентный и $x \in V$ -- вектор, имеющий высоту $k$. 
    Рассмотрим $U = \langle x, \phi(x), \ldots, \phi^{k-1}(x)\rangle$.
    Построенное инвариантное подпространство $U$ называется \textit{циклическим подпространством}, 
    порожденным вектором $x$.
\end{definition}

\begin{theorem}[о нильпотентном операторе]~
    \label{th5.3}

    Пусть $\phi: V \to V$ -- нильпотентный оператор индекса нильпотентности $k$, $x \in V$ -- ненулевой вектор высоты $k$,
    $U = \langle x, \phi(x), \ldots, \phi^{k-1}(x) \rangle$ -- циклическое подпространство, инвариантное $\phi$.
    Тогда найдется $\phi$--инвариантное пространство $W$ дополнительное к $U$ такое, что $V = U \oplus W$.
\end{theorem}

\begin{proof}~

    \begin{enumerate}
        \item Пусть $W$ -- максимальное $\phi$-инвариантное подпространство в $V$, 
        такое что $U \cap W = {0}$. Предположим что $U + W < V$. 
        Тогда найдется ненулевой $a \in V$, такой что $ a \notin U + W$.  Пусть $l$ -- наименьшее 
        значение для которого $z = \phi^{l-1}(a) \notin W + U$, $\phi^l(a) \in W + U$. 
        Такое очевидно найдется так как $a \notin W + U$ и $\phi^{k}(a) = 0 \in W + U$.
        Таким образом в этом пункте мы нашли вектор $z \notin U+W$, такой что $\phi(z) \in U + W$.

        \item Пусть $\phi(z) = \displaystyle\sum_{s = 0}^{k-1} \alpha_s \phi^s(x) + w$, 
        при этом $\phi^s(x) \in U$, $w \in W$. Тогда:
        $$\phi^{k}(z) = \alpha_0 \phi^{k-1}(x) + 0 + \dots + 0 + \phi^{k-1}(w) = 0$$ 
        Тогда $\alpha_0 \phi^{k-1}(x) + \phi^{k-1}(w) = 0$. 
        В силу линейной независимости линейных подпространств 
        $\alpha_o \phi^{k-1}(x) = 0$, $\phi^{k-1}(w) = 0$. 
        При этом в силу того, что $\phi^{k-1}(x) \neq 0$, получаем $\alpha_0 = 0$.

        \item Введем вектор $y = z - \displaystyle\sum_{s = 1}^{k-1} \alpha_s \phi^{s-1}(x) \notin U + W$ 
        (так как $z \notin U+W$, а сумма принадлежит $U$).
        Введем пространство $W' = W + \langle y \rangle$, $\dim W' = \dim W + 1$. 
        Покажем что вновь построенное подпространство так же инвариантно $\phi$:
        $$\phi(y) = \phi(z) - \displaystyle\sum_{s=1}^{k-1} \alpha_s \phi^s (x) = \phi(z) - 
        \displaystyle\sum_{s=0}^{k-1} \alpha_s \phi^s (x) = w \in W.$$

        \item Покажем теперь что $W'$ удовлетворяет условию $U \cap W' = {0}$. 
        Пусть $0 \neq u \in U \cap W'$, $u \notin U \cap W = \{ 0 \}$. 
        Тогда $u$ предствим в виде $u = \widetilde{w} + \lambda y$, $\lambda \neq 0$. Отсюда 
        $y = \frac{1}{\lambda} u - \frac{1}{\lambda} \widetilde{w} \in U + W$. 
        Значит $U \cap W \neq \{0\}$ -- противоречие.
    \end{enumerate}
\end{proof}

\begin{theorem}[о разложении в прямую сумму циклических подпространств для нильпотентного оператора]
    Пусть $\phi: V \to V$, зафиксируем индекс нильпотентности $k$. Тогда существует разложение $V$ 
    в прямую сумму инвариантных циклических подпространств $V = V_1 \oplus V_2 \oplus \ldots \oplus V_s$. 
    При этом количество слагаемых $s = \dim (\ker \phi) = \dim V_0 = geom(0)$.
\end{theorem}

\begin{proof}
    Индукция по $n = \dim V$. 
    \begin{enumerate}
        \item База: $n = 1 \Rightarrow \phi = 0$, $alg(0) = geom(0) = 1$.
        \item Предположение индукции: для пространства $V$ размерности менее $n$ утверждение выполняется. 
        Пусть теперь $\dim V = n$. 
        Тогда существует $x$ высоты $k$ такой что $\phi^k(x) = 0$, $\phi^{k-1}(x) \neq 0$. 
        Пусть $U = \langle x, \phi(x), \dots, \phi^{k-1}(x) \rangle$ -- $\phi$-инвариантное подпространство.
        По теореме \ref{th5.3} существует $\phi$-инвариантное подпространство $W$, такое что  
        $V = U \oplus W$, $\dim W \leq n-1$. Тогда $W$ раскладывается в прямую сумму 
        $\phi$-инвариантных циклических подпространств.
    \end{enumerate}
\end{proof}

\begin{definition}
    Жордановой клеткой, относящейся к  $\lambda \in F$, называется следующая матрица:
    \[J_{k}(\lambda) = \begin{pmatrix}
    \lambda      & 1      & 0      & \dots  & 0\\
        0      & \lambda      & 1      & \dots  & 0\\
        \vdots & \vdots & \vdots & \ddots & \vdots \\
        0      & 0      & \dots      & \lambda  & 1\\
        0      & 0      & 0      & \dots  & \lambda \\
    \end{pmatrix}\]
\end{definition}

\begin{definition}
    Жордановой матрицей называется блочно-диагональная матрица, по главной диагонали которой идут 
    Жордановы клетки, а остальное заполено нулями:
    \[J_{k}(\lambda) = \begin{pmatrix}
        J_{k_1}(\lambda_1)      & \dots      & 0    & 0 \\
        \vdots      & J_{k_2}(\lambda_2)      & \dots   & 0 \\
        0   & \vdots     & \ddots    & \vdots \\
        0      & 0      & \dots    & J_{k_n}(\lambda_n) \\
        \end{pmatrix}\]
\end{definition}

\begin{theorem}[Камиль Жордан]
    Пусть $\phi : V \to V$, $\phi$ -- линейно факторизуем над $F$. 
    Тогда в $V$ существует базис (Жордановый базис), в котором $\phi$ имеет Жорданову матрицу.
\end{theorem}

\begin{note}
    Жорданова матрица определена с точностью до перестановки Жордановых клеток, поэтому базис не единственен в общем случае.
\end{note}

\begin{proof}
    Заметим, что:
    \begin{enumerate}
        \item $V = V^{\lambda_1} \oplus V^{\lambda_2} \oplus \ldots V^{\lambda_k}$ (подпространства инвариантны), где $\lambda_1, \ldots, \lambda_k$ -- все попарно различные собственные значения оператора $\phi$. Тогда в базисе согласованном с таким разложением матрица имеет вид:
         \[A = \begin{pmatrix}
            A_1      & \dots      & 0    & 0 \\
            \vdots      & A_2      & \dots   & 0 \\
            0   & \vdots     & \ddots    & \vdots \\
            0      & 0      & \dots    & A_k \\
            \end{pmatrix}.\]
        \item Для $V^{\lambda_i}$ оператор $\phi_{\lambda_i} = \phi - \lambda_i E$ нильпотентен, а значит V 
        раскладывается в сумму циклических подпространств: $V^{\lambda_i} = \displaystyle\sum_{j = 1}^{geom(\lambda_i)} V_{ij}$.
    \end{enumerate}
    Пусть $\dim V_{ij} = k$. Покажем, что на $V_{ij}$ оператор $\phi$ в подходящем базисе имеет вид $J_k(\lambda_i)$:\\ 
    Пусть $k$ -- индекс нильпотентности $\phi_{\lambda_i}$ на $V_{ij}$, пусть $x$ -- корневой вектор максимальной высоты $k$.\\
    Рассмотрим базис $\langle \phi_{\lambda_i}^{k-1} x, \phi_{\lambda_i}^{k-2} x, \dots, \phi_{\lambda_i}^{1} x \rangle$. 
    Обозначим базисные вектора за $f_{ij}$ следующим образом:
    \begin{gather*}
        f_{i1} = \phi_{\lambda_i}^{k - 1},\\
        f_{i2} = \phi_{\lambda_i}^{k - 2},\\
        \dots
    \end{gather*}
    Подействуем на базис оператором $\phi_{\lambda_i}$. Под действием этого оператора каждый базисный 
    вектор перейдет в предыдущий (первый перейдет в $0$): $\phi_{\lambda_i}(f_{i1}) = \overline{0}, \dots, \phi_{\lambda_i}(f_{ik}) = f_{i(k - 1)}$. 
    Тогда матрица оператора $\phi_{\lambda_i}$ будет иметь в базисе $f$ вид $J_k(0)$.
    Тогда $\phi \vert_{V_{ij}} = \lambda_1 \epsilon + J_k(0) = J_k(\lambda_i)$
    Мы доказали, что в подходящем базисе сужение на подпростанство имеет вид Жордановой клетки. Тогда из
    $V = \displaystyle\sum_{i = 1}^{k} \displaystyle\sum_{j = 1}^{geom(\lambda_i)} V_{ij}$
    вытекает, что матрица оператора в подходящем базисе (Жордановом базисе) имеет вид Жордановой матрицы.
\end{proof}

\section{10. Жорданова диаграмма. Метод ее построения без поиска жорданова базиса. Теорема о единственности жордановой нормальной формы с точностью до перестановки клеток.}

\begin{definition}
    \textit{Жордановой диаграммой}, соответствующей Жордановой матрице $J$, называется набор точек на плоскости, изображающих вектора Жорданова базиса. При этом точка с координатой $(i, j)$ 
    изображает вектор $f_{ij}$ Жорданова базиса. Под каждым столбцом Жордановой диаграммы указывается соответствующее векторам этого 
    столбца собственные значения.
\end{definition}

\begin{example}
    Пусть $\phi$ имеет в некотором базисе следующую матрицу:
        \[A = \begin{pmatrix}
        \lambda & 1        & 0       & 0       & 0        & 0    &0    & 0\\
        0       & \lambda  & 1       & 0       & 0        & 0    &0    & 0\\
        0       & 0        & \lambda & 0       & 0        & 0    &0    & 0\\
        0       & 0        & 0       & \lambda & 1        & 0    &0    & 0\\
        0       & 0        & 0       & 0       & \lambda  & 0    &0    & 0\\
        0       & 0        & 0       & 0       & 0        & \mu  &1    & 0\\
        0       & 0        & 0       & 0       & 0        & 0    &\mu  & 0\\
        0       & 0        & 0       & 0       & 0        & 0    &0    & \nu
        \end{pmatrix}\]
        
    Четыре Жордановы клетки: порядков $2$ и $3$ с собственным значением $\lambda$, порядка $2$ с собственным значением $\mu$ и порядка $1$ с собственным значением $\nu$. 
    
    Такая матрица является Жордановой. Начнем выписывать Жорданов базис: $f_{11}$, $f_{12}$, $f_{13}$, $f_{21}$, $f_{22}$, $f_{31}$, $f_{32}$, $f_{41}$.
    
    В общем случае, если мы пишем Жорданов базис в виде $f_{ij}$, коэффициенты означают номер клетки и 
    номер вектора относительно данной клетки соответственно. Теперь вектор $f_{ij}$ можно сопоставить 
    точке на графике с координатами $(i, j)$. Если под каждым столбцом указать соответсвующие векторам столбца собственные значения, то полученный график называется Жордановой диаграммой.
    \begin{center}
        \includegraphics[width = 0.4\textwidth]{images/lec6_1.PNG}
    \end{center}
\end{example}

\begin{note}
    Столбцы не обязательно должны быть отсортированы в порядке невозрастания, диаграмма соотвествует 
    конкретной матрице и меняется при перестановке клеток местами.
\end{note}

\begin{proposition}[Свойства Жордановой диаграммы]~
    \begin{enumerate}
        \item Соответствие Жордановой матрицы $J$ и Жордановой диаграммы $J$ взаимно однозначно.
        \item Векторы Жордановой диаграммы, относящиеся к собственному значению $\lambda_i$, образуют базис 
        в корневом подпространстве $V^{\lambda_i}$.
        \item Если вектор $f_{ij}$ относится к собственному значению $\lambda_i$, 
        то он является корневым вектором, относящимся к $\lambda_i$ высоты $j$, 
        то есть $\phi_{\lambda_i}^j f_{ij} = \overline{0}$, но $\phi_{\lambda_i}^{j-1} f_{ij} \neq \overline{0}$.
        На высоте 1 в Жордановой диаграммы находятся собственные векторы оператора $\phi$.
        \item Если $f_{ij}$ относится к собственному значению $\lambda_{j}$, то $\phi_{\lambda_i} f_{ij} = f_{i(j-1)}$.
        %Нарисовать диаграмму со стрелочками вниз и нули на нижней линии.
        \item Каждый столбец в Жордановой диаграмме является изображением циклического подпространства для оператора $\phi_{\lambda_i}$. Общее число столбцов в Жордановой диаграмме $\displaystyle\sum_{i=1}^{k} geom(\lambda_i)$.
    \end{enumerate}
\end{proposition}

\begin{proposition}
    Пусть $\phi : V \to V$. Тогда справедливы следующие вложения:
    \begin{enumerate}
        \item $\ker \phi^0 \subseteq \ker \phi \subseteq \ker \phi^2 \subseteq \dots$
        \item $\im \phi^0 \supseteq \im \phi \supseteq \im \phi^2 \supseteq \dots$
    \end{enumerate}
    Причем обе цепочки стабилизируются за конечное число шагов.
\end{proposition}

\begin{proof}
    Индукция по $n \in \Z_{\geq 0}$:

    \begin{enumerate}
        \item База индукции: $\ker \phi^0 = \ker E = {\overline{0}} \subseteq \ker \phi \ \forall \phi$ и аналогично $\im \phi^0 = \im E = V \supseteq \im \phi \ \forall \phi$.
        \item Докажем, что $\ker \phi^n \subseteq \ker \phi^{n+1}$ (где $n \in \N$):
        Если $x \in \ker \phi^n$, тогда $\phi^n x = 0$ и $\phi ^{n + 1} x = \phi (\phi ^ n x) = \phi (\overline{0}) = \overline{0}$. \\
        Докажем теперь аналогичное вложение для образов: пусть $y \in \im \phi^{n + 1}$, тогда существует $x$, такой что $y = \phi^{n + 1} x = \phi^n(\phi(x)) = \phi^n z \in \im \phi^n$. Следовательно, $\im \phi^{n + 1} \subseteq \im \phi^n$.
    \end{enumerate}
\end{proof}

\begin{algorithm}[Построение Жордановой диаграммы]

    Покажем, как это использовать для нахождения Жордановой матрицы. Обозначим размерности ядер за $n_i$ соответственно: $\dim \ker \phi^i = n_i$. Выпишем для одного подпространства $U_{\lambda}$ все вложенные в него:
    \begin{eqnarray}  
        \{0\} \subseteq \ker \phi_{\lambda} = \langle f_{11}, f_{12}\rangle \subseteq \ker \phi_{\lambda}^2 
        = \langle f_{11}, f_{21}, f_{12}, f_{22} \rangle \subseteq \ker \phi_{\lambda}^3 = 
        \langle f_{11}, f_{21}, f_{12}, f_{22}, f_{13} \rangle, \\ (n_1 = 2, n_2 = 4, n_3 = 5).
    \end{eqnarray}
    Тогда число точек в Жордановой диаграмме на высоте $j$ равно $d_j = n_j - n_{j-1}$, откуда для нашего примера соответствующие $d$ равны $d_1 = 2-0=2, d_2 = 4-2=2, d_3 = 5-4=1$.
    
    Если в корневом пространстве $V^{\lambda}$ ввести обозначения $d_j$ - число векторов (точек) на высоте $j$, то $d_j = n_j - n_{j-1}$, где $n_0 = 0$, $n_k = \dim \ker (\phi - \lambda_i E)^k$.
    Это работает, потому что при применении оператора $j$ раз обнулятся все векторы на высоте не выше $j$, 
    тогда при применении на $1$ раз меньше обнулятся все, кто ниже, искомое количество - те, кто обнуляется при применении $j$ раз и не обнуляется при применении на 1 раз меньше.
    
    Строим ядра (и образы) до тех пор, пока они не стабилизируются (будут равны).
\end{algorithm}

\begin{theorem}[]
    Жорданова нормальная форма линейного оператора $\phi$ опеределена однозначно с точностью до перестановки Жордановых клеток, стоящих на главной диагонали. 
    Утверждение складывается из двух промежуточных:
    \begin{enumerate}
        \item Сумма порядков клеток, относящихся к собственному значению $\lambda_i$, не зависит от выбора Жорданова базиса.
        \item Для оператора $\phi$, имеющего единственное собственное значение, порядки Жордановых клеток определяются однозначно.
    \end{enumerate}
\end{theorem}

\begin{proof}~
    \begin{enumerate}
        \item  Зафиксируем Жорданов базис и корневое подпростарнство $V^{\lambda_i}$ и выберем все векторы Жорданова базиса, относящиеся к $\lambda_i$. Обозначим $V(\lambda_i) = \langle f_{ij} | f_{ij}$ относящиеся к $\lambda_i \rangle$. \\
        Пусть $l_i$ -- максимальный порядок Жордановых клеток Жордановой матрицы, отвечающих $\lambda_i$, $(J_k(\lambda_i) - \lambda_i \epsilon)^{l_i} = 0$. 
        Оператор нильпотентен и за несколько его применений все векторы базса обратятся в 0.
        Таким образом $(\phi - \lambda_i \epsilon)^{l_i} \vert_{V(\lambda_i)} = 0$.
        $\forall i V(\lambda_i) \subseteq V^{\lambda_i}$ -- так как все векторы аннулируются.
        \begin{enumerate}
            \item $V = V^{\lambda_1} \oplus V^{\lambda_2} \oplus \dots \oplus V^{\lambda_k}$.
            \item $V = V(\lambda_1) \oplus V(\lambda_2) \oplus \dots \oplus V(\lambda_k)$.
        \end{enumerate}
        По теореме о характеризации прямой суммы второе выражение является прямой суммой, а значит верны вложения и в обратную сторону(из соображений размерности).
        \item Пусть единственное собственное значение -- 0. Покажем, что размеры клеток в Жордановой нормальной форме определены однозначно. 
        Как было доказано на предыдущих лекциях, из того, что оператор нильпотентен, существует разложение в прямую сумму циклических подпространств.
        \begin{center}
            \includegraphics[width = 0.2\textwidth]{images/lec6_5.PNG}
        \end{center}
        Длины строк определены однозначно: $d_j = n_j - n_{j-1}$, $n_j = \dim \ker \phi^j$. Таким образом порядок клеток тоже можно определить однозначно.
    \end{enumerate}
\end{proof}

\section{11. Аннулирующий и минимальный многочлен линейного оператора. Связь минимального многочлена с жордановой нормальной формой.}

\begin{definition}
    $\phi: V \to V$, $P \in F[t]$ называется \textit{аннулирующим} для оператора $\phi$, если $P(\phi) = 0$ (иначе говоря: $\forall x \in V \hookrightarrow P(\phi) = \overline{0}$).
\end{definition}

\begin{note}
    Если $\dim V = n$, то у любого $\phi$ существует аннулирующий многочлен.
\end{note}

\begin{proof}
    Если $\phi$ соответствует матрица $A$ размером $n$ на $n$ и $\dim M_n(F) = n^2$. \\
    Тогда если рассмотреть все матрицы вида $E, A, A^2, \dots, A^{n^2}$, то существуют $\alpha_i \in F: \sum_{i = 0}^{n^2} \alpha_i A^i = 0$, тогда аннулирующий многочлен выглядит как $P = \sum_{i = 0}^{n^2} \alpha_i t^i$.
\end{proof}

\begin{definition}
    Аннулирующий многочлен для $\phi$ минимальной возможной степени называется \textit{минимальным многочленом} оператора $\phi$ и обозначается: $\mu_{\phi}$.
\end{definition}

\begin{theorem}
    \label{th4.5}
    Пусть $\phi: V \to V$, $\mu(t)$ -- минимальный многочлен $\phi$ и пусть $P(t)$ -- аннулирующий многочлен оператора $\phi$. Тогда $P \vdots \mu$.
\end{theorem}

\begin{proof}
    Пусть $P(t) = Q(t) \cdot \mu(t) + R(t)$, $\deg R < \deg \mu$ или $R = 0$. \\
    От противного, пусть $R \neq 0$ тогда выразим этот остаток из предыдущего выражения: 
    $R(\phi) = P(\phi) - Q(\phi) \cdot \mu (\phi) = 0$ -- так как аннулирующий и минимальный 
    многочлены зануляются, то и остаток равен нулю. Противоречие. Значит, $\mu(t) \vert P(t)$.
\end{proof}

\begin{corollary}
    Минимальный многочлен линейного оператора $\phi$ определяется с точностью до ассоциированности.
\end{corollary}

\begin{proposition}
    Пусть матрица отображения имеет вид Жордановой клетки: $J = J_k(\lambda)$. 
    Тогда его минимальный многочлен имеет вид $\mu_j(x) = (x - \lambda)^k$.
\end{proposition}

\begin{proof}
    Так как матрица отображения имеет вид Жордановой клетки, 
    его характеристический многочлен $\chi_J(x)$ представляется как:
    $$\chi_J(x) = (\lambda - x)^k = (-1)^k (x - \lambda)^k \sim (x - \lambda)^k.$$
    По теореме \ref{th4.4} Гамильтона-Кэли $\mu_J \vert \chi_J$, значит $\mu_J (x) = (x - \lambda)^t$, $t \leq k$.
    Если $t < k$, то $(J - \lambda \epsilon)^t \neq 0$, 
    что приводит к противоречию с определением минимального многочлена $\mu_J(J) = 0$.  
    Таким образом $t$ не может быть меньше $k$, а значит $t = k$. 
\end{proof}

\section{12. Норма в линейном пространстве. Норма линейного оператора. Вычисление многочлена и аналитической функции от линейного оператора.}

\begin{definition}
    Функция называется \textit{аналитической}, если она представляется сходящимся степенным рядом.
\end{definition}

\begin{definition}
    Функция $||\cdot||: V \to \R$ называется \textit{нормой} если 
    \begin{enumerate}
        \item $||x|| > 0$, если $x \neq 0$,
        \item $|| \lambda x|| = |\lambda| \cdot ||x||$,
        \item $||x + y|| \leq ||x|| + ||y||$.
    \end{enumerate}
\end{definition}

\begin{definition}
    Последовательность векторов $\{x^m\}$ сходится по норме к $x_0$, 
    если $||x^m - x_0|| \to 0$ при $m \to +\infty$.
\end{definition}

\begin{definition}
    Ряд $\displaystyle\sum_{m=1}^{+\infty} x^m$ называется \textit{сходящимся}, если он сходится по норме 
    $S^n = \displaystyle\sum_{m=1}^{n}x^m$.
\end{definition}

\begin{definition}
    Ряд $\displaystyle\sum_{m=1}^{+\infty} x^m$ называется \textit{абсолютно сходящимся}, если сходится ряд 
    $\displaystyle\sum_{m=1}^{+\infty} ||x^m||$.
\end{definition}

\begin{proposition}
    Если ряд $\displaystyle\sum_{m=1}^{+\infty} a_m x^m$ сходится абсолютно, то он сходится, 
    и для сумм верно:
    $$||\displaystyle\sum_{m=1}^{+\infty} x^m|| \leq \displaystyle\sum_{m=1}^{+\infty} ||x^m||.$$
\end{proposition}

\begin{proposition}
    Если ряд $\displaystyle\sum_{m=1}^{+\infty} a_m x^m$ сходится абсолютно и $\phi: \N \to \N$, то 
    ряд $\displaystyle\sum_{m=1}^{+\infty} a_{\phi(m)} x^{\phi(m)}$ сходится и для этих двух рядов 
    верно: $$||\displaystyle\sum_{m=1}^{+\infty} a_{\phi(m)} x^{\phi(m)}|| = 
    ||\displaystyle\sum_{m=1}^{+\infty} a_m x^m||$$
\end{proposition}

\begin{definition}
    \label{def6.8}
    Пусть $\phi: V \to V$, $V$ конечномерно над $\R$ или $\Cm$. Тогда:
    $$||\phi|| \overset{\text{def}}{=} \underset{x \neq 0}{\max} \frac{||\phi(x)||}{||x||} = 
    \underset{||x|| = 1}{\max} \frac{||\phi(x)||}{||x||} = \underset{||x|| = 1}{\max} ||\phi(x)||.$$
\end{definition}

\begin{note}
    Если $\lambda$ -- собственное значение оператора $\phi$, то $||\phi|| \geq \lambda$.
\end{note}

\begin{proposition}[о свойствах нормы оператора]~
    \begin{enumerate}
        \item Определение \ref{def6.8} опеределяет норму в $\mathcal{L}(V)$.
        \item $||\phi(x)|| \leq ||\phi|| \cdot ||x||$.
        \item $||\phi \cdot \psi|| \leq ||\phi|| \cdot ||\psi||$.
    \end{enumerate}
\end{proposition}

\begin{proof}~
    \begin{enumerate}
        \item Докажем неравенство треугольника для нормы:
        \begin{multline*}
            ||\phi + \psi|| \overset{\text{def}}{=} \underset{x \neq 0}{\max} 
            \frac{||(\phi + \psi)(x)||}{||x||} \leq \underset{x \neq 0}{\max} 
            \frac{||\phi(x)|| + ||\psi(x)||}{||x||} \leq \\ \leq \underset{x \neq 0}{\max} 
            \frac{||\phi(x)||}{||x||} + \underset{x \neq 0}{\max} 
            \frac{||\psi(x)||}{||x||} = ||\phi|| + ||\psi||
        \end{multline*} 

        \item Докажем непосредственной проверкой:
        \begin{eqnarray*}
            ||\phi(x)|| = \frac{||\phi(x)||}{||x||} ||x|| \leq  \underset{x \neq 0}{\max} 
            \frac{||\phi(x)||}{||x||} ||x|| = ||\phi|| \cdot ||x||
        \end{eqnarray*}
        \item Докажем непосредственной проверкой:
        \begin{multline*}
            ||\phi \cdot \psi|| = \underset{x \neq 0}{\max} \frac{||\phi \cdot \psi(x)||}{||x||} = 
            \underset{\psi(x) \neq 0}{\max} \frac{||\phi(x)||}{||x||} =
            \underset{\psi(x) \neq 0}{\max} \frac{||\phi \cdot \psi(x)||}{||\psi(x)||} \cdot 
            \frac{||\psi(x)||}{||x||} \leq \\ \leq \underset{\psi(x) \neq 0}{\max} 
            \frac{||\phi \cdot \psi(x)||}{||\psi(x)||} \cdot \underset{\psi(x) \neq 0}{\max} 
            \frac{||\psi(x)||}{||x||} \leq ||\phi|| \cdot ||\psi||
        \end{multline*}
    \end{enumerate}
\end{proof}

\begin{theorem}
    Пусть ряд $f(t) = \displaystyle\sum_{m=1}^{+\infty} a_m t^m$ сходится при $|t| < R$.
    Тогда ряд $\displaystyle\sum_{m=1}^{+\infty} a_m \phi^m$ сходится абсолютно для любого оператора 
    $\phi: ||\phi|| = R_0 < R$. Более того, $f(\phi) = \displaystyle\sum_{m=1}^{+\infty} a_m \phi^m$ 
    - задает линейный оператор в $V$.
\end{theorem}

\begin{proof}
    $\forall x \in V$ докажем, что ряд $\displaystyle\sum_{m=1}^{+\infty} a_m \phi^m(x)$ 
    сходится абсолютно:
    \begin{multline*}
        \displaystyle\sum_{m} |a_m| \cdot ||\phi^m(x)|| \leq \displaystyle\sum_{m} |a_m| \cdot 
        ||\phi^m|| \cdot ||x|| \leq \\ \leq ||x|| \displaystyle\sum_{m} |a_m| \cdot ||\phi^m|| = 
        ||x|| \displaystyle\sum_{m} |a_m| R_0^m \text{\: -- \, сходится при } R_0 < R.   
    \end{multline*}
    Ряд $f(t) = \displaystyle\sum_{m} a_m t^m$ сходится при $|t| < R$, 
    а значит $\displaystyle\sum_{m} |a_m| |t|^m$ сходится при $|t| < R$ по теореме Абеля.
\end{proof}

\begin{note}
    \begin{gather*}
        exp(\phi) = \epsilon + \frac{\phi}{1!} + \dots + \frac{\phi^n}{n!} + \dots, \, R = +\infty \\
        sin(\phi) = \phi - \frac{\phi^3}{3!} + \dots + (-1)^n \frac{\phi^{2n+1}}{(2n+1)!} + \dots, 
        \, R = +\infty \\
        cos(\phi) = \epsilon - \frac{\phi^2}{2!} + \frac{\phi^4}{4!} - \dots + 
        (-1)^n \frac{\phi^{2n}}{(2n)!} + \dots, \, R = +\infty
    \end{gather*}
\end{note}

\section{13. Линейные рекурренты. Общий вид линейной рекурренты над произвольным полем (случай, когда характеристический многочлен раскладывается на линейные множители).}

\begin{definition}
    Будем рассматривать последовательности $(a_0, a_1, \dots)$, $a_i \in F$. Множество всех таких 
    последовательностей будем обозначать $F^{\infty}$.
\end{definition}

\begin{definition}
    \label{def7.1}
    Зафиксируем многочлен $p(x) \in F[x]$ степени $S$, 
    $p(x) = x^s + p_{s-1} x^{s-1} + \ldots + p_1 x + p_0$.
    Линейным рекуррентным соотношением с характеристическим многочленом $p(x)$ 
    называется последовательность $a_n$ такая что $\forall n \in \N \cap \{0\}$ верно:
    \begin{eqnarray*}
        a_{n+s} + p_{s-1} a_{n + s - 1} + \ldots + p_1 a_{n+1} + p_0 a_n = 0, \: p_0 \neq 0.
    \end{eqnarray*}
    Рекуррентное соотношение выражает $a_{n + s}$ через $s$ предыдущих членов.
    $V_p$ -- множество всех последовательностей, удовлетворяющих рекуррентному соттношению выше.
\end{definition}

\begin{proposition}
    $V_p$ -- линейное пространство над $F$ и $\dim V_p = s$.
\end{proposition}

\begin{proof}
    Если $\{a_n\}$ и $\{ b_n \}$ удовлетворяют условию определения \ref{def7.1}, 
    то и $\{a_n + b_n\}$ удовлетворяют этому условию.
    Базис в $V_p$:
    \begin{gather*}
        e_0 = (\underbrace{1,\, 0,\, 0,\, \dots,\, 0,}_{S}\, -p_0,\, \dots) \\
        e_1 = (\underbrace{0,\, 1,\, 0,\, \dots,\, 0,}_{S}\, -p_1, \dots) \\
        \dots \\
        e_{s-1} = (\underbrace{0,\, 0,\, 0,\, \dots,\, 1,}_{S}\, -p_{s-1}, \dots)
    \end{gather*}
\end{proof}

\begin{proposition}
    \label{pr7.2}
    Рассмотрим оператор $\phi: F^{\infty} \to F^{\infty}$, такой что 
    $\phi(a_0, a_1, \dots, a_n, \dots) = (a_1, a_2, \dots, a_{n-1}, \dots)$. 
    Тогда $V_p = \ker p(\phi)$.
\end{proposition}

\begin{proof}
    По определению ядра отображения последовательность $\{b_n\}$ лежит в $\ker p(\phi)$ тогда 
    и только тогда, когда верно $p(\phi) (b_n) = (0) \in F^{+\infty}$. При этом имеет место 
    следующая равносильность:
    \begin{eqnarray*}
        (\phi^s + p_{s-1} \phi^{s-1} + \dots p_1 \phi + p_0 \epsilon) (b_n) = (0) \Leftrightarrow
        b_{n+s} + p_{s-1} b_{n+s-1} + \dots p_1 b_{n+1} + p_0 b_n = (0)
    \end{eqnarray*}
    Второе равенство эквивалентно тому, что $\{b_n\}$ лежит в $V_p$, а значит верно вложение $V_p$ и 
    $\ker p(\phi)$ друг в друга в обе стороны.
\end{proof}

\begin{note}
    Оператор $\phi$ называется оператором левого сдвига. $V_p$ инварианто относительно $\phi$.
\end{note}

\begin{corollary}
    Пусть $\psi_p = \phi \vert_{V_p}$. Тогда $p(\psi_p) = 0$.
\end{corollary}

\begin{proof}
    $p(\phi) \vert_{V_p} = 0$ так как $V_p = \ker p(\phi)$.
\end{proof}

\begin{proposition}
    $\mu_{\psi_p} (x) = p(x)$.
\end{proposition}

\begin{proof}
    Пусть $a_n \in V_p$, тогда $p(\phi) (a_n) = (0)$. По следствию из утверждения \ref{pr7.2} 
    для сужения $\psi_p = p(\phi) \vert_{V_p}$ так же верно $\psi_p (a_n) = (0)$, а
    значит $p(\psi_p) (a_n) = 0$. Таким образом $p$ - аннулирующий многочлен для $\psi_p$ и по 
    теореме \ref{th4.5} $\mu \vert p$, где $\mu = \mu_{\psi_p}$.
    По определению минимального многочлена $\mu(\psi_p) = 0$, тогда и $\mu(\phi) \vert_{V_p} = 0$.

    Отсюда следует, что $V_p$ вложено в $\ker \mu(\phi) = V_{\mu}$ (равенство верно по утверждению 
    \ref{pr7.2}). Из вложенности $V_p \subseteq V_{\mu}$ и кратности $\mu \vert p$ получаем 
    равенство степеней многочленов $\deg p = \deg \mu$, откуда следует их ассоциированность.
\end{proof}

\begin{definition}
    Пусть $p(x) = x^s + p_{s-1} x^{s-1} + \dots + p_1 x + p_0 \in F[x]$, $p_0 \neq 0$.\\
    Сопутствующей матрицей для многочлена $p(x)$ называется матрица размера $s \times s$ вида:
    \begin{gather*}
        \begin{pmatrix}
        0      & 1      & 0      & \dots  & 0        & 0          \\
        0      & 0      & 1      & \dots  & 0        & 0          \\
        0      & 0      & 0      & \dots  & 0        & 0          \\
        \vdots & \vdots & \vdots & \ddots & \vdots   & \vdots     \\
        0      & 0      & 0      & \dots  & 0        & 1          \\
        -p_0   & -p_1   & -p_2   & \dots  & -p_{s-2} & -p_{s-1}
        \end{pmatrix}
    \end{gather*}
\end{definition}

\begin{proposition}
    \label{pr7.4}
    Пусть $\psi_p = \phi_p \vert_{V_p}$. В базисе $(e_0, e_1, \dots, e_{s-1})$ из стандартных 
    последовательностей оператор $\psi_p$ имеет в точности сопутвующую матрицу $A_p$.
\end{proposition}

\begin{proof}
    \begin{flalign*}
        &\psi_p(e_0) = (0, 0, \dots, 0, 0,\, -p_0, \dots) = -p_0 e_{s-1} \\
        &\psi_p(e_1) = (1, 0, \dots, 1, 0,\, -p_1, \dots) = e_0 - p_1 e_{s-1} \\
        &\dots \\
        &\psi_p(e_i) = (0, 0, \dots 1, 0, \dots, 0,\, -p_i, \dots) = e_{i-1} - p_i e_{s-1}
    \end{flalign*}
    При этом для $e_i$ единица стоит на $i-1$ позиции, $-p_i$ всегда стоит на $s$-й позиции.
\end{proof}

\begin{proposition}
    $\chi_{\psi_p} (x) = \chi_{A_p}(x) = (-1)^s p(x)$.
\end{proposition}

\begin{proof}
    Из утверждения \ref{pr7.4} следует $\chi_{\phi_p}(x) = \chi_{A_p}(x) = (-1)^s p(x)$.
    Докажем наше утверждение по индукции:
    \begin{enumerate}
        \item База $s = 2$:
        \begin{gather*}
            \begin{vmatrix}
                -x   & 1    \\
                -p_0 & x - p_1
            \end{vmatrix} = x^2 + p_1 x + p_0 \text{ -- верно.}
        \end{gather*} 
        \item Пусть $M_{2, 3, \dots , s}^{2, 3, \dots , s} = (-1)^{s - 1} (x^{s - 2} + p_{s - 1} x^{s - 2} + \dots + p_2 x + p_1)$, тогда:
        \begin{gather*}   
        \chi_{A_p} = -x \cdot (-1)^{s - 1} (x^{s - 1} + p_{s - 1} x^{s - 2} + \dots + p_2 x + p_1 + (-p_0)(-1)^{s - 1} \cdot M = \\ = (-1)^s(x^s + p_{s - 1} x^{s - 1} + \dots + p_0 x) + (-1)^s p_0 = (-1)^s(x^s + p_{s - 1} x^{s - 1} + \dots + p_1 x + p_0) = (-1)^s p(x),
        \end{gather*}
        где матрица $M$ имеет следующий вид:
        \begin{gather*}
        M = \begin{pmatrix}
            1 & 0  & \dots  & 0 \\
            -x         & 1            & \dots & \vdots \\
            \vdots         &  \vdots                   & \ddots           & \vdots\\
            0    & \dots               & \dots               & 1 \\
        \end{pmatrix}
    \end{gather*}
    \end{enumerate}
\end{proof}

\begin{theorem}[Основная теорема о линейных рекуррентах]
    Пусть $V_p$ -- пространство линейных рекуррент, относящихся к $p(x)$ и пусть $p(x)$ 
    раскладывается на линейные множители: $p(x) = \displaystyle\prod_{i=1}^{k}(x -\lambda_i)^{l_i}$,
    $\lambda_i \in F$ - попарно различные.
    Тогда для любой $\{a_n\}_{n=0}^{\infty} \in V_p$ справедливо представление:
    \begin{gather*}
        a_n = \sum_{i=1}^{k}\sum_{s=1}^{l_i} c_{is} \cdot C_n^{s - 1} \lambda_i^{n + 1 - s}
    \end{gather*}
\end{theorem}

\begin{proof}
    Ранее было доказано, что $\mu_{\psi_p} \sim \chi_{\psi_p}$. Теперь наша цель разложить пространство 
    в прямую сумму корневых подпространств и найти для каждого циклическое подпространство 
    $\langle b_1^{(i)}, \dots b_{l_i}^{(i)} \rangle$ такое, что:
    \begin{gather*}
        (\phi - \lambda_i \epsilon)b_1^{(i)} = 0, \\
        (\phi - \lambda_i \epsilon) b_{s}^{(i)} = b_{s-1}^{(i)}.
    \end{gather*}
    Заметим, что если $b_s^{(i)}$ построены, то они дают Жорданов базис в $V_p$. 
    При этом $l_1 + l_2 +\dots l_k = s = \dim V_p$. Получаем:
    \begin{gather*}
        \prod_{i=1}^{k}(\phi - \lambda_i \epsilon)^{l_i} (b_s^{(i)}) = 0 \Leftrightarrow p(\phi) (b_s^{(i)}) = 0 \Leftrightarrow b_s^{(i)} \in V_p
    \end{gather*}

    Для упрощения вычислений отбросим индекс $i$, считая, что мы всё время работаем с одним и тем же собственным значением.

    \begin{gather*}
        b_1 = (1, \lambda, \lambda^2, \dots, \lambda^n, \dots) \\
        (\phi - \lambda \epsilon)b_1 = (\lambda, \lambda^2, \dots, \lambda^{n + 1}, \dots) - (\lambda, \lambda^2, \dots, \lambda^{n + 1}, \dots) = 0
    \end{gather*}

    Таким образом мы доказали, что $b_1 = \{\lambda^n\}_{n=0}^{\infty}$ - собственный вектор.
    Пусть вектор высоты $s-1$ построен. Тогда $b_{s-1} = f_{s-1}(n) \lambda^n, b_s = f_s(n) \lambda^n$. Заметим, что: $$f_s(n+1) \lambda^{n+1} - f_s(n) \lambda^{n+1} - f_{s-1}(n) \lambda^{n} \vdots \lambda^{n + 1}.$$ Разделим на $\lambda^{n + 1}$: $$f_s(n+1) - f_s(n) = \frac{f_{s-1}(n)}{\lambda}.$$
    При $\lambda = 1$ решением этого уравнения является $f_s(n) = C_n^{s-1}$, что можно доказать 
    самостоятельно в качестве упражнения (на самом деле это следует из формулы $C_n^{s - 1} + C_n^s = C_{n + 1}^s$). В общем случае будем искать решение в виде квазимногочлена:
    $f_s(n) = C_n^{s-1} \cdot \lambda^{\alpha(s)}$. Подставим это решение в полученное выше уравнение:
    $$C_{n+1}^{s-1} \lambda^{\alpha(s)} + C_n^{s-1} \lambda^{\alpha(s)} 
    = C_n^{s-2} \lambda^{\alpha(s-1) - 1}.$$ В силу того, что $C_{n+1}^{s-1} = C_n^{s-1} + C_n^{s-2}$,
    получаем $\alpha(s) = \alpha(s-1) - 1$. В силу того, что при $s = 1$ мы должны получить собственный 
    вектор $b_1$, полученный ранее, верно $f_1(n) = 1$, а значит $\alpha(1) = 0$. 
    Тогда $\alpha(2) = \alpha(1) - 1 = -1$, и $\alpha(s) = 1 - s$. 
    Отсюда следует, что $f_s(n) = C_{n}^{s-1} \lambda^{1-s}$, а значит $b_s = C_{n}^{s-1} \lambda^{n+1-s}$.
    Таким образом, мы получили Жорданов базис, отвечающий $\lambda$: $b_1, b_2, \dots, b_l$.
\end{proof}

\section{14. Билинейные функции. Координатная запись билинейной функции. Матрица билинейной функции и ее изменение при замене базиса. Ортогональное дополнение к подпространству относительно симметричной (кососимметричной) билинейной функции и его свойства.}

\begin{definition}
    Пусть $V$ -- линейное пространство над $F$. Функция $f: V \times V \to F$ называется \textit{билинейной}, если выполняются следующие условия:
    \begin{enumerate}
        \item Аддитивность по первому аргументу $f(x_1 + x_2, y) = f(x_1, y) + f(x_2, y)$.
        \item Линейность по первому аргументу $f(\lambda x, y) = \lambda f(x, y)$.
        \item Аддитивность по второму аргументу.
        \item Линейность по второму аргументу.
    \end{enumerate}
\end{definition}

\begin{definition}
    Если $x, y \in F^n$, то выражение $\displaystyle\sum_{i=1}^{n} \displaystyle\sum_{j=1}^{m} a_{ij} x_i y_j$ называется \textit{билинейной формой} от координатных столбцов $x$ и $y$. Билинейная форма сама является билинейной функцией: $F^n \times F^n \to F$.
\end{definition}

\begin{proposition}
    Если $f(x, y)$ -- билинейная функция $V \times V \to F$, то она может быть записана в виде билинейной формы от координат $x$ и $y$ при добавлении коэффициентов $a_{ij} = f(e_i, e_j)$ - значения функции $f$ на базисных векторах. 
\end{proposition}

\begin{proof}
    Пусть $f(x, y)$ - билинейная функция, $e = (e_1, \dots e_n)$ - базис в $V$. Запишем разложения векторов $x$ и $y$ по базису:
    \begin{align*}
        x = \sum_{i=1}^{n} x_i e_i && y = \sum_{j=1}^{n} y_j e_j
    \end{align*}
    Тогда верно следующее:
    \begin{gather*}
        f(x, y) = f(\sum_{i=1}^{n} x_i e_i, \sum_{j=1}^{n} y_j e_j) = \sum_{i=1}^{n} \sum_{j=1}^{n} x_i y_j f(e_i, e_j) = \sum_{i=1}^{n} \sum_{j=1}^{n} a_{ij} x_i y_j
    \end{gather*}
\end{proof}

\begin{proposition}
    Пусть $f(x, y)$ -- билинейная функция в $V$. $e$, $e'$ -- базисы в $V$. $A$, $A'$ -- матрицы билинейной формы $f$ в этих базисах. Тогда $A' = S^T A S$, где $S$ -- матрица перехода между $e$ и $e'$. 
\end{proposition}

\begin{proof}
    Пусть $x$ и $y$ имеют в $e$ координаты $\alpha$ и $\beta$ соответственно.\\ Было доказано, что $\alpha = S \alpha'$, $\alpha` = s^{-1} \alpha$, $\beta = S \beta'$. Тогда:
    $$f(x, y) = x^T A y = \alpha^T A \beta = (S \alpha')^T A (S \beta) = (\alpha')^T S^T A S \beta' = (\alpha')^T A' \beta'.$$ Из последнего равенства сразу следует, что $S^T A S = A'$.
\end{proof}

\begin{definition}
    Билинейная функция $f(x, y)$ называется \textit{симметричной} если для всех $x, y \in V$ верно 
    $f(x, y) = f(y, x)$.
\end{definition}

\begin{definition}
    Билинейная функция $f(x, y)$ называется \textit{кососимметричной}, если для всех $x, y \in V$ верно:
    \begin{enumerate}
        \item $f(x, y) = -f(y, x)$,
        \item $f(x, x) = 0$.
    \end{enumerate}
\end{definition}

\begin{agreement}
    Многие утверждения, доказываемые в этом разделе верны и для симметричных и для кососимметричных 
    функций. Чтобы показать, что функция $f$ лежит в $B^+$ или в $B^-$ будем использовать 
    обозначение $f \in B^{\pm}$.
\end{agreement}

\begin{definition}
    \label{def8.5}
    Пусть $f \in B^{\pm}(V)$. Тогда ядром $f$ является: 
    $$\ker f = \{x \in V \vert \, \forall y \in V \hookrightarrow f(x, y) = 0\} = 
    \{y \in V \vert \, \forall x \in V \hookrightarrow f(x, y) = 0\},$$ 
    где сначала выписано левое ядро, а затем правое ядро, и они равны.
\end{definition}

\begin{definition}
    Пусть $U \leq V$. \textit{Ортогональным дополнением} к $U$ относительно функции $f \in B^{\pm}(V)$ 
    называется подпространство $U^{\perp} = \{y \in V \,\vert \, \forall x \in U \hookrightarrow 
    f(x, y) = 0\}$.
\end{definition}

\begin{definition}
    Подпространство $U \leq V$ назовем невырожденным относительно функции $f \in B^{\pm}(V)$, если 
    сужение $f$ на $U$ невырожденно.
\end{definition}

\begin{theorem}
    \label{th8.1}
    Пусть $U \leq V$, $f \in B^{\pm}(V)$. Тогда $U$ невырожденно относительно $f$ тогда и только 
    тогда когда $V$ раскладывается в прямую сумму подпространств: $V = U \oplus U^{\perp}$.
\end{theorem}

\begin{proof}~
    \begin{enumerate}
        \item Необходимость. Пусть $f \vert_{U}$ невырождено. Покажем, что тогда 
        $\ker f \vert_{U} = \{0\}$.
        \begin{multline*}
            \ker f \vert_{U} = \{y \in U \vert \, \forall x \in U \hookrightarrow f(x, y) = 0\} = \\
            = \{y \in V \vert \, \forall x \in U \hookrightarrow f(x, y) = 0\} \cap U 
            = U^{\perp} \cap U = \{0\},
        \end{multline*}
        где первое равенство получено по определению ядра $f$ над $U$, а третье по определению 
        ортогонального дополнения. Из соображений размерностей подпространств получим:
        \begin{multline*}
            dim (U + U^{\perp}) = \dim U + \dim U^{\perp} - \dim (U \cap U^{\perp}) = \\ =
            \dim U + \dim U^{\perp} \geq \dim U + \dim V - \dim U = \dim V.
        \end{multline*} 
        Так как $U + U^{\perp} \leq V$, получаем равенство размерностей $\dim (U + U^{\perp}) = \dim V$, 
        а значит и равенство подпространств: $U + U^{\perp} = V$. 
        
        По теореме о характеристике прямой суммы получаем $V = U \oplus U^{\perp}$.

        \item Пусть $V = U \oplus U^{\perp}$. Но $\ker (f \vert_{U}) = U \cap U^{\perp} = \{0\}$, 
        а значит $f$ невырождена на $U$.
    \end{enumerate}
\end{proof}

\section{15. Симметричные билинейные и квадратичные функции, связь между ними. Поляризационное тождество. Метод Лагранжа приведения квадратичной формы к каноническому виду.}

\begin{definition}
    Пусть $f \in B(V)$, $f: V \times V \to F$. Тогда $\Delta = \{(x, x) \in V \times V\}$ -- 
    \textit{диагональ} в пространстве $V$.
\end{definition}

\begin{definition} 
    Пусть $f \in B^{+}(V)$. \textit{Квадратичной функцией} на $V$ называется произвольное сужение симметричной 
    билинейной функции $f$ на диагональ $\Delta$:
    \begin{gather*}
        q(x) = f(x, y) \vert_{\Delta} = f(x, x): \, V \to F.
    \end{gather*} 
\end{definition}

\begin{agreement}
    Будем обозначать как $Q(V)$ множество всех квадратичных функций на $V$.
\end{agreement}

\begin{theorem}
    Линейные пространства $B^+(V)$ и $Q(V)$ изоморфны, изоморфизм осуществляет отображение $\phi$ 
    сужения на диагональ $\Delta \subset V \times V$.
\end{theorem}

\begin{proof}
    Пусть $\phi: B^+(V) \to Q(V)$, переводящее $f(x, x) \in B^+(V)$ в $q(x) \in Q(V)$. 
    Операции сложения и умножения на скаляр сохраняются. Покажем его биективность:
    \begin{enumerate}
        \item Отображение $\phi$ сюръективно по определению квадратичной функции.  
        \item Проверим инъективность $\phi$. Пусть $\phi(f) = q$, $\phi(g) = q$, покажем, что тогда 
        $f = g$. 
        
        По определению $q(x) = f(x, x)$, тогда:
        \begin{gather*}
            q(x \pm y) = f(x \pm y, x \pm y) = f(x, x) \pm 2 f(x, y) + f(y, y) =
            q(x) \pm 2 f(x, y) + q(y).
        \end{gather*}
        Аналогично $q(x \pm y) = q(x) \pm 2 g(x, y) + q(y)$, откуда:
        \begin{gather*}
            f(x, y) = \frac{1}{4} (q(x+y) - q(x-y)) = g(x, y).
        \end{gather*} 
    \end{enumerate} 
    Таким образом полученное отображение - биекция, сохраняющая необходимые операции, а значит 
    получен изоморфизм между $B^+(V)$ и $Q(V)$.
\end{proof}

\begin{definition}
    Выражение $f(x, y)$ через $q(x)$ и $q(y)$ называется \textit{поляризационным тождеством}.
    Обратное отображение $\psi:\, Q(V) \to B^+(V)$ называется \textit{поляризацией},
    $f(x, y)$ -- \textit{полярной функцией} к $q(x)$.
\end{definition}

\begin{definition}
    Базис в $V$ называется \textit{ортогональным} относительно $f$ если для всех $i$, $j$, $i \neq j$ верно 
    $a_{ij} = f(e_i, e_j) = 0$.
\end{definition}

\begin{theorem}[Лагранжа]
    Всякую билинейную симметричную функцию $f$ и ассоциированную с ней квадратичную функцию 
    подходящим выбором базиса можно привести к диагональному виду.
\end{theorem}

\begin{proof}
    Индукция по размерности пространства.
    \begin{enumerate}
        \item База: при $n=1$ матрица уже имеет диагональный вид.
        \item Предположение индукции: пусть для пространств $V$ размерности меньшей чем $n$ 
        теорема верна. Совершим переход к подпространствам размерности $n+1$.

        Если функция $f$ тождественно нулевая, её матрица так же очевидно диагональная. В случае ненулевой функции $f$ в силу поляризационного тождества функция $q$ так же является ненулевой. Тогда существует вектор $e_1$, такой что $q(e_1) = a_{11} = f(e_1, e_1) \neq 0$. Рассмотрим тогда $U = \langle e_1 \rangle$. Тогда $f \vert_{U}$ невырождена, а значит $V = U \oplus U^{\perp}$. 

        По предположению индукции в $U^{\perp}$ найдется ортогональный относительно сужения 
        $f \vert_{U^{\perp}}$ базис $(e_2, \dots e_n)$. Матрица $A_{U^{\perp}}$ в нем будет иметь вид:
        \begin{gather*}
            A_{U^{\perp}} = \begin{pmatrix}
                \lambda_2  & 0         & \dots  & 0         \\
                0          & \lambda_3 & \dots  & 0         \\
                \vdots     & \vdots    & \ddots & \vdots    \\
                0          & 0         & \dots  & \lambda_n
            \end{pmatrix}
        \end{gather*}
        В силу того, что $U$ и $U^{\perp}$ образуют прямую сумму, равную всему пространству $V$,
        при добавлении в $(e_2, \dots e_n)$ вектора $e_1$ получится базис в $V$, 
        являющийся ортогональным относительно $f$. Матрица $f$ в базисе $(e_1, e_2, \dots e_n)$ имеет 
        вид:
        \begin{gather*}
            A = \begin{pmatrix}
                \lambda_1  & 0         & \dots  & 0         \\
                0          & \lambda_2 & \dots  & 0         \\
                \vdots     & \vdots    & \ddots & \vdots    \\
                0          & 0         & \dots  & \lambda_n
            \end{pmatrix}
        \end{gather*}
        При этом коэффициенты в матрице равны $\lambda_i = q(e_i)$.
    \end{enumerate}
\end{proof}

\begin{definition}
    Пусть $F = \R$. Вид квадратичной функции
    \begin{align*}
        q(x) = x_1^2 + x_2^2 + \dots + x_p^2 - x_{p+1}^2 - \dots - x_{p+q}^2,
    \end{align*} 
    где $p + q = \rk q$, называется \textit{каноническим видом} квадратичной функции в $V$ над $\R$.
\end{definition}

\begin{corollary}
    Если $F = \R$, то всякую квадратическую функцию выбором базиса можно привести к каноническому виду 
    выбором базиса.
\end{corollary}

\begin{algorithm}[Поиск преобразования, приводящего к каноническому виду]~\\
    В нашем базисе $q(x)$ имеет следующее предстваление:
    \begin{gather*}
        q(x) = \sum_{i=1}^{n}\sum_{j=1}^{n} a_{ij}x_i x_j
    \end{gather*}
    Его можно преобразовать к виду: 
    \begin{gather*}
        q(x) = \frac{1}{a_{11}} (a_{11} x_1 + a_{12} x_2 + \dots + a_{1n} x_n)^2 - 
        \sum_{i=2}^{n}\sum_{j=2}^{n} a_{ij}x_i x_j.
    \end{gather*}
    Сумма после вынесения первого слагаемого не содержит $x_1$ ни в одном члене. Обозначим тогда 
    $(a_{11} x_1 + a_{12} x_2 + \dots + a_{1n} x_n)$ за $\xi_1$, который будет являться первым 
    искомым каноническим вектором. После этого $q(x)$ можно записать как:
    \begin{gather*}
        q(x) = \lambda_1  \xi_1^2 + \sum_{i=1}^{n}\sum_{j=1}^{n} a_{ij}x_i x_j.
    \end{gather*}
    Таким образом можно продолжать преобразования суммы до получения разложения в канонический вид:
    \begin{gather*}
        q(x) = \lambda_1  \xi_1^2 + \lambda_2 \xi_2^2 + \dots \lambda \xi_n^n
    \end{gather*}
    При этом столбцы матрицы $S$ будут являться координатами векторов базиса в каноническом базисе.
\end{algorithm}

\section{16. Индексы инерции квадратичной формы в действительном линейном пространстве. Закон инерции. Метод Якоби приведения квадратичной формы к диагональному виду.}

\begin{definition}
    Квадратичная функция $q(x)$ называется \textit{положительно определенной (отрицательно опеределенной)}, 
    если для всех $x \neq 0$ верно $q(x) > 0$ ($q(x) < 0$).
\end{definition}

\begin{definition}
    Квадратичная функция $q(x)$ называется \textit{положительно полуопределенной (отрицательно полуоопределенной)} 
    если для всех $x \in V$ верно $q(x) \geq 0$ ($q(x) \leq 0$).
\end{definition}

\begin{agreement}
    До конца раздела будем считать что $V$ -- поле над пространством действительных чисел.
\end{agreement}

\begin{definition}
    Пусть $e$ -- канонический базис. Представим $q(x)$ как: 
    $$q(x) = \xi_1^2 + \, \dots \,+ \xi_p^2 - \xi^2_{p+1} - \, \dots \,- \xi^2_{p+q}.$$ 
    Числа $p$ и $q$ называются \textit{индексами инерции} относительно канонического базиса $e$.
\end{definition}

\begin{theorem}[Закон инерции]
    Пусть $q \in Q(V)$, $e$ -- канонический базис в $V$, $p$ и $q$ -- положительный и отрицательный 
    индексы инерции относительно базиса $e$. Тогда верно следующее: 
    \begin{enumerate}
        \item $p = \max\{\dim U \, \vert \, U \leq V : q \vert_{U} \text{ -- положительно определена}\}$,
        \item $q = \max\{\dim U \, \vert \, U \leq V : q \vert_{U} \text{ -- отрицательно определена}\}$,
        \item Индексы $p$ и $q$ не зависят от выбора базиса в $V$.
    \end{enumerate}
\end{theorem}

\begin{proof}~
    \begin{enumerate}
        \item Пусть $e = (e_1, e_2, \, \dots \,e_n)$. Рассмотрим следующие подпространства V:
        \begin{align*}
            U_0 = \langle e_1, e_2, \, \dots \,e_p \rangle && W_0 = \langle e_{p+1}, e_{p+2}, \, \dots \,e_n\rangle. 
        \end{align*}
        Их размерности равны $\dim U_0 = p$ и $\dim W_0 = n-p$ соответственно.

        Пусть $m = \max \{ \dim U \vert U \leq V: q\vert_U \text{ -- положительно определена}\}$.

        По построению $U_0$ верно что $q \vert_{U_0}$ положительно определена, а значит $m \geq p$. Пусть $m > p$.
        Тогда по построению $m$ существует $U_1  \leq V$ такое, что $q \vert_{U_1}$ положительно определена
        и $\dim U_1 = m$.
        При этом по формуле Грассмана: 
        $$\dim (U_1 \cap W_0) = \dim U_1 + \dim W_0  - \dim (U_1 + W_0) = m + n - p - \dim (U_1 + W_0) \geq 
        m + n - p - n > 0.$$
        Тогда $\exists z \in U_1 \cap W_0$. Однако по построению этих подпространств получим:
        \begin{align*}
            z \in U_1 \, & \Rightarrow q(z) > 0, \\
            z \in W_0 & \Rightarrow q(z) \leq 0. \\
        \end{align*}
        Таким образом предположение $m > p$ приводит к противоречию из-за невозможности существования 
        нетривиального пересечения $U_1$ и $W_0$. Это значит, что $m = p$. 

        \item Доказательство аналогично первому пункту.
        \item Истинность утверждения вытекает из первых двух пунктов, так как размерность подпространств 
        не зависит от выбора базисов в них.
    \end{enumerate}
\end{proof}

\begin{definition}
    Пусть квадратичная билинейная форма $q$ представляется как: \begin{gather*}
        q \leftrightarrow \begin{pmatrix}
        a_{11} & a_{12} & \dots  & a_{1n}   \\
        a_{21} & a_{22} & \dots  & a_{2n}   \\
        \vdots & \vdots & \ddots & \vdots   \\
        a_{n1} & a_{n2} & \dots  & a_{nn}
        \end{pmatrix}
    \end{gather*}
    \textit{Главным минором} $\Delta_i$ называется определитель левой верхней подматрицы размера $i \times i$:
    \begin{gather*}
        \Delta_i = \begin{pmatrix}
            a_{11} & a_{12} & \dots  & a_{1i}   \\
            a_{21} & a_{22} & \dots  & a_{2i}   \\
            \vdots & \vdots & \ddots & \vdots   \\
            a_{i1} & a_{i2} & \dots  & a_{ii}
        \end{pmatrix}
    \end{gather*} 
\end{definition}

\begin{theorem}[Як\'{о}би]
    Пусть $q(x)$ -- квадратичная функция в линейном пространстве над $\R$, $A$ -- её матрица относительно 
    некоторого базиса $e$ в $V$ и пусть $\forall i \: = 1, \dots n$ верно $\Delta_i \neq 0$. Тогда 
    существует базис $e'$ в $V$ такой что в нем $q(x)$ принимает вид: 
    \begin{gather*}
        q(x) = \frac{\Delta_0}{\Delta_1} \xi_1^2 + \frac{\Delta_1}{\Delta_2} \xi_2^2 + \, \dots \,+ 
        \frac{\Delta_{n-1}}{\Delta_n} \xi_n^2, \: \text{где } \, \Delta_0 = 1. 
    \end{gather*}
    Более того, $e'$ можно выбрать так, что матрица перехода $S = S_{e \to e'}$ 
    является верхнетреугольной. 
\end{theorem}

\begin{proof}
    Индукция по $n$ -- размерности пространства $V$:
    \begin{enumerate}
        \item База $n = 1$:  
        
        В пространстве размерности $1$ форма принимает вид $q(x) = a_{11} x_1^2$. 

        Тогда можно осуществить переход $e_1 \to e'_1 = \frac{1}{a_{11}} e_1$. Для нового базисного вектора: 
        $$q(e'_1) = f(\frac{e_1}{a_{11}}, \frac{e_1}{a_{11}}) = \frac{1}{a_{11}^2} a_{11} = \frac{1}{a_{11}}.$$
        Тогда в новом базисе $q(x) = a_{11} \xi_1 = \frac{1}{\Delta_1} \xi_1^2$, что и требовалось.

        \item Пусть теорема справедлива для любого $V$ для которого верно $\dim V < n$. 
        
        Рассмотрим пространство $V$ размерности $n$, 
        и его подпространство $U = \langle e_1, e_2, \, \dots \,e_{n-1} \rangle$. 

        По предположению индукции существует базис $e' = \langle e'_1, e'_2, \dots e'_{n-1} \rangle$ в $U$
        такой что $q$ имеет вид:
        $$q(x)\vert_U = \frac{\Delta_0}{\Delta_1} \xi_1^2 + \frac{\Delta_1}{\Delta_2} \xi_2^2 + \dots + 
        \frac{\Delta_{n-2}}{\Delta_{n-1}} \xi_{n-1}^2,$$ и матрица перехода от него к нашему базису 
        имеет верхнетреугольный вид:
        \begin{gather*}
            S_{e \to e'} = \begin{pmatrix}
            S_{11} & S_{12} & \dots  & S_{1, n-1} \\
            0      & S_{22} & \dots  & S_{2, n-1} \\
            \vdots & \vdots & \ddots & \vdots     \\
            0      & 0      & \dots  & S_{n-1, n-1}
            \end{pmatrix}
        \end{gather*}
        При этом форма $q(x)\vert_{U}$ невырождена, так как $\Delta_{n-1} \neq 0$. 

        Тогда по теореме о невырожденном подпространстве $V = U \oplus U^{\perp}$, 
        где ортогональное дополнение $U^{\perp}$ используется в смысле $f$ ассоциированного с $q$, 
        $\dim U^{\perp} = 1$. 
        
        Заметим, что в $U^{\perp}$ есть ненулевой вектор $e$, для которого верно $f(e_n, e) \neq 0$. 
        
        В противном случае для любого вектора $e \in U^{\perp}$ верно $f(e_n, e) = 0$, что значит, 
        что все вектора $e \in U^{\perp}$ перпендикулярны $e_n$. При этом $e \perp U = \langle e_1, \dots e_n\rangle$, 
        откуда $e \in \ker f$. 
        
        Это противоречит тому, что $\dim (\ker f) = \dim V - \rk f = 0$, а значит необходимый нам вектор существует.
        
        Положим $f(e, e_n) = c \neq 0$, 
        тогда $f(e_n, \frac{e}{c}) = 1$. Пусть $e'_n = \frac{e}{c} \in U^{\perp}$, откуда $f(e_n, e'_n)  = 1.$

        Покажем, что $e' = \langle e'_1, \dots e'_{n-1}, e'_n\rangle$ -- искомый базис. 

        Рассмотрим матрицу перехода $S = S_{e \to e'}$. Заметим, что $S_{ni} = 0$ для всех $i < n$ 
        в силу того, что $e'_i \in U$, а значит при переходе к новому базису вектор $e_n$ не повлияет 
        на него. Таким образом матрица $S_{e \to e'}$ диагональна.

        Осталось показать, что в новом базисе форма $q$ имеет необходимый вид. Благодаря предположению 
        индукции мы имеем:
        $$q(x)\vert_U = \frac{\Delta_0}{\Delta_1} \xi_1^2 + \frac{\Delta_1}{\Delta_2} \xi_2^2 + \dots + 
        \frac{\Delta_{n-2}}{\Delta_{n-1}} \xi_{n-1}^2.$$ 

        Таким образом необходимо только показать, что коэффициент при $\xi_n$ равен 
        $\frac{\Delta_{n-1}}{\Delta_n}$. Заметим, что этот коэффициент равен $q(e'_n)$.

        Вектор $e'_n$ выражается через коэффициенты матрицы перехода и векторы начального базиса: 
        $$e'_n = S_{1n}e_1 + \dots S_{nn}e_n.$$ Тогда:
        \begin{equation*}
            \begin{cases}
                f(e_1, e'_n) = 0,         \\
                f(e_2, e'_n) = 0,         \\
                \dots                     \\
                f(e_{n-1}, e'_n) = 0,     \\
                f(e_n, e'_n) = 1.
            \end{cases}
        \end{equation*}

        Первые $n-1$ значений равны $0$ в силу того, что $e'_n \in U^{\perp}$, $e_i \in U$.

        Тогда $q(e'_n)$ можно выразить следующим образом: \begin{gather*}
            q(e'_n) = f(e'_n, e'_n) = f(S_{1n} e_1 + \, \dots \, + S_{n-1, n} e_{n-1} + S_{nn}e_n, e_n') = \\
            = S_{1n} \cdot f(e_1, e'_1) + \, \dots \, + S_{nn} \cdot f(e_n, e'_n) = S_{nn}. 
        \end{gather*}  
        
        Выразим $S_{nn}$ из системы выше:

        \begin{equation*}
            \begin{cases}
                f(e_1, e'_n) = f(e_1, S_{1n}e_1 + \dots S_{nn}e_n) = 0,         \\
                f(e_2, e'_n) = f(e_2, S_{1n}e_1 + \dots S_{nn}e_n) = 0,         \\
                \dots                                                           \\
                f(e_{n-1}, e'_n) = f(e_{n-1}, S_{1n}e_1 + \dots S_{nn}e_n) = 0, \\
                f(e_n, e'_n) = f(e_1, S_{1n}e_1 + \dots S_{nn}e_n) = 1.
            \end{cases}
        \end{equation*}

        В силу невырожденности $q$ матрица перехода невырождена, а значит и система уравнений невырождена,
        так как её матрица в точности является матрицей оператора $q$ в базисе $e$.
        
        Тогда по теореме Крамера для неё существует единственное решение и $S_{nn} = \frac{\Delta_{n-1}}{\Delta_n}$.

        Таким образом мы получили диагональную матрицу $S_{e \to e'}$ и необходимое нам представление 
        $q$ в базисе $e'$ для пространства размерности $V$, что завершает доказательство по индукции.
    \end{enumerate} 
\end{proof}

\section{17. Положительно определенные квадратичные функции. Критерий Сильвестра. Кососимметрические билинейные функции, приведение их к каноническому виду.}

\begin{proposition}~
    \label{pr10.1}
    \begin{enumerate}
        \item Функция $q(x)$ положительно определена тогда и только тогда когда приводится к каноническому
        виду с матрицей $E$.
        \item Функция $q(x)$ положительно полуопределена тогда и только тогда когда приводится к 
        каноническому виду с матрицей, не имеющей $-1$ на главной диагонали.
    \end{enumerate}
\end{proposition}

\begin{proof}~
    \begin{enumerate}
        \item \begin{enumerate}
            \item Необходимость. 
            
            Пусть $q(x)$ положительно определена. Рассмотрим канонический базис $e$.
            В этом базисе $i$-й элемент матрицы $q$ равен $a_{ii} = q(e_i) > 0$. 
            
            В силу того, что в каноническом 
            базисе матрица может иметь только значения $\pm 1$ и $0$ на главной диагонали, получаем $a_{ii} = 1$.
            Таким образом матрица формы $q$ является единичной.
            \item Достаточность. 
            
            Пусть $q$ приводится к каноническому виду с $E$. Тогда в каноническом базисе: 
            $$q(x) = \xi_1^2 + \, \dots \,+ \xi_n^2, \, \text{ где } \, n = \dim V.$$ 
            Это значит, что для всех $x \neq 0$ верно $q(x) > 0$, так как в каноническом базисе 
            $x$ представляется в виде $x = (\xi_1 \, \xi_2 \, \dots \, \xi_n)^T$. Таким образом 
            $q$ положительно определена.
        \end{enumerate} 

        \item \begin{enumerate}
            \item Необходимость.
            
            Пусть $q(x)$ положительно полуопределена. Тогда в каноническом базисе $i$-й элемент 
            главной диагонали матрицы $q$ равен $a_{ii} = q(x_i) \geq 0$, 
            откуда $a_{ii} \in \{0, 1\}$.

            \item Достаточность.
            
            Пусть в каноническом базисе $a_{ii} \in \{ 0, 1\}$. Тогда $q$ в нем имеет вид:
            $$q(x) = \xi_1^2 + \, \dots \,+ \xi_p^2, \, \text{ где } \, p < \dim V.$$
            Таким образом для всех $x$ верно $q(x) \geq 0$, что значит, что $q$ положительно
            полуопределена.
        \end{enumerate}
    \end{enumerate}
\end{proof}

\begin{lemma}
    \label{pr10.3}
    Пусть $B \in M_n(\R)$ -- квадратная матрица над полем вещественных чисел. Тогда $B$ положительно 
    определена тогда и только тогда, когда существует невырожденная $A \in M_n(\R)$ такая, что 
    $B = A^T A$.
\end{lemma}

\begin{proof}~
    \begin{enumerate}
        \item Необходимость.
        
        Пусть $B$ положительно определена. Тогда она является матрицей некоторой 
        квадратичной функции $q$, что значит, что существует матрица $S = S_{e \to e'}$ такая, что 
        $S^T B S = E$. 
        
        Домножим выражение на $(S^T)^{-1}$ слева и на $S^{-1}$ справа и получим $B = (S^T)^{-1} S^{-1}$.

        Тогда искомая $A$ существует и равна $A = S^{-1}$. 
        \item Достаточность.
        
        Пусть $B = A^T A$, $\det A \neq 0$ (в силу невырожденности $A$). Тогда положим $S = A^{-1}$. 

        В новом базисе $B' = S^TBS = (A^{-1})^T A^T A A^{-1} = E$, откуда $B$ положительно определена по 
        утверждению \ref{pr10.1}.
    \end{enumerate}
\end{proof}

\begin{theorem}[Критерий Сильвестра]
    Пусть $q(x) \in Q(V)$. Тогда верно следующее:
    \begin{enumerate}
        \item Форма $q(x)$ положительно определена тогда и только тогда когда для всех $i$ главный минор 
        положителен: $\Delta_i > 0$.
        \item Форма $q(x)$ отрицательно определена тогда и только тогда когда знаки главных миноров чередуются:
        $\sgn(\Delta_i) = (-1)^i$.
    \end{enumerate}
\end{theorem}

\begin{proof}~
    \begin{enumerate}
        \item \begin{enumerate}
            \item Необходимость.
            
            Пусть $B$ -- матрица квадратичной функции $q(x)$ и $q$ положительно определена.
            Тогда по лемме \ref{pr10.3} верно $B = A^T A$, $\det A \neq 0$. В таком случае: 
            \begin{gather*}
                |B| = |A^T| \cdot |A| = |A|^2 > 0.
            \end{gather*}
            \item Достаточность.
            
            Пусть $\Delta_1 > 0, \dots \Delta_n > 0$. Тогда: 
            \begin{gather*}
                q(x) = \frac{\Delta_0}{\Delta_1} \xi_1^2 + \frac{\Delta_1}{\Delta_2} \xi_2^2 + \dots + 
                \frac{\Delta_{n-1}}{\Delta_n} \xi_n^2,
            \end{gather*}
            что значит, что $q(x)$ положительно определена так как при $x \neq 0$ верно $q(x) > 0$.
        \end{enumerate}
        \item Заметим, что если $q(x)$ положительно определена, то $-q(x)$ отрицательно определена. Пусть 
        $q(x)$ определена отрицательно, тогда $-q(x)$ определена положительно. Выпишем её матрицу:
        \begin{gather*}
            \begin{pmatrix}
                -a_{11} & -a_{12} & \dots  & -a_{2n} \\
                -a_{21} & -a_{22} & \dots  & -a_{2n} \\
                \vdots  & \vdots  & \ddots & \vdots  \\
                -a_{n1} & -a_{n2} & \dots  & -a_{nn}
            \end{pmatrix}
        \end{gather*}

        Тогда $\Delta_1 = -a_{11} > 0$, откуда $a_{11} < 0$. 
        
        Продолжим вычислять миноры:
        $\Delta_2 = \begin{vmatrix}
        -a_{11} & -a_{12}  \\
        -a_{21} & -a_{22}  
        \end{vmatrix} = \begin{vmatrix}
            a_{11} & a_{12}  \\
            a_{21} & a_{22}  
        \end{vmatrix} > 0$. 
        
        Вычисляя аналогично миноры большего размера получим, 
        что знак меняется на каждом шаге, что значит, что $\sgn(\Delta_i) = (-1)^i$.
    \end{enumerate}
\end{proof}

\begin{reminder}
    Билинейная функция $f(x, y)$ называется \textit{кососимметричной}, если для всех $x, y \in V$ верно:
    \begin{enumerate}
        \item $f(x, y) = -f(y, x)$,
        \item $f(x, x) = 0$.
    \end{enumerate}
\end{reminder}

\begin{definition}
    Базис $e = \langle e_1, \dots e_n \rangle$ называется \textit{симплектическим} для билинейной формы $f(x, y)$, 
    если для $S = 1,\dots n$ верно:
    \begin{gather*}
        f(e_{2S - 1}, e_{2S}) = 1 \Rightarrow f(e_{2S}, e_{2S-1}) = -1,
    \end{gather*} а для остальных значений $i, j$ верно $f(e_i, e_j) = 0$. 
    Матрица в таком случае имеет следующий вид: 
    \[A_f = \left(\begin{array}{@{}cccc@{}}
        \cline{1-1}
        \multicolumn{1}{|c|}{A_1} & 0 & \dots & 0\\
        \cline{1-2}
        0 & \multicolumn{1}{|c|}{A_2} & \dots & 0\\
        \cline{2-2}
        \vdots & \vdots & \ddots & \vdots\\
        \cline{4-4}
        0 & 0 & \dots & \multicolumn{1}{|c|}{A_m}\\
        \cline{4-4}
    \end{array}\right),\]
    
    где для всех $i$ матрица $A_i$ нулевая или имеет вид $A_i = 
    \begin{pmatrix}
        0  &1 \\
        -1 &0
    \end{pmatrix}$.
\end{definition}

\begin{theorem}[О каноническом виде кососимметричной билинейной функции]~

    Если $f(x, y)$ -- кососимметричная билинейная функция в $V$, то в $V$ существует симплектический базис.
\end{theorem}

\begin{proof}~
    Докажем по индукции по размерности пространства $V$. 
    \begin{enumerate}
        \item Если $f(x, y) = 0$ для всех $x, y$, то $S = 0$ -- очевидно.
        \item Если $f \neq 0$, то найдутся векторы $e_1, e_2$ такие, что $f(e_1, e_2) = c \neq 0$. 
        
        Рассмотрим тогда векторы $e_1' = e_1$, $e_2' = \frac{e_2}{c}$, для которых верно $f(e_1', e_2') = 1$.
        
        Тогда в $V$ существует невырожденное подпространство $U = \langle e'_1, e'_2 \rangle$,
        в котором матрица будет иметь вид $A_{f \vert_{U}} = \begin{pmatrix}
            0  &1 \\
            -1 &0
        \end{pmatrix}$.
        
        По теореме о невырожденном пространстве $V = U \oplus U^{\perp}$. Таким образом 
        если $\dim V = 2$, то искомый базис получен. Иначе по предположению
        индукции искомый базис найдется для $U^{\perp}$, а значит при объединении с $e'_1$ и $e'_2$ 
        получим базис для $V$.
    \end{enumerate}
\end{proof}

\section{18. Полуторалинейные формы в комплексном линейном пространстве. Эрмитовы полуторалинейные и квадратичные формы, связь мужду ними. Приведение их к каноническому виду. Закон инерции для эрмитовых квадратичных форм. Критерий Сильвестра.}

\begin{definition}
    Если рассматривать $V$ над $\Cm$, то в $V$ не бывает положительных функций в привычном нам виде.
    Для сравнения функции с $0$ на комплексных значениях будем считать, что если  $q(x) > 0$, то 
    $q(ix) = f(ix, ix) = -f(x,x) = -q(x) < 0$.
\end{definition}

\begin{definition}
    \textit{Полуторалинейными функциями} будем называть такие $f: V \times V \to \Cm$, для которых верны: 
    \begin{enumerate}
        \item Аддитивность по первому аргументу: $f(x_1 + x_2, y) = f(x_1, y) + f(x_2, y)$,
        \item Однородность по первому аргументу: $f(\lambda x, y) = \lambda f(x, y)$ для всех $\lambda \in \Cm$,
        \item Аддитивность по второму аргументу: $f(x, y_1 + y_2) = f(x, y_1) + f(x, y_2)$.
        \item $f(x, \lambda y) = \overline{\lambda} f(x, y)$.
    \end{enumerate}
\end{definition}

\begin{definition}
    Пусть $f$ -- полуторалинейная функция на $V$, $e$ -- базис в $V$, и векторы $x, y \in V$ имеют 
    координаты $x \leftrightarrow (x_1, x_2, \dots x_n)^T$, $y \leftrightarrow (y_1, y_2, \dots y_n)$.
    \textit{Полуторалинейной формой} от $x$, $y$ называют:
    \begin{gather*}
        f(x, y) = \sum_{i=1}^{n} \sum_{j=1}^{n} a_{ij} x_i \overline{y_j} = x^T A \overline{y}.
    \end{gather*} 
\end{definition}

\begin{proposition}
    Пусть $f$ -- полуторалинейная функция в $V$, $e$, $f$ -- базисы в $V$, $S$ -- матрица перехода 
    $S = S_{e \to f}$ и функция $f$ представляется в базисах $V$ матрицами
    $f \underset{e}{\leftrightarrow} A$, $f  \underset{f}{\leftrightarrow} B$, то $B = S^T A \overline{S}$.
\end{proposition}

\begin{proof}
    В базисе $e$ функция $f$ выражается как $f(x, y) = x^T A \overline{y}$. При переходе к базису 
    $f$ получим $x = S x'$, $y =  S y'$. Тогда:
    $$f(x, y) = (Sx')^T A \overline{(Sy')} = (x')^T S^TA \overline{S} \overline{y'} = (x')^TB\overline{y'},$$
    откуда $B = S^TA\overline{S}$.
\end{proof}

\begin{definition}
    Полуторалинейная функция $f(x, y)$ называется \textit{эрмитовой} (или \textit{эрмитово-симметричной}) если для всех 
    $x, y \in V$ верно $f(x, y) = \overline{f(y, x)}$. Матрица называется эрмитово-симметричной если 
    $A = \overline{A^T}$. 
\end{definition}

\begin{note}
    Комплексное сопряжение $\overline{A}$ к матрице $A$ стоит воспринимать как замену всех её элементов 
    на комплексно-сопряженные к ним.
\end{note}

\begin{proposition}
    Полуторалинейная функция $f$ эрмитова тогда и только тогда, когда в произвольном базисе $e$ её 
    матрица эрмитова.
\end{proposition}

\begin{proof}~
    \begin{enumerate}
        \item Необходимость. Пусть $f$ эрмитова. Тогда верно: \begin{gather*}
            a_{ij} = f(e_i, e_j) = \overline{f(e_j, e_i)} = \overline{a_{ji}}.
        \end{gather*} Отсюда следует $A = \overline{A^T}$.
        \item Достаточность. Пусть $A = \overline{A^T}$, откуда $A^T = \overline{A}$.
        Тогда: 
        \begin{gather*}
            f(x, y) = (x^T A \overline{y}) = (x^T A \overline{y})^T = \overline{y^T} A^T x = 
            \overline{y^T} A^T \overline{\overline{x}} = \overline{y^T A \overline{x}} = \overline{f(y, x)}.
        \end{gather*}
    \end{enumerate}
\end{proof}

\begin{definition}
    Пусть $\Delta = \{(x, x) \vert x \in V\}$ -- диагональ декартового квадрата. Тогда функция $q: V \to \Cm$ 
    назвается эрмитовой квадратичной функцией $q(x) = f(x, x) = f \vert_{\Delta}$, где $f$ -- эрмитова 
    симметричная функция.
\end{definition}

\begin{theorem}[О существовании канонического базиса]~

    Пусть $q$ -- эрмитова квадратичная функция (или соответствующая ей эрмитова симметричная функция $f$).
    Тогда в $V$ существует базис $e$, в котором матрица $q(f)$ диагональна, причем на главной диагонали 
    стоят числа $\pm 1$ и $0$.
\end{theorem}

\begin{idea}
    Пусть $q \neq 0$. Тогда в $V$ существует такой ненулевой вектор $e_1$, что $q(e_1) \neq 0$.
    Без ограничения общности можно перейти к $q(e_1) = \pm 1$. \\ Тогда можно рассмотреть пространство 
    $U = \langle e_1 \rangle$ и ортогональное дополнение к нему, образующие прямую сумму.
\end{idea}

\begin{proposition}[Закон инерции для квадратичных эрмитовых функций]
    Пусть $e$ - произвольный канонический базис для $q(x)$ и пусть $p, q$ - положительнй и отрицательный
    индексы инерции относительно $e$. Тогда:
    \begin{enumerate}
        \item $p = \max \{\dim U | U \leq V: q \vert_{U} \text{ -- положительно опеределена}\}$.
        \item $q = \max \{\dim U | U \leq V: q \vert_{U} \text{ -- отрицательно опеределена}\}$.
        \item $p$ и $q$ не зависят от выбора канонического базиса.
    \end{enumerate}
\end{proposition}

\begin{proof}
    Доказательство аналогично билинейному случаю.
\end{proof}

\begin{proposition}[Аналог критерия Сильвестра]
    Пусть $q(x) \in H(V)$ -- эрмитова квадратичная функция, $A$ -- её матрица в произвольном базисе, где 
    выполняется условие эрмитовости $\tilde{A^T} = A$. Тогда:
    \begin{enumerate}
        \item $q(x)$ положительно определена тогда и только тогда, когда $\Delta_1 > 0$, $\Delta_2 > 0$, $\dots$, 
        $\Delta_n > 0$.
        \item $q(x)$ отрицательно определена тогда и только тогда, когда $\Delta_1 < 0$, $\Delta_2 > 0$, $\dots$, 
        $\sgn (\Delta_n) = (-1)^n$.
    \end{enumerate}
\end{proposition}

\begin{proof}
    Доказательство аналогично билинейному случаю.
\end{proof}

\section{19. Евклидово и эрмитово пространство. Выражение скалярного произведения в координатах. Матрица Грама системы векторов и ее свойства. Неравенства Коши-Буняковского и треугольника.}

\begin{definition}
    Пусть $V$ - линейное пространство над полем действительных чисел. $V$ называется Евклидовым, если 
    на нем определена положительно определенная билинейная симметрическая функция $f(x, y)$. По 
    определению $f(x, y)$ назвается скалярным произведением и обозначается $(x, y)$.
\end{definition}

\begin{definition}
    Пусть $V$ - линейное пространство над $\Cm$. $V$ называется эрмитовым, если на $V$ определена 
    положительно определенная эрмитова полуторалинейная функция $f(x, y)$. Аналогично с евклидовыми 
    пространствами $f(x, y)$ называется скалярным произведением и обозначается $(x, y)$.
\end{definition}

\begin{definition}
    Матрицей Грама системы $a_1, a_2, \dots a_k$ называется матрица 
    \begin{gather*}
        \Gamma(a_1, \dots a_n) = \begin{pmatrix}
        (a_1, a_1)      & (a_1, a_2)      & \dots  & (a_1, a_n)       \\
        (a_2, a_1)      & (a_2, a_2)      & \dots  & (a_2, a_n)       \\
        \vdots & \vdots & \ddots & \vdots   \\
        (a_n, a_1)      & (a_n, a_2)      & \dots  & (a_n, a_n)
        \end{pmatrix}
    \end{gather*}
\end{definition}

\begin{theorem}
    \begin{enumerate}
        \item
        Пусть $e_1, e_2, \dots e_n$ - базис в $V$, $\Gamma = \Gamma(e)$. Тогда $\forall x, y \in V$ верно 
        $(x, y) = x^T \Gamma \tilde{y}$
        \item Пусть $a_1, a_2, \dots a_k$ - произвольная система векторов в $V$. Тогда $|\Gamma(a_1, \dots a_n)| \geq 0$, 
        прричем равенство достигается тогда и только тогда, когда система линейно зависима.
    \end{enumerate}
\end{theorem}

\begin{proof}
    \begin{enumerate}
        \item $f(x, y) = x^T A \tilde{y} = x^T \Gamma \tilde{y}$.
        \item Пусть система линейно независима. Тогда $U = \langle a_1, a_2, \dots a_k \rangle$, 
        $f \vert_{U}$ - положительно определена, а значит по критерию Сильвестра $|\Gamma(a_1, \dots a_n)| > 0$.

        Пусть теперь система линейно зависима и без ограничения общности $a_k = \lambda_1 a_1 + \dots + \lambda_{k-1} a_{k-1}$.
        Тогда нижняя строка будет состоять из нулей в силу того, что $(a_k, a_i) = (\lambda_1 a_1 + \dots + \lambda_{k-1} a_{k-1}, a_i)$
    \end{enumerate}
\end{proof}

\begin{theorem}[Неравенство Коши-Буняковского]
    Пусть $V$ - пространство со скалярным произведением, и пусть $x, y \in V$. Тогда 
    \begin{gather*}
        |(x, y)|^2 \leq (x, x) \cdot (y, y)
    \end{gather*}
\end{theorem}

\begin{proof}
    \begin{enumerate}
        \item Пусть $x$ или $y$ - нулевой вектор, тогда $0 = 0$.
        \item Пусть $x, y \neq 0$ и коллинеарны, то есть $y = \lambda x$. Тогда 
        \begin{gather*}
            |(x, \lambda x)|^2 = |\lambda|^2 |(x, x)|^2 = \lambda \tilde{\lambda} |(x, x)|^2 = 
            (x, x) (y, y)
        \end{gather*}
        \item Пусть $x, y \neq 0$ и неколлинеарны. Тогда система из $x$ и $y$ линейно независима, а значит 
        по теореме 1:
        \begin{gather*}
            0 < |\Gamma(x, y)| = (x, x)(y, y) - (x, y)(y, x) = (x, x)(y, y) - |(x, y)|^2.
        \end{gather*}
    \end{enumerate}
\end{proof}

\begin{corollary}[Неравенство треугольника]
    Для всех $x, y \in V$ верно:
    \begin{gather*}
        |x + y| \leq |x| + |y|
    \end{gather*}
\end{corollary}

\begin{proof}
    Неравенство треугольника эквивалентно неравенству $(x +y, x+y) \leq (x, x) + 2\sqrt{(x, x)(y, y)} + (y, y)$.
    $(x, x) + (y, y) + 2 \Re (x, y)$
\end{proof}

\section{20. Ортонормированные базисы и ортогональные (унитарные) матрицы. Существование ортонормированного базиса в пространстве со скалярным произведением. Изоморфизм евклидовых и эрмитовых пространств. Канонический изоморфизм евклидова пространства и сопряженного к нему.}

\begin{definition}
    Система векторов $x_1, x_2, \dots x_k$ называется ортогональной тогда и только тогда, когда 
    $(x_i, x_j) = 0$ для всех $i \neq j$.
\end{definition}

\begin{definition}
    Система векторов $x_1, x_2, \dots x_k$ называется ортонормированной тогда и только тогда, когда 
    она ортогональна и нормирована. Нормированность означает, что $(x_i, x_i) = 1$ для всех $i$.
\end{definition}

\begin{definition}
    Система подпространств $U_1$, $U_2, \dots$, $U_k$ называется ортогональной тогда и только тогда,
    когда для любой системы векторов $u_1 \in U_1$, $u_2 \in U_2$, $\dots u_k \in U_k$ верно, 
    что она ортогональна.
\end{definition}

\begin{definition}
    Матрица $A \in M_n(\R)$ называется ортогональной, если $A^T A = E$, откуда так же $A A^T = E$.
\end{definition}

\begin{definition}
    Матрица $A \in M_n(\Cm)$ называется унитарной, если $\tilde{A^T} A = E = A \tilde{A^T}$.
\end{definition}
 
\begin{proposition}
    Пусть $V$ - пространство со скалярным произведением, $e$ - ортонормированный базис в $V$, 
    $f$ - произвольный базис в $V$. Тогда матрица перехода $S = S_{e \to f}$ является 
    ортонональной тогда и только тогда, когда $f$ - ортонормированный базис.
\end{proposition}

\begin{proof}
    Пусть $f(x, y)$ имеет матрицу $\Gamma$. Тогда так как $e$ - ортонормированный базис, $\Gamma(e) = E$.
    Тогда $\Gamma(f) = S^T \Gamma(e) \tilde{S} = S^T \tilde{S}$.
\end{proof}

\begin{proposition}
    Пусть $V$ - пространство со скалярным произведением. Тогда в нём существует ортонормированный базис.
\end{proposition}

\begin{proof}
    Пусть $f(x, y) = (x, y)$, тогда для неё существует канонический базис, в котором $f$ имеет 
    матрицу $E$. $f(e_i, e_j) = (e_i, e_j) = \delta_{ij}$, откуда этот базис - ортонормированный.
\end{proof}

\begin{definition}
    Пусть $V_1$ и $V_2$ "--- евклидовы (эрмитовы) пространства. Отображение $\phi: V_1 \rightarrow V_2$ называется \textit{изоморфизмом евклидовых (эрмитовых) пространств}, если:
    \begin{enumerate}
        \item $\phi$ "--- изоморфизм линейных пространств $V_1$ и $V_2$
        \item $\forall \overline{u}, \overline{v} \in V_1: (\overline{u}, \overline{v}) = (\phi(\overline{u}), \phi(\overline{v}))$
    \end{enumerate}
\end{definition}

\begin{theorem}
    Пусть $V_1$ и $V_2$ "--- евклидовы (эрмитовы) пространства. Тогда $V_1 \cong V_2 \hm\lra \dim{V_1} = \dim{V_2}$.
\end{theorem}

\begin{proof}~
    \begin{itemize}
        \item[$\Leftarrow$]Пусть $e_1$, $e_2$ "--- ортонормированные базисы в $V_1$ и $V_2$, $\phi$ "--- линейное отображение такое, что $\phi(e_1) = e_2$. Тогда $\phi$ "--- изоморфизм линейных пространств, причем для любых $\overline{u}, \overline{v} \in V_1$, $\overline{u} \leftrightarrow_{e_1} x, \overline{v} \leftrightarrow_{e_1} y$, выполнено $(\overline{u}, \overline{v}) = x^TE\overline{y} \hm{=} x^T\overline{y} = (\phi(\overline{u}), \phi(\overline{v}))$.
        \item[$\Rightarrow$]Поскольку $V_1 \cong V_2$, то они в частности изоморфны как линейные пространства, откуда $\dim{V_1} = \dim{V_2}$.\qedhere
    \end{itemize}
\end{proof}

Рассмотрим $V$ "--- евклидово пространство.

\begin{definition}
    \textit{Сопряженным к V пространством} называется пространство линейных функционалов на $V$. Обозначение "--- $V^*$.
\end{definition}

\begin{theorem}
    Для каждого $\overline{v} \in V$ положим $f_{\overline{v}}(\overline{u}) := (\overline{v}, \overline{u})$. Тогда сопоставление $\overline{v} \mapsto f_{\overline{v}}$ осуществляет изоморфизм между $V$ и $V^*$.
\end{theorem}

\begin{proof}
    Проверим, что заданное сопоставление линейно:
    \begin{gather*}
        f_{\overline{v_1} + \overline{v_2}}(\overline{u}) = (\overline{v_1} + \overline{v_2}, \overline{u}) = (\overline{v_1}, \overline{u}) + (\overline{v_2}, \overline{u}) = f_{\overline{v_1}}(\overline{u}) + f_{\overline{v_2}}(\overline{u})\\
        f_{\alpha\overline{v}}(\overline{u}) = (\alpha\overline{v}, \overline{u}) = \alpha(\overline{v}, \overline{u}) = \alpha f_{\overline{v}}(\overline{u})
    \end{gather*}
    
    Поскольку $\dim{V} = \dim{V^*}$ и отображение линейно, то нам достаточно проверить его инъективность, что эквивалентно условию $\forall \overline{v} \in V, \overline{v} \ne \overline{0}: f_{\overline{v}} \ne 0$. Но это условие выполнено в силу положительной определенности скалярного произведения: $\forall \overline{v} \in V, \overline{v} \ne \overline{0}: f_{\overline{v}}(\overline{v}) \hm= (\overline{v}, \overline{v}) > 0$.
\end{proof}

\section{21. Ортогональное дополнение к подпространству. Задача об ортогональной проекции и ортогональной составляющей. Процедура ортогонализации Грама-Шмидта. Объем параллелепипеда.}

\begin{definition}
    Пусть $U \leq V$. Ортогональным дополнением к $U$ относительно функции $f \in B^{\pm}(V)$ 
    называется подпространство $U^{\perp} = \{y \in V \,\vert \, \forall x \in U \hookrightarrow 
    f(x, y) = 0\}$.
\end{definition}

\begin{proposition}
\label{pr 12.2}
    Пусть $U \subseteq V$, тогда $U^{\perp} = \psi (U^{\circ})$, где $U^{\circ}$ -- аннулятор пространства $U$ в $V^*$.
\end{proposition}

\begin{proof}
    $y \in U^{\perp} \Leftrightarrow \forall x \in U \hookrightarrow (x, y) = 0 \Leftrightarrow \forall x \in U f_y(x) = 0 \Leftrightarrow f_y \in U^{\circ} \Leftrightarrow \psi(f_y) \in \psi(U^{\circ})$. Значит, мы доказали, что для любого вектора из ортогонального дополнения его образ принадлежит образу аннулятора, а так как в обратную сторону очевидно, то $U^{\perp} = \psi(U^{\circ})$.
\end{proof}

\begin{proposition}
    Свойства ортогонального дополнения:
    \begin{enumerate}
        \item $(U^{\perp})^{\perp} = U$
        \item $(U + W)^{\perp} = U^{\perp}$
        \item $(U \cap W)^{\perp} = U^{\perp} + W^{\perp}$
    \end{enumerate}
\end{proposition}

\begin{proof}
    \begin{enumerate}
        \item $x \in (U^{\perp})^{\perp} \Leftrightarrow \forall y \in U^{\perp} (x, y) = 0$. Но, с другой стороны, $\forall x \in U (x, y) = 0$. Значит, любой вектор из $U$ лежит в $(U^{\perp})^{\perp}$. И из того, что размерности равны, следует равенство пространств: $\dim (U^{\perp})^{\perp} = \dim V - \dim U^{\perp} = \dim V - (\dim V - \dim U) = \dim U$.
        \item По утверждению \ref{pr 12.2}: 
        \begin{gather*}
            (U + W)^{\perp} = \psi((U + W)^{\circ}) = \psi(U^{\circ} \cap W^{\circ}) = \psi(U^{\circ} \cap \psi(W^{\circ}) = U^{\perp} \cap W^{\circ}
        \end{gather*}
    \end{enumerate}
\end{proof}

\begin{problem}[Задача об ортогональной проекции]
    Пусть $V$ -- пространство со скалярным произведением, $U$ -- подпространство $V$. Обозначим 
    размерность $V$ за $n$, размерность $U$ за $k$. Тогда сужение на $U$
    невырожденной функции $f(x, y)$, являющейся скалярным произведением в $V$, так же будет являться
    скалярным произведением и в $U$. \\
    Пространство $V$ будет представляться как $V = U + U^{\perp}$. \\
    Дан базис в $U$, вектор $x \in V$. Требуется представить вектор $x$ в виде суммы его проекций 
    $\tilde{x} = \pr_U x$ и $\stackrel{o}{x} = \ort_U x$ на $U$ и $U^{\perp}$ соответственно.
\end{problem}

\begin{algorithm}~
    \begin{enumerate}
        \item Зафиксируем в $U$ ортонормированный базис $e_1, \dots e_k$, достроив его до базиса 
        $e_1, \dots e_n$ в $V$. Тогда:
        \begin{gather*}
            x = \sum_{i=1}^{k} \alpha_i e_i + \sum_{i=k+1}^{n}\alpha_i e_i.
        \end{gather*}
        При этом $\displaystyle\sum_{i=1}^{k} \alpha_i e_i \in U$, 
        $\displaystyle \sum_{i=k+1}^{n}\alpha_i e_i \in U^{\perp}$. Тогда 
        $\tilde{x} = \displaystyle\sum_{i=1}^{k}\alpha_i e_i$, откуда $\stackrel{o}{x} = x - \tilde{x}$.
        \item Зафиксируем в $U$ ортогональный базис $e_1, e_2, \dots e_k$, достроив его до 
        ортогонального базиса $e_1, \dots e_n$ в $V$. Тогда рассмотрим базис $e'$ такой, что
        $e'_i = \frac{e_i}{|e_i|}$, очевидно являющийся ортонормированным. Тогда 
        \begin{gather*}
            \tilde{x} = \displaystyle\sum_{i=1}^{k} (x, e'_i)e'_i = \displaystyle\sum_{i=1}^{k} (x, e'_i) 
            \frac{e_i}{|e_i|} = \displaystyle\sum_{i=1}^{k} \frac{(x, e_i)}{(e_i, e_i)} e_i = 
            \pr_e x = \frac{(x, e)}{(e, e)} e.
        \end{gather*}
        \item Зафиксируем произвольный базис $e_1, e_2, \dots e_k$ в $U$, достроив его до базиса
        $e_1, \dots e_n$ в $V$.
        Тогда необходимые нам векторы выражаются как $\tilde{x} = \displaystyle\sum_{i=1}^{k} \lambda_i e_i$ и $\stackrel{o}{x} = x - \tilde{x} = 
        x - \displaystyle\sum_{i=1}^{k} \lambda_i e_i \perp e_1, \, \dots , \, e_k$.
        
        Чтобы получить коэффициенты $\lambda_i$ запишем следующую систему:
        \begin{eqnarray*}
            \begin{cases*}
                (x - \displaystyle\sum_{i=1}^{k} \lambda_i e_i, e_1) = 0,
                \\
                (x - \displaystyle\sum_{i=1}^{k} \lambda_i e_i, e_2) = 0,
                \\
                \dots
                \\
                (x - \displaystyle\sum_{i=1}^{k} \lambda_i e_i, e_k) = 0.
            \end{cases*} \Leftrightarrow \begin{cases*}
                (e_1, e_1) \lambda_1 + (e_2, e_1) \lambda_2 + \, ... \, + (e_k, e_1) = (x, e_1),
                \\
                (e_1, e_2) \lambda_1 + (e_2, e_2) \lambda_2 + \, ... \, + (e_k, e_2) = (x, e_2),
                \\
                \dots
                \\
                (e_1, e_k) \lambda_1 + (e_2, e_k) \lambda_2 + \, ... \, + (e_k, e_k) = (x, e_k).
            \end{cases*}
        \end{eqnarray*}
        Матрица системы является сужением $\Gamma$ на $U$, а значит $|\Gamma \vert_{U} (e_1, \, \dots, \, e_k)| > 0$.
        Таким образом по теореме Крамера система имеет единственное решение.
    \end{enumerate}
\end{algorithm}

\begin{theorem}[метод Грама-Шмидта]
    Пусть $(\overline{f_1}, \dots, \overline{f_n})$ "--- базис в $V$. Тогда в $V$ существует ортогональный базис $(\overline{e_1}, \dots, \overline{e_n})$ такой, что $\forall k \hm{\in} \{1, \dots, n\}: \langle\overline{e_1}, \dots, \overline{e_k}\rangle \hm= \langle\overline{f_1}, \dots, \overline{f_k}\rangle$, причем матрица перехода $S$ "--- верхнетреугольная с единицами на главной диагонали.
\end{theorem}

\begin{proof}
    Положим $\overline{e_1} := \overline{f_1}$ и $\overline{e_k} := \overline{f_k} \hm{-} \pr_{\langle\overline{f_1}, \dots, \overline{f_{k - 1}}\rangle}\overline{f_k}$ при всех $k \in \{2, \dots, n\}$. Тогда матрица перехода $S$ "--- верхнетреугольная с единицами на главной диагонали, поэтому $(\overline{e_1}, \dotsc, \overline{e_n})$ является базисом в $V$. Проверим равенство $\langle\overline{e_1}, \dots, \overline{e_k}\rangle \hm= \langle\overline{f_1}, \dots, \overline{f_k}\rangle$ индукцией по $k$. База, $k = 1$, тривиальна. Пусть теперь $k > 1$, тогда: $\langle\overline{e_1}, \dots, \overline{e_{k-1}}, \overline{e_k}\rangle \hm= \langle\overline{e_1}, \dots, \overline{e_{k-1}}, \overline{f_k}\rangle = \langle\overline{f_1}, \dots, \overline{f_{k-1}}, \overline{f_i}\rangle$.
\end{proof}

\begin{note}
    Получим явную формулу для $\overline{e_k}$ при всех $k \in \{2, \dots, n\}$:
    \[\overline{e_k} = \overline{f_k} - \pr_{\langle\overline{f_1}, \dots, \overline{f_{k - 1}}\rangle}{\overline{f_k}} = \overline{f_k} - \pr_{\langle\overline{e_1}, \dots, \overline{e_{k - 1}}\rangle}{\overline{f_k}} = \overline{f_k} - \sum_{j = 1}^{k - 1}\frac{(\overline{f_k}, \overline{e_j})}{||\overline{e_j}||^2}\overline{e_j}\]
\end{note}

\begin{corollary}
    Пусть $(\overline{e_1}, \dots, \overline{e_k})$ "--- ортогональная система ненулевых векторов из $V$. Тогда в $V$ существует ортогональный базис $(\overline{e_1}, \dots, \overline{e_n}) \supset (\overline{e_1}, \dots, \overline{e_k})$.
\end{corollary}

\begin{proof}
    Дополним систему $(\overline{e_1}, \dots, \overline{e_k})$ до произвольного базиса и применим метод Грама-Шмидта. Тогда базис станет ортогональным, при этом первые $k$ векторов в нем не изменятся, поскольку $\forall i \hm{\in} \{1, \dots, k\}: \overline{e_i} \hm{\mapsto} \overline{e_i} - \pr_{\langle\overline{e_1}, \dots, \overline{e_{i - 1}}\rangle}{\overline{e_i}} = \overline{e_i}$.
\end{proof}

\begin{definition}
    Определеим объем системы векторов по индукции:
    \begin{enumerate}
        \item $V_1(x_1) = |x_1|$
        \item $V_k(x_1, \dots x_k) = V_{k -1} (x_1, \dots, x_{k - 1}) \rho(x_k, \langle x_1, \dots, x_{k - 1} \rangle)$
    \end{enumerate}
\end{definition}

\begin{corollary}
    $V_k(x_1, \dots, x_k) \geq 0$, причем равенство достигается только в том случае, когда $\exists i \hookrightarrow \rho(x_k, \langle x_1, \dots, x_{k - 1} \rangle) = 0$. Что возможно только когда система $\langle x_1, \dots, x_{k - 1} \rangle$ -- линейно зависима.
\end{corollary}

\begin{proposition}
    $\rho (U, x) = |\ort_U x|$
\end{proposition}

\begin{proof}
    $|x - u| \geq |\stackrel{\circ}{x - u}|$ -- по определению. Тогда по теореме Пифагора: $|\stackrel{\circ}{x} - \stackrel{\circ}{u}| = |\stackrel{\circ}{x}|$ -- ортогональное дополнение. Значит $\inf_{y \in U} |x - y| \geq |\stackrel{\circ}{x}|$ -- достигается.
\end{proof}

\begin{theorem}[о геометрическом свойстве определителя Грама системы векторов]
    Если $x_1, \dots, x_k$ -- система векторов в пространстве со скалярным произведением, то $(V_k)^2(x_1, \dots, x_k) = |\Gamma(x_1, \dots, x_k)|$
\end{theorem}

\begin{proof}
    Если система $x_1, \dots, x_k$ -- линейно зависима, то $0 = 0$ -- теорема выполняется. Пусть система линейно независима.
    \begin{enumerate}
        \item Покажем, что преобразование Грама-Шмидта не изменяет левую и правую части равенства: для этого возьмем унипотентную матрицу перехода $S$: $(y_1, \dots, y_k) = (x_1, \dots, x_k)S$, тогда:
        $$|\Gamma(y_1, \dots, y_k)| = |S^T \Gamma(x_1, \dots, x_k)S| = |\det S|^2 |\Gamma(x_1, \dots, x_k)| = |\Gamma (x_1, \dots, x_k)|$$
        \item Теперь покажем равенство квадратов объемов индукцией по $k$: при $k = 1$ -- очевидно, что $y_1 = x_1$. Теперь пусть $V_{k - 1} (x_1, \dots, x_{k - 1}) = V_{k - 1}(y_1, \dots, y_{k - 1})$. По определению объема делаем шаг индукции:
        $$\rho(x_k, \langle x_1, \dots, x_{k -1} \rangle) = |\ort_{\langle x_1, \dots, x_{k -1} \rangle} x_k| = |\ort_{\langle x_1, \dots, x_{k -1} \rangle} y_k| = |\ort_{\langle y_1, \dots, y_{k -1} \rangle} y_k| = \rho(y_k, \langle y_1, \dots, y_k \rangle)$$
        \item Теперь равенство достаточно доказать для ортонормированного базиса:
        \[(V_k)^2(y_1, \dots, y_k) = (V_{k - 1})^2(y_1, \dots, y_k) \rho^2(y_k, \langle y_1, \dots, y_k \rangle) = \]\[ = (V_{k - 1})^2(y_1, \dots, y_k) |y_k|^2 = \displaystyle\sum_{i = 1}^{k} (y_i, y_i) = |\Gamma (y_1, \dots, y_k)\]
    \end{enumerate}
\end{proof}

\begin{definition}
    Параллелепипедом, порожденным $a_1, a_2, \ldots, a_n$, называется множество $P(a_1, \ldots, a_n) = \{\sum_{i = 1}^{n}\alpha_i a_i , \ 0 \leq \alpha_i \leq 1\}$. 
\end{definition}

\begin{definition}
    Пусть $V$ -- евклидово пространство с определенной ориентацией (базис положительно определен). $V_{or}P(a_{1}, \ldots, a_{n}) = \epsilon V_{n}(a_{1}, \ldots, a_{n})$, $\epsilon = 
    \begin{cases}
        1,  \text{ если } (a_1, \ldots, a_n) \text{ положит. опр. } \\
        -1, \text{ иначе }
    \end{cases}$ 
\end{definition}

\section{22. Преобразование, сопряженное данному. Существование и единственность такого преобразования, его свойства. Теорема Фредгольма.}

\begin{definition}
    Пусть $\phi \in \mathcal{L}(V)$. Для всех $\overline{u}, \overline{v} \in V$ положим $f_\phi(\overline{u}, \overline{v}) := (\phi(\overline{u}), \overline{v})$.
\end{definition}

\begin{definition}
    Пусть $\phi \in \mathcal{L}(V)$. Оператором, \textit{сопряженным к $\phi$}, называется оператор $\phi^* \in \mathcal{L}(V)$ такой, что $f_\phi = g_{\phi^*}$, то есть $\forall \overline{u}, \overline{v} \in V: (\phi(\overline{u}), \overline{v}) \hm{=} (\overline{u}, \phi^*(\overline{v}))$.
\end{definition}

\begin{note}
    Поскольку сопоставления $\phi \mapsto f_\phi = g_{\phi^*} \hm{\mapsto} \phi^*$ биективны, то сопряженный оператор $\phi^*$ существует и единственен.
\end{note}

\begin{note}
    $(\phi(\overline{u}), \overline{v}) \hm{=} (\overline{u}, \phi^*(\overline{v})) \lra A_{\phi^*} = \Gamma^{-1}A_{\phi}^{T}\Gamma$.
\end{note}

\begin{proposition} Сопряженные операторы обладают следующими свойствами:
    \begin{enumerate}
        \item $\forall \phi, \psi \in \mathcal{L}(V): (\phi + \psi)^* = \phi^* + \psi^*$
        \item $\forall \phi, \psi \in \mathcal{L}(V): (\phi\psi)^* = \psi^*\phi^*$
        \item $\forall \phi \in \mathcal{L}(V): \phi^{**} = \phi$
    \end{enumerate}
\end{proposition}

\begin{proof}
    Доказательство вытекает из свойств матриц соответсвующих операторов.
\end{proof}

\begin{theorem}[Фредгольма]
    Пусть $\phi \in \mathcal{L}(V)$. Тогда $\ke{\phi^*} \hm{=} (\im{\phi})^\perp$.
\end{theorem}

\begin{proof}~
    \begin{itemize}
        \item[$\subset$] Пусть $\overline{v} \in \ke{\phi^*}$, тогда $\phi^*(\overline{v}) = \overline{0}$, и $\forall \overline{u} \in V: (\phi(\overline{u}), \overline{v}) \hm{=} (\overline{u}, \phi^*(\overline{v})) = 0 \ra \overline{v} \in (\im{\phi})^\perp$.
        
        \item[$\supset$] Заметим, что $\rk{\phi} = \rk{\phi^*} = \dim{\im{\phi}} = \dim{\im{\phi^*}}$, тогда $\dim{\ke{\phi^*}} = \dim{(\im{\phi})^\perp}$, из чего следует требуемое в силу обратного включения.\qedhere
    \end{itemize}
\end{proof}

\section{23. Самосопряженное линейное преобразование. Свойства самосопряженных преобразований. Основная теорема о самосопряженных операторах (существование ортонормированного базиса из собственных векторов).}

\begin{definition}
    Оператор $\phi \in \mathcal{L}(V)$ называется \textit{самосопряженным}, если $\phi^* = \phi$, то есть $\forall \overline{u}, \overline{v} \in V: (\phi(\overline{u}), \overline{v}) = (\overline{u}, \phi(\overline{v}))$.
\end{definition}

\begin{note}
    Если самосопряженный оператор $\phi \in \mathcal{L}(V)$ в ортонормированном базисе имеет матрицу $A$, то $A \leftrightarrow_e \phi = \phi^* \leftrightarrow_e A^*$, то есть $A = A^*$ --- симметрична в евклидовом случае и эрмитова в эрмитовом случае.
\end{note}

\begin{proposition}
    Пусть $\phi \in \mathcal{L}(V)$, и подпространство $U \le V$ инвариантно относительно $\phi$. Тогда $U^\perp$ тоже инвариантно относительно $\phi^*$.
\end{proposition}

\begin{proof}
    Пусть $\overline{v} \in U^\perp$. Тогда $\forall \overline{u} \in U: (\overline{u}, \phi^*(\overline{v})) = (\phi(\overline{u}), \overline{v}) = (\phi(\overline{u}), \overline{v}) = 0$ в силу инвариантности $U$. Значит, $\phi^*(\overline{v}) \in U^\perp$.
\end{proof}

\begin{proposition}
    Пусть $\phi \in \mathcal{L}(V)$ "--- самосопряженный. Тогда его характеристический многочлен $\chi_\phi$ раскладывается на линейные сомножители над $\mathbb{R}$.
\end{proposition}

\begin{proof}
    Пусть сначала $V$ "--- эрмитово пространство, $\lambda \in \Cm$ "--- корень $\chi_\phi$. Тогда $\lambda$ является собственным значением оператора $\phi$ с собственным вектором $\overline{v} \in V$, $\overline{v} \ne \overline{0}$, откуда $\lambda||\overline{v}||^2 = (\phi(\overline{v}), \overline{v}) = (\overline{v}, \phi(\overline{v})) = \overline{\lambda}||\overline{v}||^2$. Значит, $\lambda = \overline\lambda \ra \lambda \in \R$.
    
    Пусть теперь $V$ "--- евклидово пространство с ортонормированным базисом $e$, тогда $\phi \leftrightarrow_e A \in M_n(\mathbb{R})$, $A = A^T$. Рассмотрим $U$ "--- эрмитово пространство той же размерности с ортонормированным базиом $\mathcal{F}$ и оператор $\psi \in \mathcal{L}(U)$, $\psi \leftrightarrow_{\mathcal{F}} A$. Тогда $\psi$ "--- тоже самосопряженный, поэтому для $\chi_\psi$ утверждение выполнено. Остается заметить, что $\chi_\psi \hm{=} \chi_A = \chi_\phi$.
\end{proof}

\begin{proposition}
    Пусть $\phi \in \mathcal{L}(V)$ "--- самосопряженный, $\lambda_1, \lambda_2 \hm{\in} \mathbb{R}$ "--- два различных собственных значения $\phi$. Тогда $V_{\lambda_1} \perp V_{\lambda_2}$.
\end{proposition}

\begin{proof}
    Пусть $\overline{v_1} \in V_{\lambda_1}, \overline{v_2} \in V_{\lambda_2}$. Тогда:
    \[\lambda_1(\overline{v_1}, \overline{v_2}) = (\phi(\overline{v_1}), \overline{v_2}) = (\overline{v_1}, \phi(\overline{v_2})) = \lambda_2(\overline{v_1}, \overline{v_2}) \Rightarrow (\overline{v_1}, \overline{v_2}) = 0\qedhere\]
\end{proof}

\begin{theorem}
    Пусть $\phi \in \mathcal{L}(V)$ "--- самосопряженный. Тогда в $V$ существует ортонормированный базис $e$, в котором матрица оператора $\phi$ диагональна.
\end{theorem}

\begin{proof}
    Проведем индукцию по $n := \dim{V}$. База, $n = 1$, тривиальна. Пусть теперь $n > 1$. Поскольку корни $\chi_\phi$ вещественны, то у $\phi$ есть собственное значение $\lambda_0 \in \mathbb{R}$. Пусть $\overline{e_0} \in V$ "--- соответствующий ему собственный вектор длины $1$. Тогда подпространство $U \hm{:=} \langle\overline{e_0}\rangle^\perp$ инвариантно относительно $\phi$, поэтому можно рассмотреть оператор $\phi|_{U} \in \mathcal{L}(U)$, который также является самосопряженным. По предположению индукции, в $U$ есть ортонормированный базис из собственных векторов, тогда его объединение с $\overline{e_0}$ дает искомый базис в $V$.
\end{proof}

\section{24. Ортогональные преобразования и их свойства. Канонический вид ортогонального преобразования. Инвариантные подпространства малых размерностей для линейного оператора в действительном линейном пространстве.}

\begin{definition}
    Оператор $\phi \in \mathcal{L}(V)$ называется \textit{ортогональным} (\textit{унитарным}), если $\forall \overline{u}, \overline{v} \in V: (\phi(\overline{u}), \phi(\overline{v})) = (\overline{u}, \overline{v})$.
\end{definition}

\begin{theorem}
    Пусть $\phi \in \mathcal{L}(V)$. Тогда оператор $\phi$ ортогонален (унитарен) $\hm\Leftrightarrow$ $\phi$ обратим и $\phi^{-1} = \phi^{*}$.
\end{theorem}

\begin{proof}
    По определению, $\phi$ ортогональнен (унитарен) $\hm{\Leftrightarrow}$ для любых векторов $\overline{u}, \overline{v} \in V$ выполнено $(\overline{u}, \overline{v}) = (\phi(\overline{u}), \phi(\overline{v})) = (\overline{u}, (\phi^*\phi)(\overline{v}))$. В силу единственности сопряженного оператора, это равносильно равенству $\phi^*\phi = \id^* = \id$. Это, в свою очередь, равносильно тому, что $\phi$ обратим и $\phi^{-1} \hm{=} \phi^{*}$.
\end{proof}

\begin{proposition}
    Пусть $\phi\in \mathcal{L}(V)$ "--- ортогональный (унитарный), $U \le V$. Тогда $U$ инвариантно относительно $\phi$ $\Leftrightarrow$ $U^\perp$ инвариантно относительно $\phi$.
\end{proposition}

\begin{proof}
    Поскольку $(U^\perp)^\perp = U$, то достаточно доказать импликацию $\ra$. Так как $U$ инвариантно относительно $\phi$, то $U^\perp$ инвариантно относительно $\phi^* = \phi^{-1}$, то есть $\phi^{-1}(U^\perp) \le U^\perp$. Но оператор $\phi$ биективен, поэтому $\phi^{-1}(U^\perp) = U^\perp$ и $\phi(U^\perp) = U^\perp$, откуда $U^\perp$ инвариантно относительно $\phi$.
\end{proof}

\begin{theorem}[о каноническом виде ортогонального оператора]
    Пусть $V$ - евклидово пространство, $\phi: V \to V$ -- ортогональный оператор. Тогда существует ортонормированный базис $e$, в котором матрица $\phi$ состоит из матриц поворота и единиц на главной диагонали.
     \[\phi = \left(\begin{array}{@{}cccc@{}}
        \cline{1-1}
        \multicolumn{1}{|c|}{R(\alpha_1)} & 0 & \dots & 0\\
        \cline{1-2}
        0 & \multicolumn{1}{|c|}{R(\alpha_2)} & \dots & 0\\
        \cline{2-2}
        \vdots & \vdots & \ddots & \vdots\\
        0 & 0 & \dots & 1\\
    \end{array}\right),\]
\end{theorem}

\begin{proof}
    $\phi$ имеет в $V$ одномерные или двумерные инвариантные подпространства. Пусть $U$ - 
    одномерное подпространство, или, если таких нет, двумерное инвариантное подпространство.
    \begin{enumerate}
        \item Пусть $\dim U = 1$, $e \in U$, $|e|= 1$. Покажем, что в таком случае модуль $\lambda$ равен единице. В $U$ верно $\phi(e) = \lambda e$. Тогда $(e, e) = (\phi(e), \phi(e)) = 
        \lambda^2 (e, e)$. Отсюда $\lambda^2 = 1$, а значит $\lambda = \pm 1$.
        \item Пусть $\dim V = 2$, $(e_1, e_2)$ - ортонормированный базис в $U$. Тогда $A^T A = E$. Найдем вид $A$.
        Пусть $ A = \begin{pmatrix}
                        a     & b \\
                        c     & d        
                    \end{pmatrix}$. Тогда:
        \begin{gather*}
            \begin{pmatrix}
                a     & c \\
                b     & d        
            \end{pmatrix} \begin{pmatrix}
                a     & b \\
                c     & d        
            \end{pmatrix} = \begin{pmatrix}
                1     & 0 \\
                0     & 1        
            \end{pmatrix}
        \end{gather*}
        Получим следующую систему уравнений:
        \begin{gather*}
            a^2 + c^2 = 1 (1) \\
            b^2 + d^2 = 1 (2) \\
            ab + cd = 0 (3) \\
        \end{gather*}
        Положим 
        \begin{gather*}
            a = \cos(\alpha), c = \sin(\alpha), b = - \sin(\beta), d = \cos(\beta)
        \end{gather*}
        Условия $(1)$ и $(2)$ очевидно выполняются. Проверим $(3)$ и найдем при помощи него связь 
        между углами $\alpha$ и $\beta$.
        \begin{gather*}
            -\cos(\alpha) \sin(\beta) + \sin(\alpha) \cos(\beta) = 0 \\
            \sin(\alpha - \beta) = 0 \hookrightarrow \alpha - \beta = \pi k, k \in Z
        \end{gather*}
        Рассмотрим случаи:
        \begin{enumerate}
            \item $\alpha = \beta$ -- по модулю $2\pi$.
            \item $\alpha = \beta + \pi$ -- по модулю $2\pi$.
            \item Покажем, что $\alpha + \beta = \pi$ быть не может:
            \begin{gather*}
                \cos(\beta) = \cos(\alpha - \pi) = -\cos(\alpha) \\
                \sin(\beta) = \sin(\alpha - \pi) = -\sin(\alpha) \Mapsto
                A = \begin{pmatrix}
                        \cos(\alpha)     & \sin(\alpha) \\
                        \sin(\alpha)     & -\cos(\alpha)       
                    \end{pmatrix}
            \end{gather*}
            Где $A^T = A$ и получаем два собственных вектора: $v_1 = (\cos(\frac{\alpha}{2}), \sin(\frac{\alpha}{2}))^T$, $v_{-1} = (-\sin(\frac{\alpha}{2}), \cos(\frac{\alpha}{2}))^T$ -- это противоречит с тем, что нет одномерных инвариантных подпространств.
        \end{enumerate}
        Теперь пространство $V$ раскладывается в прямую сумму $V = U \oplus U^{\perp}$. По предположению индукции для ортогонального дополнения $U$ теорема верна. Тогда она верна и для всего $V$.
    \end{enumerate}
\end{proof}

\begin{proposition}
    Пусть $V$ "--- линейное пространство над $\mathbb{R}$, $\dim{V} \ge 1$, $\phi \in \mathcal{L}(V)$. Тогда у $\phi$ существует одномерное или двумерное инвариантное подпространство.
\end{proposition}

\begin{proof}
    По основной теореме алгебры, минимальный многочлен $\mu_\phi$ имеет следующий вид:
    \[\mu_\phi(x) = \prod_{i = 1}^k(x - \alpha_i)\prod_{j = 1}^m(x^2 + \beta_jx + \gamma_j)\]
    
    Поскольку $\mu_\phi(\phi) = 0$, то хотя бы один из операторов $\phi - \alpha_i$, $\phi^2 + \beta_j\phi + \gamma_j$ "--- вырожденный. Более того, все они вырожденные в силу минимальности многочлена $\mu_\phi$. Значит, возможны два случая:
    \begin{enumerate}
        \item Если $\phi - \alpha$ "--- вырожденный, то $\exists \overline{v} \in V$, $\overline{v} \ne \overline{0}$ "--- собственный вектор с собственным значением $\alpha$, и $\langle\overline{v}\rangle \le V$ "--- искомое подпространство.
        \item Если $\phi^2 + \beta\phi + \gamma$ "--- вырожденный, то $\exists \overline{v} \in V$, $\overline{v} \ne \overline{0}: (\phi^2 + \beta\phi \hm{+} \gamma)(\overline{v}) \hm{=} \overline{0}$. Поскольку $\phi^2(\overline{v}) = -\beta\phi(\overline{v}) - \gamma\overline{v}$, то $\langle\overline{v}, \phi(\overline{v})\rangle \hm{\le} V$ "--- искомое подпространство.\qedhere
    \end{enumerate}
\end{proof}

\section{25. Полярное разложение линейного преобразования в евклидовом пространстве. Единственность полярного разложения для невырожденного оператора.}

\begin{theorem}
    Пусть $\phi \in \mathcal{L}(V)$. Тогда существуют $\psi, \Theta \in \mathcal{L}(V)$ такие, что $\psi$ "--- самосопряженный с неотрицательными собственными значениями, $\Theta$ "--- ортогональный (унитарный), и $\phi = \psi\Theta$.
\end{theorem}

\begin{proof}
    Рассмотрим оператор $\eta := \phi^*\phi$, тогда $\eta^* = \phi^*\phi \hm{=} \eta$, то есть $\eta$ "--- самосопряженный. Более того, если $\overline{v} \in V \backslash \{\overline{0}\}$ "--- собственный вектор оператора $\eta$ с собственным значением $\lambda \in \R$, то $\eta(\overline{v}) = \lambda\overline{v}$, тогда $0 \le (\phi(\overline{v}), \phi(\overline{v})) = (\overline{v}, \eta(\overline{v})) = \lambda(\overline{v}, \overline{v}) \Rightarrow \lambda \ge 0$.
    
    Пусть $(\overline{e_1}, \dots, \overline{e_n})$ "--- ортонормированный базис в $V$ из собственных векторов оператора $\eta$ с собственными значениями $\lambda_1, \dotsc, \lambda_n \ge 0$. Положим $\overline{f_i} := \phi(\overline{e_i})$, $i \in \{1, \dotsc, n\}$. Тогда для любых $i, j \in \{1, \dots, n\}$ выполнено $(\overline{f_i}, \overline{f_j}) = (\phi(\overline{e_i}), \phi(\overline{e_j})) = (\overline{e_i}, \eta(\overline{e_j})) = \lambda_j(\overline{e_i}, \overline{e_j})$. Значит, система $(\overline{f_1}, \dots, \overline{f_n})$ ортогональна, и, более того, для любого $i \in \{1, \dotsc, n\}$ выполнено $||\overline{f_i}||^2 = \lambda_i||\overline{e_i}||^2 = \lambda_i$.
    
    Будем без ограничения общности считать, что $\lambda_1, \dots, \lambda_k > 0$ и $\lambda_{k + 1} = \dots = \lambda_n = 0$. Положим $\overline{g_i} := \frac{1}{\sqrt{\lambda_i}}\overline{f_i}$, $i \in \{1, \dots, k\}$, и дополним $(\overline{g_1}, \dots, \overline{g_k})$ до ортонормированного базиса $(\overline{g_1}, \dots, \overline{g_n})$. Тогда оператор $\phi$ имеет следующий вид:
    $\overline{e_i} \mapsto \overline{g_i} \mapsto \sqrt{\lambda_{i}}\overline{g_i} = \overline{f_i}$. Зададим $\psi, \Theta \in \mathcal{L}(V)$ на базисах $(\overline{e_1}, \dots, \overline{e_n})$ и $(\overline{g_1}, \dots, \overline{g_n})$ следующим образом:
    \begin{align*}
        \Theta&: \overline{e_i} \mapsto \overline{g_i}\\
        \psi&: \overline{g_i} \mapsto \sqrt{\lambda_{i}}\overline{g_i} = \overline{f_i}
    \end{align*}
    
    Таким образом, $\psi\Theta = \phi$. Наконец, $\Theta$ переводит ортонормированный базис $(\overline{e_1}, \dots, \overline{e_n})$ в ортонормированный базис $(\overline{g_1}, \dots, \overline{g_n})$, поэтому $\Theta$ "--- ортогональный (унитарный), а $\psi$ имеет в ортонормированном базисе $(\overline{g_1}, \dots, \overline{g_n})$ диагональный вид, поэтому $\psi$ "--- самосопряженный.
\end{proof}

\begin{note}
    Порядок операторов в композиции несущественен: если $\phi = \psi\Theta$, то $\phi^* \hm= \Theta^*\psi^* = \Theta^{-1}\psi$ "--- теперь ортогональный (унитарный) оператор $\Theta^{-1}$ идет перед самосопряженным оператором $\psi$.
\end{note}

\begin{definition}
    Представление $\phi \in \mathcal{L}(V)$ в виде $\psi\Theta$ (или в виде $\Theta'\psi'$) с соответствующими требованиями из теоремы выше называется \textit{полярным разложением} $\phi$, а базисы $(\overline{e_1}, \dots, \overline{e_n})$ и $(\overline{g_1}, \dots, \overline{g_n})$ из доказательства теоремы "--- \textit{сингулярными базисами} $\phi$, причем эти базисы одинаковы в случаях $\psi\Theta$ и $\Theta'\psi'$.
\end{definition}

\begin{note}
    Геометрический смысл полярного разложения "--- представление оператора $\phi$ в виде композиции движения $\Theta$ и растяжения $\psi$ (с неотрицательными коэффициентами) вдоль нескольких взаимно ортогональных осей.
\end{note}

\begin{note}
    Можно показать, что если оператор $\phi \in \mathcal{L}(V)$ "--- невырожденный, то полярное разложение $\phi$ единственно.
\end{note}

\section{26. Приведение квадратичной формы в пространстве со скалярным произведением к главным осям. Одновременное приведение пары квадратичных форм к диагональному виду.}

\begin{theorem}[о приведении к главным осям]
    Пусть $V$ "--- евклидово (эрмитово) пространство,  $q \in \mathcal{Q}(V)$. Тогда в $V$ существует ортонормированный базис $e$, в котором $q$ имеет диагональный вид.
\end{theorem}

\begin{proof}
    Пусть $b \in \mathcal{B}^+(V)$ "--- $\theta$-линейная форма, полярная к $q$. Тогда $\exists \phi \in \mathcal{L}(V)$ такой, что $b(\overline{u}, \overline{v}) = (\phi(\overline{u}), \overline{v})$. При этом:
    \[(\phi(\overline{u}), \overline{v}) = b(\overline{u}, \overline{v}) = \overline{b(\overline{v}, \overline{u})} = \overline{(\phi(\overline{v}), \overline{u})} = (\overline{u}, \phi(\overline{v}))\]
    
    Значит, $\phi$ "--- самосопряженный, и в $V$ существует ортонормированный базис $e$, в котором $\phi$ диагонализуем. Тогда если $\phi \leftrightarrow_e A$, то $b \leftrightarrow_e A^T$ и $q \leftrightarrow_e A^T$, поэтому форма $q$ тоже имеет диагональную матрицу в базисе $e$.
\end{proof}

\begin{note}
    Напротив, если в ортонормированном базисе $e$ матрица формы $q$ диагональна, то и матрица оператора $\phi$ диагональна и, следовательно, задана однозначно собственными значениями $\phi$. Значит, диагональный вид $q$ в ортонормированном базисе определен однозначно.
\end{note}

\begin{theorem}
    Пусть $V$ "--- линейное пространство над $\mathbb{R}$ (над $\mathbb{C}$), $q_1, q_2 \hm{\in} \mathcal{Q}(V)$, и $q_2$ положительно определена. Тогда в $V$ существует такой базис $e$, в котором матрицы форм $q_1$ и $q_2$ диагональны.
\end{theorem}

\begin{proof}
    Пусть $b$ "--- $\theta$-линейная форма, полярная к $q_2$. Тогда $b$  можно объявить $b$ скалярным (эрмитовым скалярным) произведением на $V$. В полученном евклидовом (эрмитовом) пространстве форма $q_1$ приводится к главным осям в некотором ортонормированном базисе $e$. Поскольку базис $e$ "--- ортонормированный, то в этом же базисе $q_2$ имеет диагональный вид $E$.
\end{proof}

\begin{note}
    Требование положительной определенности в теореме существенно.
\end{note}

\section{27. Унитарные преобразования, их свойства. Канонический вид унитарного преобразования.}

\begin{definition}
    Оператор $\phi \in \mathcal{L}(V)$ называется \textit{ортогональным} (\textit{унитарным}), если $\forall \overline{u}, \overline{v} \in V: (\phi(\overline{u}), \phi(\overline{v})) = (\overline{u}, \overline{v})$.
\end{definition}

\begin{theorem}
    Пусть $\phi \in \mathcal{L}(V)$. Тогда оператор $\phi$ ортогонален (унитарен) $\hm\Leftrightarrow$ $\phi$ обратим и $\phi^{-1} = \phi^{*}$.
\end{theorem}

\begin{proof}
    По определению, $\phi$ ортогональнен (унитарен) $\hm{\Leftrightarrow}$ для любых векторов $\overline{u}, \overline{v} \in V$ выполнено $(\overline{u}, \overline{v}) = (\phi(\overline{u}), \phi(\overline{v})) = (\overline{u}, (\phi^*\phi)(\overline{v}))$. В силу единственности сопряженного оператора, это равносильно равенству $\phi^*\phi = \id^* = \id$. Это, в свою очередь, равносильно тому, что $\phi$ обратим и $\phi^{-1} \hm{=} \phi^{*}$.
\end{proof}

\begin{proposition}
    Пусть $\phi\in \mathcal{L}(V)$ "--- ортогональный (унитарный), $U \le V$. Тогда $U$ инвариантно относительно $\phi$ $\Leftrightarrow$ $U^\perp$ инвариантно относительно $\phi$.
\end{proposition}

\begin{proof}
    Поскольку $(U^\perp)^\perp = U$, то достаточно доказать импликацию $\ra$. Так как $U$ инвариантно относительно $\phi$, то $U^\perp$ инвариантно относительно $\phi^* = \phi^{-1}$, то есть $\phi^{-1}(U^\perp) \le U^\perp$. Но оператор $\phi$ биективен, поэтому $\phi^{-1}(U^\perp) = U^\perp$ и $\phi(U^\perp) = U^\perp$, откуда $U^\perp$ инвариантно относительно $\phi$.
\end{proof}

\begin{theorem}
    Пусть $V$ "--- эрмитово пространство, $\phi \in \mathcal{L}(V)$ "--- унитарный. Тогда в $V$ существует ортонормированный базис $e$, в котором матрица оператора $\phi$ диагональна с числами модуля $1$ на главной диагонали.
\end{theorem}

\begin{proof}
    Докажем диагонализуемость оператора $\phi$ в ортонормированном базисе индукцией по $n = \dim{V}$. База, $n = 1$, тривиальна. Пусть теперь $n > 1$. Поскольку у $\chi_\phi$ есть корень над $\Cm$, то у $\phi$ есть собственный вектор $\overline{e_0}$ длины $1$. Тогда $U := \langle\overline{e_0}\rangle^\perp$ инвариантно относительно $\phi$, поэтому можно расмотреть оператор $\phi|_{U} \in \mathcal{L}(V)$, который также является унитарным. По предположению индукции, в $U$ есть ортонормированный базис из собственных векторов, тогда объединение с $\overline{e_0}$ дает искомый базис в $V$.
    
    Покажем теперь, что все собственные значения оператора $\phi$ имеют модуль $1$. Действительно, если $\overline{v} \in V$, $\overline{v} \ne 0$ "--- собственный вектор со значением $\lambda$, то $(\overline{v}, \overline{v}) = (\phi(\overline{v}), \phi(\overline{v})) \hm= |\lambda|^2(\overline{v}, \overline{v}) \Rightarrow |\lambda| = 1$.
\end{proof}

\section{28. Тензоры типа $(p, q)$. Тензорное произведение тензоров. Координатная запись тензора, изменение координат при замене базиса, тензорный базис.}

Зафиксируем линейное пространство $V$ размерности $n$ над полем~$F$ и сопряженное к нему пространство $V^* = \mathcal{L}(V, F)$. Для любых $s, t \in \N \cup \{0\}$ положим $V^s := \underbrace{V \times \dots \times V}_{s}$, $(V^*)^t := \underbrace{V^* \times \dots \times V^*}_{t}$.

\begin{definition}
    \textit{Тензором типа $(p, q)$}, или \textit{$p$ раз контравариантныым и $q$ раз ковариантным тензором} называется полилинейное отображение $t: (V^*)^p \times V^q \rightarrow F$. Все тензоры типа $(p, q)$ образуют линейное пространство над $F$, обозначение "--- $\mathbb{T}^p_q(V)$ или $\mathcal{L}(\underbrace{V^*, \dots, V^*}_{p}, \underbrace{V, \dots, V}_{q}; F)$.
\end{definition}

\begin{note}
    Тензор задается однозначно своими значениями на всевозможных комбинациях аргументов из базиса в $V$ и базиса в $V^*$, то есть на $n^{p + q}$ наборах векторов.
\end{note}

\begin{example}
    Рассмотрим несколько тензоров различных типов:
    \begin{enumerate}
        \item Тензор\;типа\;$(0, 1)$\:"---\:это\;линейный функционал на $V$, поэтому $\mathbb{T}_1^0 = V^*$.
        
        \item Тензор\;типа\;$(1, 0)$\:"---\:это\;элемент пространства $V^{**} \cong V$, поэтому $\mathbb{T}^1_0 \cong V$, причем эти пространства можно отождествить в силу канонического изоморфизма.
        
        \item Тензор\;типа\;$(0, 2)$\:"---\:это\;билинейная форма на $V$, поэтому $\mathbb{T}^0_2 = \mathcal{B}(V)$.
        
        \item Тензор\;типа\;$(1, 1)$\:"---\:это\;билинейное отображение $t: V^* \times V \hm{\rightarrow} F$. Зафиксируем $\overline{v} \in V$, тогда $t_{\overline{v}}(f) \hm{:=} t(f, \overline{v})$ "--- линейный функционал на $V^*$, то есть $t_{\overline{v}} = \overline{u} \in V$. Тензору $t$ можно поставить в соответствие линейный оператор $\phi \in \mathcal{L}(V)$, $\phi(\overline{v}) = t_{\overline{v}} = \overline{u}$. Это соответствие линейно, поскольку $t$ линеен по второму аргументу, и обратимо: $\forall \phi \in \mathcal{L}(V): \phi \mapsto t$, где $t \in \mathbb{T}^1_1$ "--- такой, что $t(f, \overline{v}) = f(\phi(\overline{v}))$. Значит, $\mathbb{T}^1_1 \cong \mathcal{L}(V)$, причем эти пространства можно отождествить в силу канонического изоморфизма.
        
        \item Пусть $A$ "--- алгебра над $F$. Тогда умножение $\cdot: A \times A \rightarrow A$ "--- это билинейное отображение, $\cdot \in \mathcal{L}(A, A; A)$, и ему соответствует тензор $t \in \mathbb{T}^1_2(A)$ следующего вида:
        \[t(f, \overline{a_1}, \overline{a_2}) := f(\overline{a_1} \cdot \overline{a_2})\]
        
        Аналогично прошлому примеру, соответствие $\cdot \mapsto t$ линейно и обратимо, поэтому $\mathbb T^1_2(A) \cong \mathcal{L}(A \times A, A)$, причем эти пространства можно отождествить в силу канонического изоморфизма.
            
        \item Один из тензоров типа $(0, n)$, $n \in \mathbb{N}$, "--- это определитель.
    \end{enumerate}
\end{example}

\begin{definition}
    Пусть $t \in \mathbb{T}^p_q$, $t' \in \mathbb{T}^{p'}_{q'}$. Тогда \textit{тензорным произведением} тензоров $t$ и $t'$ называется тензор $t \otimes t' \in \mathbb{T}^{p + p'}_{q + q'}$ следующего вида:
    \[t \otimes t' (f_1, \dots, f_{p + p'}, \overline{v_1}, \dots, \overline{v_{q + q'}}) := t(f_1, \dots, f_p, \overline{v_1}, \dots, \overline{v_q})t'(f_{p+1}, \dots, f_{p + p'}, \overline{v_{q+1}}, \dots, \overline{v_{q+q'}})\]
\end{definition}

\begin{example} Рассмотрим несколько тензорных произведений:
    \begin{enumerate}
        \item Пусть $f_1, f_2 \in \mathbb{T}^0_1 = V^*$. Тогда $f_1 \otimes f_2 \in \mathbb{T}^0_2 = \mathcal{B}(V)$, причем $f_1 \otimes f_2(\overline{v_1}, \overline{v_2}) = f_1(\overline{v_1})f_2(\overline{v_2})$ и легко видеть, что $\rk{f_1 \otimes f_2} \le 1$.
        
        \item Пусть $g \in V^*$, $\overline{u} \in V$. Тогда $g \otimes \overline{u} \in \mathbb{T}^1_1 \hm{=} \mathcal{L}(V)$, и данному тензору соответствует оператор $\phi \in \mathcal{L}(V)$ такой, что $\phi(\overline{v}) = g(\overline{v})\overline{u}$. В частности, $\rk{\phi} \le 1$.
    \end{enumerate}
\end{example}

\begin{proposition} Тензорное произведение обладает следующими свойствами:
    \begin{enumerate}
        \item $\otimes$ линейно по обоим аргументам.
        \item $\otimes$ ассоциативно, но необязательно коммутативно.
    \end{enumerate}
\end{proposition}

\begin{proof}
    Оба свойства следуют непосредственно из формулы в определении тензорного произведения. В то же время, если, например, $t_1, t_2 \in T^0_1$, то:
    \begin{gather*}
        t_1 \otimes t_2 (\overline{v_1}, \overline{v_2}) = t_1(\overline{v_1})t_2(\overline{v_2})\\
        t_2 \otimes t_1 (\overline{v_1}, \overline{v_2}) = t_2(\overline{v_1})t_1(\overline{v_2})
    \end{gather*}
    
    Видно, что при $\dim{V} > 0$ можно подобрать такие тензоры и такие векторы, на которых значения выражений выше будут отличаться.
\end{proof}

\begin{note}
    Далее в записях будут применяться нижние и верхние индексы, не означающие возведение в степень. Они нужны исключительно для упрощения формул.
\end{note}

\begin{definition}
    Пусть $e = (e_1, \dots, e_n)$ "--- базис в $V$. \textit{Взаимным (биортогональным)} к $e$ базисом называется базис $e^* = (e^1, \dots, e^n)$ в $V^*$ такой, что:
    \[\forall i, j \in \{1, \dotsc, n\}: e^j(e_i) = e_i(e^j) = \delta^j_i = \left\{\begin{aligned}
    0, i \ne j\\
    1, i = j
    \end{aligned}\right.\]
    
    Будем обозначать через $v^j := e^j(\overline{v})$ $j$-ую координату вектора $\overline{v}$ в базисе $e$, а через $f_i := e_i(f)$ "--- $i$-ую координату функционала $f$ в базисе $e^*$.
\end{definition}

\begin{definition}
    Пусть $e$ и $e^*$ "--- взаимные базисы в $V$ и $V^*$, $t \in \mathbb{T}^p_q$. \textit{Координатами тензора $t$ в базисе $e$} называется набор из следующих величин:
    \[t^{i_1, \dots, i_p}_{j_1, \dots, j_q} = t(e^{i_1}, \dots, e^{i_p}, e_{j_1}, \dots, e_{j_q}),~i_1, \dots, i_p, j_1, \dots, j_q \in \{1, \dots, n\}\]
\end{definition}

\begin{note}
    Как уже было отмечено, тензор $t \in \mathbb T^p_q$ однозначно задается своими координатами в неотором базисе. Заметим, что тензор $t = e_{i_1} \otimes \dots \otimes e_{i_p} \otimes e^{j_1} \hm{\otimes} \dots \otimes e^{j_q} \hm{\in} \mathbb{T}^p_q$ имеет координаты следующего вида:
    \[t^{i'_1, \dots, i'_p}_{j'_1, \dots, j'_q} = \delta_{i_1}^{i'_1}\dots\delta_{i_p}^{i'_p}\delta_{j'_1}^{j_1}\dots\delta_{j'_q}^{j_q}\]
    
    Значит, произвольный тензор $t \in \mathbb{T}^p_q$ можно записать в таком виде:
    \[t = t^{i_1, \dots, i_p}_{j_1, \dots, j_q}e_{i_1} \otimes \dots \otimes e_{i_p} \otimes e^{j_1} \hm{\otimes} \dots \otimes e^{j_q}\]
    
    Равенство выше справедливо потому, что значения тензоров в левой и правой части совпадают на всех наборах вида $(e^{i_1}, \dots, e^{i_p}, e_{j_1}, \dots, e_{j_q})$.
\end{note}
    
\begin{note}
    Как уже было отмечено, тензоры вида $e_{i_1} \otimes \dots \otimes e_{i_p} \otimes e^{j_1} \hm{\otimes} \dots \otimes e^{j_q}$ "--- это порождающая система в $\mathbb T^p_q$. Более того, она линейно независима, поскольку для каждого тензора вида $e_{i_1} \otimes \dots \otimes e_{i_p} \otimes e^{j_1} \hm{\otimes} \dots \otimes e^{j_q}$ можно выбрать такой набор $(e^{i_1}, \dots, e^{i_p}, e_{j_1}, \dots, e_{j_q})$, который обнулит все тензоры системы кроме данного. Значит, эта система образует базис в пространстве $\mathbb{T}^p_q$.
\end{note}

\begin{theorem}
    Пусть $e$, $e'$ "--- базисы в $V$ такие, что $e_j' = a_j^ie_i$, $e^k = a^k_ie'^i$. Тогда преобразование координат тензора $t \in \mathbb T^p_q$ при замене базиса имеет следующий вид:
    \[t^{i_1, \dots, i_p}_{j_1, \dots, j_q} = a^{i_1}_{i_1'}\dots a^{i_p}_{i'_p}b^{j_1'}_{j_1}\dots b^{j'_q}_{j_q}t'^{i'_1, \dots, i'_p}_{j'_1, \dots, j'_q}\]
\end{theorem}

\begin{proof}
    Для простоты выполним проверку в случае, когда $t \in \mathbb{T}^1_1$, поскольку в общем случае рассуждение аналогично:
    \[t = t^i_je_i \otimes e^j = t^{i'}_{j'}e'_{i'} \otimes e'^{j'} = t^{i'}_{j'}(a_{i'}^ie_i) \otimes (b^{j'}_je^{j}) = t^{i'}_{j'}a^i_{i'}b^{j'}_j e_i \otimes e^j\]
    
    Получено разложение тензора $t$ по базису $e$ двумя способами, поэтому $t^i_j = t'^{i'}_{j'}a^i_{i'}b^{j'}_j$.
\end{proof}

\section{29. Алгебраические операции над тензорами (перестановка индексов, свертка). Симметричные и кососимметричные тензоры. Операторы симметрирования и альтернирования и их свойства.}

\begin{definition}
    \textit{Сверткой} тензора $t \in \mathbb{T}^p_q$ по индексам $i_p, j_q$ называется тензор $t' \in \mathbb T^{p-1}_{q-1}$ с координатами следующего вида:
    \[\widetilde t^{i_1, \dots, i_{p-1}}_{j_1, \dots, j_{q-1}} = t^{i_1, \dots, i_{p - 1}, i}_{j_1, \dots, j_{q-1}, i}\]
    
    Свертка по другим парам из верхнего и нижнего индексов определяется аналогично.
\end{definition}

\begin{example}
    Рассмотрим несколько примеров свертки:
    \begin{enumerate}
        \item Пусть $\overline{v} \in V$, $u \in V^*$. Тогда свертка тензора $u \otimes \overline{v}$ "--- это скаляр $u(\overline{v})$.
        \item Пусть $b \in \mathcal{B}(V)$ "--- тензор с координатами $b_{ij}$, $\overline{u}, \overline{v} \in V$. Тогда скаляр $b(\overline{u}, \overline{v}) = u^ib_{ij}v^j$ получается как \textit{двойная}, или \textit{полная}, свертка тензора $\overline{u} \otimes b \otimes \overline{v}$.
        \item Пусть $\phi \in \mathcal{L}(V)$ "--- тензор с координатами $\phi^i_j$, $\overline{v} \in V$. Тогда вектор $\phi(\overline{v})$ имеет координаты $\phi^i_jv^j$.
        \item Пусть $\phi, \psi \in \mathcal{L}(V)$ "--- тензоры с координатами $\phi^i_j, \psi^k_l$. Тогда тензор $\phi \circ \psi$ имеет координаты $\phi^i_j\psi^j_k$.
        \item Пусть $V$ "--- евклидово пространство, в нем введено скалярное произведение, или \textit{метрический тензор}, с координатами $g_{ij}$. Тогда канонический изоморфизм между $V$ и $V^*$ осуществляется сопоставлением $v^i \mapsto v^ig_{ij}$, называемым \textit{опусканием индекса}. На $V^*$ тоже можно задать скалярное произведение как тензор с координатами $g^{ij}$, тоже называемый метрическим тензором, позволяющий, наоборот, поднимать индексы. Можно также показать, что $g_{ij}g^{ik} = \delta^k_j$.
        \item Пусть $\phi \in \mathcal{L}(V)$ "--- тензор с координатами $\phi^i_j$. Если в пространстве $V$ задано скалярное произведение с координатами $g_{ij}$, то сопоставление $\phi^i_j \mapsto \phi^i_jg_{ik}$ осуществляет это изоморфизм между $\mathcal{L}(V)$ и $\mathcal{B}(V)$.
    \end{enumerate}
\end{example}

\begin{definition}
    Пусть $t \in \mathbb{T}^p_q$. Тензор $t$ называется \textit{симметричным по первым двум координатам}, если для любых функционалов $f_1, \dots, f_p \in V^*$ и векторов $\overline{v_1}, \dots, \overline{v_q} \in V$ выполнено $t(f_1, f_2, \dots, f_p, \overline{v_1}, \dots, \overline{v_q}) = t(f_2, f_1, \dots, f_p, \overline{v_1}, \dots, \overline{v_q})$.
\end{definition}

\begin{note}
    Легко видеть, что $t$ симметричен по первым двум верхним индексам $\lra$ его координаты симметричны по первым двум верхним индексам. Симметричность по другим наборам координат одного типа определяется аналогично.
\end{note}

\begin{definition}
    Пусть $t \in \mathbb{T}^p_0$, $\sigma \in S_p$. Будем обозначать через $g_\sigma(t)$ такой тензор $g \in \mathbb{T}^p_0$, что $\forall f_1, \dotsc, f_p \in V^*: g(f_1, \dots, f_p) = t(f_{\sigma(1)}, \dots, f_{\sigma(p)})$.
\end{definition}

\begin{note}
    Пусть $e$ "--- базис в $V$. Если $t$ имеет в базисе $e$ координаты $t^{i_1, \dots, i_p}$, то $g_\sigma(t)$ в этом же базисе имеет координаты $t^{i_{\sigma(1)}, \dots, i_{\sigma(p)}}$.
\end{note}

\begin{definition}
    Тензор $t \in \mathbb{T}^p_0$ называется \textit{симметричным}, если $\forall \sigma \in S_p: g_\sigma(t) = t$. Такие тензоры образуют подпространство в $\mathbb{T}^p_0$, обозначаемое через $\mathbb{ST}^p$.
\end{definition}

\begin{definition}
    \textit{Симметризацией} тензора $t \in \mathbb{T}^p_0$ называется следующий тензор:
    \[s(t) := \frac1{p!}\sum_{\sigma \in S_p}g_\sigma(t) \in \mathbb{T}^p_0\]
    
    Симметризация определена, если $\cha{F} \nmid p$.
\end{definition}

\begin{proposition} Симметризация обладает следующими свойствами:
    \begin{enumerate}
        \item Для любого тензора $t \in \mathbb{T}^p_0$ выполнено $s(t) \in \mathbb{ST}^p$.
        \item Если $t \in \mathbb{ST}^p$, то $s(t) = t$.
        \item $\im{s} = \mathbb{ST}^p$.
    \end{enumerate}
\end{proposition}

\begin{definition}
    Тензор $t \in \mathbb{T}^p_0$ называется \textit{кососимметричным}, если $\forall \sigma \in S_p: g_\sigma(t) \hm= \sgn\sigma\cdot t$. Такие тензоры образуют подпространство в $\mathbb{T}^p_0$, обозначаемое через $\Lambda^p$.
\end{definition}

\begin{definition}
    \textit{Альтернированием} тензора $t \in \mathbb{T}^p_0$ называется следующий тензор:
    \[a(t) := \frac1{p!}\sum_{\sigma \in S_p}\sgn\sigma\cdot g_\sigma(t) \in \mathbb{T}^p_0\]
    
    Альтернирование определено, если $\cha{F} \nmid p$.
\end{definition}

\begin{proposition} 
    Альтернирование обладает следующими свойствами:
    \begin{enumerate}
        \item Для любого тензора $t \in \mathbb{T}^p_0$ выполнено $a(t) \in \Lambda^p$.
        \item Если $t \in \Lambda^p$, то $a(t) = t$.
        \item $\im{a} = \Lambda^p$.
    \end{enumerate}
\end{proposition}

\begin{proof}
    Доказательство аналогично симметричному случаю.
\end{proof}